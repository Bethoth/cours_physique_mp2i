% Set up the document's format to A4 and the font's size to 12pt.
\documentclass[a4paper,12pt]{report}

% Set up the input's encoding to UTF-8, the document's font and language to T1 (adapted to french) and french (the grammar linter uses this parameter).
\usepackage[utf8]{inputenc}
\usepackage[T1]{fontenc}
\usepackage[british,french]{babel}

% Set up the document's margins.
\usepackage{geometry}
\geometry{hmargin=1.5cm,vmargin=1.5cm}

% Set up the document's title, author and date.
\title{Physique -- MP2I}
\author{Romain Bricout}
\date{\today}

% The three main maths packages. They are used for a lot of things.
\usepackage{amssymb,amsmath}
\usepackage{mathtools}

% Useful to create nice and easy signs or variations tables.
\usepackage{tkz-tab}

% Useful to create any kind of visual representation (graph functions, illustrate geometry problems, etc)
\usepackage{tikz}
\usepackage{pgfplots}
\pgfplotsset{compat=newest}
\usepgfplotslibrary{fillbetween}
\usetikzlibrary{patterns,patterns.meta,angles,quotes,arrows,arrows.meta,bending,decorations.markings,babel,decorations.pathreplacing,calligraphy,shapes.misc}
\tikzset{cross/.style={cross out, draw=black, minimum size=2*(#1-\pgflinewidth), inner sep=0pt, outer sep=0pt},
%default radius will be 1pt.
cross/.default={1pt}}

% Allows to edit the itemize environment's default item document-wide.
\usepackage{enumitem}

% Allows to define \notfoo or \nfoo (not recommended) in order for \not\foo to work as wished.
\usepackage{newtxmath}

% Makes the table of contents clickable and gives useful commands for links in general.
\usepackage{hyperref}
\hypersetup{colorlinks=false,linktoc=all}

% Gives the llbracket and rrbracket commands for integer intervals.
\usepackage{stmaryrd}

% Useful to insert nice-looking quotes.
\usepackage{epigraph}

% Allows to insert chapter-specific table of contents.
\usepackage{minitoc}
\mtcselectlanguage{french}
\setcounter{minitocdepth}{6}

% Useful when units are needed.
\usepackage{siunitx}
\sisetup{
locale=FR,
detect-all,
inter-unit-product=\ensuremath{\cdot},
list-final-separator={et},
list-pair-separator={et},
range-phrase={\ensuremath{\xleftrightarrow{}}},
exponent-product=\ensuremath{\cdot},
per-mode=power-positive-first
}

\usepackage[thmmarks]{ntheorem}
\makeatletter
\let\old@thm\@thm
\usepackage[lowercase]{theoremref}
\def\@thm#1#2#3{\def\thmref@currname{#3}\old@thm{#1}{#2}{#3}}
\makeatother

% Allows whiteboard digits with \mathds
\usepackage{dsfont}

\usepackage{needspace}

% Useful for better-looking oneline fractions
\usepackage{nicefrac}

% Set up the horizontal space before the first line of a new paragraph to 2em and the vertical space between two paragraphs to 1em.
\setlength{\parindent}{0pt}
\setlength{\parskip}{1em}

% Adds 0.5em to the vertical space between two lines in an align environment. It looks better.
\addtolength{\jot}{0.5em}

% Allows align environment to break if it's too long to fit in the page where it began.
\allowdisplaybreaks[1]

% Trick to make semicolons considered like relation operators (such as =) and therefore being equidistantly spaced from the two numbers around it.
\mathcode`;=\numexpr\mathcode`;-"3000

% Commands for size-adaptative parentheses, brackets, curly brackets, absolute value and magnitude.
\newcommand{\paren}[1]{\left(#1\right)} % (x)
\newcommand{\croch}[1]{\left[#1\right]} % [x]
\newcommand{\accol}[1]{\left\lbrace#1\right\rbrace} % {x}
\newcommand{\abs}[1]{\left\lvert#1\right\rvert} % |x|
\newcommand{\norme}[1]{\left\|#1\right\|} % ||x||
\newcommand{\floor}[1]{\left\lfloor#1\right\rfloor} % ⌊x⌋
\newcommand{\ceil}[1]{\left\lceil#1\right\rceil} % ⌈x⌉

% Commands for size-adaptative intervals and integer intervals. The commands' roots are "interv" and "interventier" and the added e or i at the end mean "excluded" and "included" respectively.
\newcommand{\intervii}[2]{\left[#1;#2\right]} % [a;b]
\newcommand{\intervee}[2]{\left]#1;#2\right[} % ]a;b[
\newcommand{\intervie}[2]{\left[#1;#2\right[} % [a;b[
\newcommand{\intervei}[2]{\left]#1;#2\right]} % ]a;b]
\newcommand{\interventierii}[2]{\left\llbracket#1;#2\right\rrbracket} % non-ASCII characters needed
\newcommand{\interventieree}[2]{\left\rrbracket#1;#2\right\llbracket} % non-ASCII characters needed
\newcommand{\interventierie}[2]{\left\llbracket#1;#2\right\llbracket} % non-ASCII characters needed
\newcommand{\interventierei}[2]{\left\rrbracket#1;#2\right\rrbracket} % non-ASCII characters needed

% Commands for usually used sets.
\newcommand{\N}{\mathbb{N}} % natural integers
\newcommand{\Ns}{\mathbb{N}^*}

\newcommand{\Z}{\mathbb{Z}} % relative integers
\newcommand{\Zp}{\mathbb{Z}_+}
\newcommand{\Zs}{\mathbb{Z}^*}
\newcommand{\Zps}{\mathbb{Z}_+^*}

\newcommand{\D}{\mathbb{D}} % decimal numbers
\newcommand{\Dp}{\mathbb{D}_+}
\newcommand{\Dm}{\mathbb{D}_-}
\newcommand{\Ds}{\mathbb{D}^*}
\newcommand{\Dps}{\mathbb{D}_+^*}
\newcommand{\Dms}{\mathbb{D}_-^*}

\newcommand{\Q}{\mathbb{Q}} % rational numbers
\newcommand{\Qp}{\mathbb{Q}_+}
\newcommand{\Qm}{\mathbb{Q}_-}
\newcommand{\Qs}{\mathbb{Q}^*}
\newcommand{\Qps}{\mathbb{Q}_+^*}
\newcommand{\Qms}{\mathbb{Q}_-^*}

\newcommand{\R}{\mathbb{R}} % real numbers
\newcommand{\Rp}{\mathbb{R}_+}
\newcommand{\Rm}{\mathbb{R}_-}
\newcommand{\Rs}{\mathbb{R}^*}
\newcommand{\Rps}{\mathbb{R}_+^*}
\newcommand{\Rms}{\mathbb{R}_-^*}
\newcommand{\Rb}{\overline{\mathbb{R}}}

\newcommand{\C}{\mathbb{C}} % complex numbers
\newcommand{\Cs}{\mathbb{C}^*}

\newcommand{\U}{\mathbb{U}} % complex numbers whose modulus is 1
\renewcommand{\H}{\mathbb{H}} % quaternions (\H normally prints slanted quotation marks)
\renewcommand{\O}{\mathbb{O}} % octonions (\O normally prints a slashed capital o : Ø)
\newcommand{\M}{\mathcal{M}} % matrices
\newcommand{\GL}{\mathrm{GL}} % invertible matrices
\renewcommand{\S}{\mathcal{S}} % solutions of an equation (\S normally prints a silcrow : §)

\renewcommand{\P}[1]{\mathcal{P}\paren{#1}} % subsets of a set
\newcommand{\F}[2]{\mathcal{F}\paren{#1,#2}} % functions from 1 to 2
\newcommand{\V}[1]{\mathcal{V}\paren{#1}} % neighborhood of a number

% Redefines \Re and \Im to print Re and Im (the same way as ln or lim) instead of fraktur R and I which don't look nice and are less readable.
\renewcommand{\Re}{\operatorname{Re}}
\renewcommand{\Im}{\operatorname{Im}}

% Command to print an upright e for the exponential instead of a slanted e and put the exponent.
\newcommand{\e}[1]{\mathrm{e}^{#1}}

% Command to print the imaginary i with a little space on the right. This way, the exponents don't look confusing. \i normally prints a dotless i.
\renewcommand{\i}{i\mkern1mu}

% Redefines \vec such that the arrow covers the whole name of the vector.
%\renewcommand{\vec}[1]{\overrightarrow{#1}}

% Commands for 2D and 3D vectors' coordinates
\newcommand{\dcoords}[2]{\begin{pmatrix}#1\\#2\end{pmatrix}}
\newcommand{\tcoords}[3]{\begin{pmatrix}#1\\#2\\#3\end{pmatrix}}

% Redefines binom to print nicer parentheses around the numbers.
\renewcommand{\binom}[2]{\begin{pmatrix}#2\\#1\end{pmatrix}}

% Command for a QED black square. It automatically prints a whitespace before the square such that it looks nice.
\newcommand{\cqfd}{\text{ }\blacksquare}

% Commands with more explicit names for the best way to express divisibility (mid and nmid).
\newcommand{\divise}{\mid}
\newcommand{\notdivise}{\nmid}

% Commands that do the exact same thing but with explicit names for a complex number's conjugate and an event's negation in probability.
\newcommand{\conj}[1]{\overline{#1}}

% Command for a size-adaptative middle bar meaning "such that" (with spacing around it in order to look nice).
\newcommand{\tq}{\;\middle|\;}

% Command with an explicit name for the scalar product.
\newcommand{\scal}{\cdot}
\newcommand{\vecto}{\operatorname{_\wedge}}

% Shortcut for forcing displaystyle in inline mode.
\newcommand{\ds}{\displaystyle}

% Make the not version of implies, impliedby and iff look nice.
\newcommand{\notimplies}{\centernot{\imp}}
\newcommand{\notimpliedby}{\centernot{\impr}}
\newcommand{\notiff}{\centernot{\ssi}}

% Shortcut for P(event).
\newcommand{\proba}[1]{P\paren{#1}}

% More explicit names for land (logical and) and lor (logical or).
\newcommand{\et}{\land}
\newcommand{\ou}{\lor}
\newcommand{\non}{\lnot}

% Explicitly named environment for tkz-tab tables. Automatically centers the table and handles the tikzpicture environment.
\newenvironment{tkz}
{
\begin{center}
\begin{tikzpicture}
}
{
\end{tikzpicture}
\end{center}
}

% More explicitly named commands for the creation of tkz-tab tables.
\newcommand{\tableauinit}[2]{\tkzTabInit{#1}{#2}}
\newcommand{\tableausignes}[1]{\tkzTabLine{#1}}
\newcommand{\tableauvariations}[1]{\tkzTabVar{#1}}

% Shortcut for the curve and the domain of the given function.
\newcommand{\graphe}[1]{\Gamma_{#1}}
\newcommand{\ensembledef}[1]{\mathcal{D}_{#1}}

% Various environments that create boxes. Each one is one type of thing (example, proof, etc). Each type has its own automatic counter.
\theoremstyle{break}
\theorembodyfont{\upshape}
\theoremheaderfont{\itshape}
\theoremprework{\bigskip\needspace{\baselineskip}\color{green}\hrule\color{black}}
\theorempostwork{\bigskip}
\newtheorem{rem}{Remarque}[chapter]

\theoremstyle{break}
\theorembodyfont{\upshape}
\theoremheaderfont{\itshape}
\theoremprework{\bigskip\needspace{\baselineskip}\color{green}\hrule\color{black}}
\theorempostwork{\bigskip}
\newtheorem{ex}{Exemple}[chapter]

\theoremstyle{break}
\theorembodyfont{\upshape}
\theoremheaderfont{\itshape}
\theoremprework{\bigskip\needspace{\baselineskip}\color{green}\hrule\color{black}}
\theorempostwork{\bigskip}
\newtheorem{rappel}{Rappel}[chapter]

\theoremstyle{break}
\theorembodyfont{\upshape}
\theoremheaderfont{\itshape}
\theoremprework{\bigskip\needspace{\baselineskip}\color{brown}\hrule\color{black}}
\theorempostwork{\bigskip}
\newtheorem{oubli}{Oubli}[chapter]

\theoremstyle{break}
\theorembodyfont{\upshape}
\theoremheaderfont{\normalfont\bfseries}
\theoremprework{\bigskip\needspace{\baselineskip}\color{blue}\hrule\color{black}}
\theorempostwork{\bigskip}
\newtheorem{defi}{Définition}[chapter]

\theoremstyle{break}
\theorembodyfont{\upshape}
\theoremheaderfont{\normalfont\bfseries}
\theoremprework{\bigskip\needspace{\baselineskip}\color{blue}\hrule\color{black}}
\theorempostwork{\bigskip}
\newtheorem{defprop}{Définition/Proposition}[chapter]

\theoremstyle{break}
\theorembodyfont{\upshape}
\theoremheaderfont{\normalfont\bfseries}
\theoremprework{\bigskip\needspace{\baselineskip}\color{blue}\hrule\color{black}}
\theorempostwork{\bigskip}
\newtheorem{nota}{Notation}[chapter]

\theoremstyle{break}
\theorembodyfont{\itshape}
\theoremheaderfont{\normalfont\bfseries}
\theoremprework{\bigskip\needspace{\baselineskip}\color{red}\hrule\color{black}}
\theorempostwork{\bigskip}
\newtheorem{theo}{Théorème}[chapter]

\theoremstyle{break}
\theorembodyfont{\itshape}
\theoremheaderfont{\normalfont\bfseries}
\theoremprework{\bigskip\needspace{\baselineskip}\color{red}\hrule\color{black}}
\theorempostwork{\bigskip}
\newtheorem{prop}{Proposition}[chapter]

\theoremstyle{break}
\theorembodyfont{\itshape}
\theoremheaderfont{\normalfont\bfseries}
\theoremprework{\bigskip\needspace{\baselineskip}\color{red}\hrule\color{black}}
\theorempostwork{\bigskip}
\newtheorem{cor}{Corollaire}[chapter]

\theoremstyle{break}
\theorembodyfont{\upshape}
\theoremheaderfont{\normalfont\bfseries}
\theoremprework{\bigskip\needspace{\baselineskip}\color{purple}\hrule\color{black}}
\theorempostwork{\bigskip}
\newtheorem{meth}{Méthode}[chapter]

\theoremstyle{nonumberbreak}
\theorembodyfont{\upshape}
\theoremheaderfont{\itshape}
\theoremsymbol{\ensuremath{\cqfd}}
\theoremprework{\bigskip\needspace{\baselineskip}\color{yellow}\hrule\color{black}}
\theorempostwork{\bigskip}
\newtheorem{dem}{Démonstration}[chapter]

% Commands to make proofs easier to write
\newcommand{\impdir}{\fbox{\(\imp\)}~}
\newcommand{\imprec}{\fbox{\(\impr\)}~}
\newcommand{\incdir}{\fbox{\(\subset\)}~}
\newcommand{\increc}{\fbox{\(\supset\)}~}
\newcommand{\unicite}{\fbox{unicité}~}
\newcommand{\existence}{\fbox{existence}~}

\renewcommand{\to}{\longrightarrow}
\renewcommand{\mapsto}{\longmapsto}

\newcommand{\fonction}[5]{\begin{array}[t]{cccl}#1 : & #2 & \to & #3 \\ & #4 & \mapsto & #5\end{array}}
\newcommand{\fonctionlambda}[4]{\begin{array}[t]{ccl}#1 & \to & #2 \\ #3 & \mapsto & #4\end{array}}

\renewcommand{\subset}{\subseteq}
\renewcommand{\supset}{\supseteq}

\renewcommand{\leq}{\leqslant}
\renewcommand{\geq}{\geqslant}

\newcommand{\pinf}{+\infty}
\newcommand{\minf}{-\infty}

\newcommand{\id}[1]{\mathrm{id}_{#1}}

\renewcommand{\phi}{\varphi}
\renewcommand{\epsilon}{\varepsilon}

\newcommand{\ind}[1]{\mathds{1}_{#1}}

\newcommand{\iR}{\i\R}

\newcommand{\tcheby}[2]{T_{#1}\paren{#2}}
\newcommand{\utcheby}[2]{U_{#1}\paren{#2}}

\mathcode`l="8000
\begingroup
\makeatletter
\lccode`\~=`\l
\DeclareMathSymbol{\lsb@l}{\mathalpha}{letters}{`l}
\lowercase{\gdef~{\ifnum\the\mathgroup=\m@ne \ell \else \lsb@l \fi}}%
\endgroup

\newcommand{\ensvide}{\emptyset}

\newcommand{\rond}{\circ}

\newcommand{\union}{\cup}
\newcommand{\inter}{\cap}
\newcommand{\bigunion}{\bigcup}
\newcommand{\biginter}{\bigcap}

\newcommand{\ssi}{\iff}
\newcommand{\imp}{\implies}
\newcommand{\impr}{\impliedby}

\newcommand{\excluant}{\setminus}

\newcommand{\littletaller}{\mathchoice{\vphantom{\big|}}{}{}{}}
\newcommand{\restr}[2]{{
\left.\kern-\nulldelimiterspace#1\littletaller\right|_{#2}
}}
\newcommand{\corestr}[2]{{
\left.\kern-\nulldelimiterspace#1\littletaller\right|^{#2}
}}
\newcommand{\restrbar}[1]{{
\left.\kern-\nulldelimiterspace#1\littletaller\right|
}}

\newcommand{\rel}{\mathcal{R}}

\newcommand{\classesdequiv}[1]{\nicefrac{#1}{\sim}}

\newcommand{\majo}[1]{\mathrm{majorants}\paren{#1}}
\newcommand{\mino}[1]{\mathrm{minorants}\paren{#1}}

\newcommand{\ensdiv}[1]{\operatorname{div}#1}

\newcommand{\E}[1]{\times 10^{#1}}

\newcommand{\siecle}[1]{\textsc{#1}\ieme~}

\usepackage{derivative}
\derivset{\pdv}[delims-eval=.)]

\newcommand{\moy}[1]{\left\langle#1\right\rangle}

\newcommand{\cte}{\mathrm{cte}}
\newcommand{\ext}{\mathrm{ext}}
\newcommand{\inte}{\mathrm{int}}
\newcommand{\ncons}{\mathrm{nc}}
\newcommand{\cons}{\mathrm{c}}
\newcommand{\eq}{\mathrm{eq}}
\newcommand{\ther}{\mathrm{th}}
\newcommand{\univ}{\mathrm{univ}}
\newcommand{\maxi}{\mathrm{max}}
\newcommand{\ie}{\textit{i.e.} }
\newcommand{\cf}{\textit{cf.} }

\usepackage{diagbox}

\usepackage{witharrows}

\newcommand{\dif}{\mathrm{d}}

\DeclareSIUnit{\cal}{\text{cal}}
\DeclareSIUnit{\Longueur}{L}
\DeclareSIUnit{\Temps}{T}
\DeclareSIUnit{\Masse}{M}
\DeclareSIUnit{\Temperature}{\Theta}
\DeclareSIUnit{\IntensiteElec}{I}
\DeclareSIUnit{\Quantite}{N}
\DeclareSIUnit{\IntensiteLumi}{J}

\newcommand{\guillemets}[1]{\og #1 \fg{}}

\usepackage{abstract}
\addto\captionsfrench{\renewcommand{\abstractname}{\Large Introduction}}

\hypersetup{
pdftitle=Physique MP2I,
pdfauthor=Romain Bricout
}

\newenvironment{Tabular}[2][1]{\def\arraystretch{#1}\tabular{#2}}{\endtabular}

\makeatletter
\@addtoreset{chapter}{part}
\makeatother

\newcommand{\supinf}{\lessgtr}

\usepackage[europeanresistors,americaninductors,straightvoltages,siunitx,RPvoltages]{circuitikz}

\newenvironment{circuit}{\begin{center}\begin{circuitikz}}{\end{circuitikz}\end{center}}

\usepackage{bclogo}

\newcommand{\attention}{\bcattention\;\;}

\setcounter{secnumdepth}{3}

\newcommand{\note}[1]{\textbf{\(\star\star\) #1 \(\star\star\)}}
\newcommand{\cad}{c'est-à-dire }
\newcommand{\Cad}{C'est-à-dire }

\begin{document}
\renewcommand{\labelitemi}{\(\bullet\)}
\renewcommand{\labelenumi}{(\arabic{enumi})}

\everymath{\ds}

\maketitle

\begin{abstract}
Ce document réunit l'ensemble de mes cours de Physique de MP2I. Le professeur était M. Jacob. J'ai adapté certaines formulations me paraissant floues ou ne me plaisant pas mais le contenu pur des cours est strictement équivalent. Le document est organisé selon la hiérarchie suivante : partie, chapitre, I), 1), a). Les parties sont celles du programme : thermodynamique, électricité, mécanique, ... Leur ordre dans ce document est arbitraire et ne reflète pas l'ordre de traitement des chapitres durant l'année.

Les éléments des tables des matières initiale et présentes au début de chaque chapitre sont cliquables (amenant directement à la partie cliquée).

Code couleur des vecteurs (et flèches en général) : \begin{itemize}
\item noir : légendes ou mouvements ;
\item violet : champs (par exemple le champ de pesanteur \(\vec{g}\)) ;
\item rouge : forces ;
\item bleu : vecteurs unitaires ;
\item vert : vecteurs vitesses. \\
\end{itemize}
\end{abstract}

\dominitoc\tableofcontents

\part{Méthodes \& divers}

\chapter{Dimensions et unités}

\minitoc

\section{Dimension d'une grandeur physique}

\subsection{Définition}

La dimension d'une grandeur physique traduit la nature physique de cette grandeur.

Deux grandeurs de même dimension sont dites homogènes. Si elles sont homogènes, elles peuvent être comparées.

Par exemple : \begin{itemize}
\item \(v_1=\SI{10}{\meter\per\second}\) et \(v_2=\SI{30}{\meter\per\second}\) sont des vitesses. Elles sont homogènes. On peut donc écrire \(v_2>v_1\) et \(v_2=3v_1\).
\item \(m=\SI{1000}{\kilo\gram}\) est une masse donc elle n'est pas homogène avec \(v_1\). On ne peut donc pas écrire \(m>v_1\).
\end{itemize}

\subsection{Les sept dimensions fondamentales}

Il existe sept dimensions fondamentales indépendantes :

\begin{center}
\begin{Tabular}[2]{c|c|c}
Dimension & Symbole & Unité \\
\hline
Longueur & \(\unit{\Longueur}\) & mètre (\(\unit{\metre}\)) \\
Temps & \(\unit{\Temps}\) & seconde (\(\unit{\second}\)) \\
Masse & \(\unit{\Masse}\) & kilogramme (\(\unit{\kilo\gram}\)) \\
Température & \(\unit{\Temperature}\) & kelvin (\(\unit{\kelvin}\)) \\
Intensité électrique & \(\unit{\IntensiteElec}\) & ampère (\(\unit{\ampere}\)) \\
Quantité de matière & \(\unit{\Quantite}\) & mole (\(\unit{\mole}\)) \\
Intensité lumineuse & \(\unit{\IntensiteLumi}\) & candela (\(\unit{\candela}\))
\end{Tabular}
\end{center}

Toutes les autres dimensions se déduisent des sept dimensions fondamentales.

Exemples : \begin{itemize}
\item on a \(\text{vitesse}=\dfrac{\text{distance}}{\text{temps}}\) donc vitesse : \(\unit{\Longueur\per\Temps}\) ; \\

\item d'après la deuxième loi de Newton, on a \(\sum\vec{F}=m\vec{a}\). Or \(m:\unit{\Masse}\) et \(a:\unit{\Longueur\per\Temps\squared}\) donc \(F:\unit{\Masse\Longueur\per\Temps\squared}\). \\

\item Déterminons la dimension de la tension électrique.

On a \(U=RI\) (loi d'Ohm) mais on ne connaît pas la dimension de \(R\) donc elle est inutile. On a aussi la puissance électrique \(P=UI\).

Or \(E=P\adif{t}\) et \(E=\dfrac{1}{2}mv^2\) donc \(E:\unit{\Masse\Longueur\squared\per\Temps\squared}\).

Donc \(P=\dfrac{E}{\adif{t}}:\dfrac{\unit{\Masse\Longueur\squared\per\Temps\squared}}{\unit{\Temps}}=\unit{\Masse\Longueur\squared\per\Temps\cubed}\).

Donc \(U=\dfrac{P}{I}:\dfrac{\unit{\Masse\Longueur\squared\per\Temps\cubed}}{\unit{\IntensiteElec}}=\unit{\Masse\Longueur\squared\per\Temps\cubed\per\IntensiteElec}\).

On en déduit l'unité : volt (\(\unit{\volt}=\unit{\kilo\gram\meter\squared\per\second\cubed\per\ampere}\)). \\
\end{itemize}

Un nombre (comme \(1\), \(\pi\) ou \(j\)) n'a pas de dimension.

\subsection{Équations dimensionnelles}

Si \(A\) est une grandeur, on note \(\croch{A}\) sa dimension.

Soient \(A,B,C,D\) des grandeurs.

Si \(A=B\) alors \(\croch{A}=\croch{B}\).

Si \(A+B=C+D\) alors \(\croch{A}=\croch{B}=\croch{C}=\croch{D}\) : tous les termes d'une somme sont homogènes.

Si \(A=BC\) alors \(\croch{A}=\croch{B}\times\croch{C}\).

Par exemple, considérons l'égalité suivante : \[a=\dfrac{v^2}{R^2}\] avec \begin{description}
\item \(a\) désignant une accélération ;
\item \(v\) désignant une vitesse ;
\item \(R\) désignant une distance. \\
\end{description}

On a : \[\croch{R}=\unit{\Longueur}\qquad\croch{v}=\unit{\Longueur\per\Temps}\qquad\croch{a}=\unit{\Longueur\per\Temps\squared}.\]

On remplace : \[\croch{\dfrac{v^2}{R^2}}=\dfrac{\croch{v}^2}{\croch{R}^2}=\dfrac{\unit{\Longueur\squared\per\Temps\squared}}{\unit{\Longueur\squared}}=\unit{\per\Temps\squared}.\]

Donc \(\dfrac{v^2}{R^2}\) n'est pas une accélération. Donc \(a=\dfrac{v^2}{R^2}\) est faux.

Il faut donc procéder régulièrement à l'analyse dimensionnelles de ses résultats intermédiaires.

\section{Les unités}

\subsection{Le Système International d'unités}

Le Système International d'unités donne les unités associées aux dimensions fondamentales : le mètre (\(\unit{\meter}\)), le kilogramme (\(\unit{\kilo\gram}\)), la seconde (\(\unit{\second}\)), l'ampère (\(\unit{\ampere}\)), la candela (\(\unit{\candela}\)), la mole (\(\unit{\mole}\)) et le kelvin (\(\unit{\kelvin}\)), et deux pseudo-unités sans dimension : le radian (\(\unit{\radian}\)) pour les angles et le stéradian (\(\unit{\steradian}\)) pour les angles solides.

Il a été adopté en 1960.

\subsection{Les étalons de mesure}

Toute mesure d'une grandeur se fait en comparaison à une grandeur de référence : l'étalon.

\subsubsection{L'étalon de masse}

C'est un cylindre de platine iridié conservé au bureau international des poids et mesures (Saint-Cloud), jusqu'en 2019. Depuis, un kilogramme est défini en fixant certaines constantes physiques.

\subsubsection{L'étalon de durée}

Une seconde correspond à \(\num{9192631770}\) périodes de la transition entre deux niveaux hyperfins de l'atome de Césium 133.

\subsubsection{L'étalon de longueur}

Un mètre est la distance parcourue dans le vide par la lumière pendant une durée de \(\dfrac{1}{\num{299792458}}\unit{\second}\).

Conséquence : \[c=\SI{299792458}{\metre\per\second}\]

C'est une valeur exacte qui ne se mesure plus.

\subsubsection{L'étalon de quantité de matière}

Une mole est la quantité de matière contenue dans un échantillon de \(\num{6.02214076e23}\) atomes. C'est une valeur exacte qui ne se mesure plus.

Avant 2019, on avait une quantité de matière étalon : un échantillon de \(\SI{12}{\gram}\) de Carbone 12.

\chapter{Incertitudes expérimentales}

\minitoc

La notion d'incertitude est essentielle dans la démarche expérimentale. Sans elle, on ne peut juger de la qualité d'une mesure, de sa pertinence ou de sa compatibilité avec une loi physique. On introduit ici les outils nécessaires à l'analyse des résultats expérimentaux.

\section{Mesure d'une grandeur physique}

Le processus d'attribution d'une valeur expérimentale à une grandeur physique s'appelle le mesurage.

L'instrument de mesure fournit une valeur mesurée.

La grandeur soumise au mesurage s'appelle le mesurande.

Par exemple, on mesure l'intensité du courant qui circule dans un circuit électrique à l'aide d'un ampèremètre. On effectue donc le mesurage du mesurande (l'intensité du courant électrique), la valeur mesurée étant affichée par l'ampèremètre (qui se place en série dans le montage).

\section{Erreur et incertitude}

Beaucoup de scientifiques confondent ces deux termes et parlent de calculs d'erreurs au lieu de calculs d'incertitudes.

\subsection{Erreurs -- définition de l'erreur}

Lors de la mesure d'une grandeur physique \(x\), l'erreur est la différence entre la valeur mesurée \(x\) et la valeur vraie \(X\). La valeur vraie est en général inconnue (puisqu'on la cherche).

\subsubsection{Erreurs aléatoires}

Lorsqu'on mesure la période d'oscillation d'un pendule en opérant avec un chronomètre manuel, on constate qu'en répétant les mesures on trouve des résultats légèrement différents, dus surtout aux retards de déclenchement qui vont réduire ou accroître la valeur de la période suivant qu'ils ont eu lieu au début ou à la fin de la mesure. Ce phénomène sera détecté par une étude statistique (en effectuant un grand nombre de mesures).

On parle d'erreur aléatoire.

Le résultat de la mesure est caractérisé par une distribution de probabilité répartie autour de la valeur vraie dans le cas d'erreurs purement aléatoires.

\subsubsection{Erreurs systématiques}

Supposons maintenant qu'on mesure la période d'oscillation d'un pendule avec un chronomètre faussé qui indique toujours des temps 2\% trop faibles. Une étude statistique ne le détectera pas, on parle d'erreur systématique. C'est la composante de l'erreur qui ne varie pas dans des conditions de mesures répétées.

Les erreurs systématiques sont difficiles à détecter a priori mais une fois détectées, on peut souvent les corriger.

Exemple : influence de la température sur la vitesse du son (si on ne précise pas la température, il est impossible de comparer la valeur mesurée à une valeur de référence).

On représente classiquement les rôles respectifs des erreurs aléatoires et systématiques par analogie avec un tir sur cible, le centre de la cible représentant la valeur vraie de la grandeur à mesurer :

\begin{center}
\begin{tikzpicture}
\draw (0,0) circle (1);
\draw (-1.2,-1.2) -- (1.2,-1.2) -- (1.2,1.2) -- (-1.2,1.2) -- (-1.2,-1.2);
\node[cross=2pt,red] at (0,0) {};
\node[cross=2pt] at (0.1,0.2) {};
\node[cross=2pt] at (0.2,0.1) {};
\node[cross=2pt] at (-0.15,0.1) {};
\node[cross=2pt] at (0.15,-0.1) {};
\node[cross=2pt] at (-0.15,-0.15) {};
\end{tikzpicture}
\hskip5pt
\begin{tikzpicture}
\draw (0,0) circle (1);
\draw (-1.2,-1.2) -- (1.2,-1.2) -- (1.2,1.2) -- (-1.2,1.2) -- (-1.2,-1.2);
\node[cross=2pt,red] at (0,0) {};
\node[cross=2pt] at (0.31,0.45) {};
\node[cross=2pt] at (0.1,0.49) {};
\node[cross=2pt] at (-0.45,0.35) {};
\node[cross=2pt] at (0.45,-0.41) {};
\node[cross=2pt] at (-0.45,-0.45) {};
\end{tikzpicture}
\hskip5pt
\begin{tikzpicture}
\draw (0,0) circle (1);
\draw (-1.2,-1.2) -- (2.4,-1.2) -- (2.4,1.2) -- (-1.2,1.2) -- (-1.2,-1.2);
\node[cross=2pt,red] at (0,0) {};
\node[cross=2pt] at (2.1,0.2) {};
\node[cross=2pt] at (2.2,0.1) {};
\node[cross=2pt] at (1.75,0.1) {};
\node[cross=2pt] at (2.15,-0.1) {};
\node[cross=2pt] at (1.75,-0.15) {};
\end{tikzpicture}
\begin{tikzpicture}
\draw (0,0) circle (1);
\draw (-1.2,-1.2) -- (2.6,-1.2) -- (2.6,1.2) -- (-1.2,1.2) -- (-1.2,-1.2);
\node[cross=2pt,red] at (0,0) {};
\node[cross=2pt] at (2.31,0.45) {};
\node[cross=2pt] at (2.1,0.49) {};
\node[cross=2pt] at (1.55,0.35) {};
\node[cross=2pt] at (2.45,-0.41) {};
\node[cross=2pt] at (1.55,-0.45) {};
\end{tikzpicture}
\end{center}

\begin{itemize}
\item Si tous les impacts sont proches du centre : faibles erreurs aléatoires et faible erreur systématique.
\item Si les impacts sont très étalés mais centrés en moyenne sur la cible : fortes erreurs aléatoires et faible erreur systématique.
\item Si les impacts sont groupés mais loin du centre : faibles erreurs aléatoires mais forte erreur systématique.
\item Si les impacts sont étalés et loin du centre : fortes erreurs aléatoires et forte erreur systématique. \\
\end{itemize}

Le défaut de cette analogie est qu'en général, dans les mesures physiques, on ne connaît pas le centre de la cible.

\subsection{Incertitudes}

L'incertitude \(\fdif{x}\) traduit les tentatives scientifiques pour estimer l'importance de l'erreur aléatoire commise. En l'absence d'erreur systématique, elle définit un intervalle de confiance autour de la valeur mesurée qui inclut la valeur vraie avec un niveau de confiance déterminé. La détermination de l'incertitude n'est pas simple a priori.

On rencontre en pratique deux situations :

\subsubsection{Évaluation de type A}

\(\fdif{x}\) est évaluée statistiquement. On cherche dans ce cas à caractériser la distribution de probabilité des valeurs de \(x\), en évaluant le mieux possible la valeur moyenne et l'écart-type de cette distribution. Ceci se fait par l'analyse statistique d'un ensemble de mesures de \(x\).

En l'absence d'erreurs systématiques, l'estimation de la valeur moyenne est la meilleure estimation de la valeur vraie \(X\) tandis que l'incertitude \(\fdif{x}\), directement reliée à l'estimation de l'écart-type de la distribution, définit un intervalle dans lequel la valeur vraie de \(x\) se trouve avec un niveau de confiance connu. On choisit le plus souvent comme incertitude l'estimation de l'écart-type de la distribution. On parle alors d'incertitude-type.

\subsubsection{Évaluation de type B}

\(\fdif{x}\) est évaluée par d'autres moyens. Si on ne dispose pas du temps nécessaire pour faire une série de mesures, on estime \(\fdif{x}\) à partir des spécifications des appareils de mesures et des conditions expérimentales.

\subsection{Présentation d'un résultat expérimental}

L'écriture rapportant la mesure d'une grandeur physique \(x\) est : \[\text{valeur mesurée de }x=\overline{x}\pm\fdif{x}\] où \begin{description}
\item \(\overline{x}\) est la meilleure estimation de la valeur vraie \(X\)
\item \(\fdif{x}\) est l'incertitude-type sur la mesure (incertitude absolue). \\
\end{description}

En l'absence d'erreur systématique, on considère que la valeur vraie \(X\) de \(x\) se trouve dans l'intervalle \begin{itemize}
\item \(\intervee{\overline{x}-\fdif{x}}{\overline{x}+\fdif{x}}\) avec une probabilité de 68\% ;
\item \(\intervee{\overline{x}-2\fdif{x}}{\overline{x}+2\fdif{x}}\) avec une probabilité de 95\% ;
\item \(\intervee{\overline{x}-3\fdif{x}}{\overline{x}+3\fdif{x}}\) avec une probabilité de \(\num{99.7}\)\%.
\end{itemize}

\subsection{Comparaison entre valeur mesurée et valeur acceptée}

Ayant obtenu la valeur mesurée avec son intervalle d'incertitude, on la compare à la valeur de référence (pour une valeur expérimentale de référence, on ne parle pas de valeur exacte mais de valeur tabulée).

Il n'est pas anormal que l'intervalle ne contienne pas la valeur de référence.

On commencera à douter de la qualité de la mesure lorsque l'écart entre la valeur tabulée et la valeur mesurée atteint plus de \(2\fdif{x}\).

\subsection{Comparaison de deux mesures}

Pour pouvoir comparer deux mesures entre elles, il faut un critère quantitatif pour indiquer si ces deux mesures sont considérées comme compatibles ou incompatibles.

On définit donc l'écart normalisé \(E_N\) entre deux processus de mesure donnant les valeurs \(m_1\) et \(m_2\) et d'incertitudes-types \(u\paren{m_1}\) et \(u\paren{m_2}\) par : \[E_N=\dfrac{\abs{m_1-m_2}}{\sqrt{u\paren{m_1}^2+u\paren{m_2}^2}}.\]

Par convention, on qualifie souvent deux résultats de compatibles si leur écart normalisé vérifie la propriété \(E_N<2\).

\section{Évaluation de type A de l'incertitude}

On s'occupe ici de la mesure d'une grandeur physique \(x\) dont les sources de variabilité sont uniquement aléatoires.

Dans la pratique, on réalise un nombre fini \(n\) de mesures de résultats respectifs \(x_1,x_2,\ldots,x_n\) dont on cherche à extraire les meilleures estimations de \(X\), valeur moyenne et \(\sigma\), écart-type, de la distribution de probabilité de \(x\).

\subsection{Meilleure estimation de la moyenne de la distribution des valeurs de \(x\)}

La meilleure estimation de la valeur vraie \(X\), notée \(\overline{x}\), obtenue à partir des \(n\) mesures \(x_1,x_2,\ldots,x_n\) est la moyenne de ces mesures : \[\text{meilleure estimation de }X=\overline{x}=\dfrac{x_1+x_2+\ldots+x_n}{n}.\]

\subsection{Meilleure estimation de l'écart-type de la distribution des valeurs de \(x\)}

La meilleure estimation de \(\sigma\) déduite des \(n\) mesures \(x_1,x_2,\ldots,x_n\) notée \(\sigma_x\) est donnée par : \[\text{meilleure estimation de }\sigma=\sigma_x=\sqrt{\dfrac{1}{n-1}\sum_{i=1}^n\paren{x_i-\overline{x}}^2}.\]

\subsection{Incertitude-type \(\fdif{x}\)}

L'incertitude-type \(\fdif{x}\) sur la mesure de \(x\) se déduit de l'écart-type de la distribution des valeurs de \(x\), \(\sigma_x\), par : \[\fdif{x}=\dfrac{\sigma_x}{\sqrt{n}}\] où \(n\) est le nombre de mesures des valeurs de \(x\).

\subsection{Bilan}

Si on réalise \(n\) mesures de \(x\), avec les résultats \(x_1,x_2,\ldots,x_n\), on écrira le résultat final sous la forme : \[x=\overline{x}\pm\dfrac{\sigma_x}{\sqrt{n}}\] où \(\overline{x}\) et \(\dfrac{\sigma_x}{\sqrt{n}}\) sont les meilleures estimations de la valeur vraie et de l'incertitude-type.

Exemple pratique :

Huit étudiants mesurent la longueur d'onde de la raie verte du mercure et obtiennent les résultats suivants :

\begin{center}
\begin{tabular}{|c|c|c|c|c|c|c|c|c|}
\hline
\(i\) (n° de l'étudiant) & 1 & 2 & 3 & 4 & 5 & 6 & 7 & 8 \\
\hline
\(\lambda\) trouvée (en \(\unit{\nano\metre}\)) & \(\num{538.2}\) & \(\num{554.3}\) & \(\num{545.7}\) & \(\num{552.3}\) & \(\num{566.4}\) & \(\num{537.9}\) & \(\num{549.2}\) & \(\num{540.3}\) \\
\hline
\end{tabular}
\end{center}

En utilisant la calculatrice, on peut déterminer aisément \(\overline{\lambda}=\SI{548.04}{\nano\metre}\) et \(\sigma_\lambda=\SI{9.72}{\nano\metre}\).

On en déduit que l'incertitude sur la moyenne des huit valeurs vaut \(\fdif{\lambda}=\dfrac{\sigma_x}{\sqrt{8}}=\dfrac{\num{9.72}}{\sqrt{8}}=\SI{3.44}{\nano\metre}\).

On peut donc écrire : meilleure estimation de \(\lambda=548\pm\SI{3}{\nano\metre}\).

On peut comparer à la valeur tabulée \(\lambda_\text{tab}=\SI{545.07}{\nano\metre}\) et conclure qu'il y a une bonne concordance.

\section{Évaluation de type B de l'incertitude}

On rappelle que l'évaluation de type B de l'incertitude est réalisée lorsqu'il est trop long ou impossible de procéder à une évaluation de type A. Une connaissance générale de l'expérience est nécessaire pour rechercher et évaluer les sources d'erreurs.

L'évaluation de type B de l'incertitude d'une mesure effectuée sur un instrument de précision \(\Delta\) est : \[\fdif{x}=\dfrac{\Delta}{\sqrt{3}}.\]

On retiendra que : \begin{itemize}
\item la précision des instruments de mesure gradués est égale à une demi-graduation ;
\item la précision des autres instruments notamment numériques est à chercher sur la notice de l'instrument ;
\item la précision de la méthode de mesure est à déterminer expérimentalement. \\
\end{itemize}

Exemples : \begin{itemize}
\item sur une règle graduée au \(\unit{\milli\metre}\), la précision vaut \(\Delta=\SI{0.5}{\milli\metre}\) ;
\item sur la notice d'un multimètre numérique utilisé en DC sur le calibre \(\SI{5}{\volt}\), on lit \guillemets{\foreignlanguage{british}{Accuracy: \(\num{0.3}\)\% rdg + 2 digits}}. La précision de l'appareil est donc de \(\num{0.3}\)\% de la valeur lue (rdg signifie \foreignlanguage{british}{reading}) à laquelle on ajoute deux fois la valeur du dernier chiffre affiché.

Pour une valeur lue de \(\SI{2.5462}{\volt}\), la précision vaut : \(\Delta=\dfrac{\num{0.3}}{100}\times\num{2.5462}+\num{0.0002}=\SI{0.0076}{V}\).
\end{itemize}

\section{Propagation des incertitudes}

Lorsqu'on réalise une mesure indirecte, on calcule la valeur d'une grandeur physique à partir de grandeurs mesurées. Les incertitudes de détermination des grandeurs mesurées se propagent sur la grandeur calculée et on doit déterminer l'incertitude induite sur cette dernière. Cette compétence est particulièrement utile lorsqu'on effectue une évaluation de type B de l'incertitude.

On s'intéresse donc au problème suivant : on connaît les grandeurs expérimentales \(x,y,\ldots\) avec les incertitudes \(\fdif{x},\fdif{y},\ldots\). Quelle est l'incertitude \(\fdif{q}\) sur la grandeur physique \(q=f\paren{x,y,\ldots}\) ? On peut montrer qu'on a : \[\fdif{q}=\sqrt{\paren{\pdv{f}{x}}^2\times\paren{\fdif{x}}^2+\paren{\pdv{f}{y}}^2\times\paren{\fdif{y}}^2+\ldots}\] où \(\pdv{f}{x}\) est la dérivée partielle de \(f\) par rapport à \(x\), les autres variables étant considérées comme constantes.

\chapter{Propagation d'un signal}

\minitoc

\note{À VENIR}

\chapter{Optique géométrique}

\minitoc

\note{À VENIR}

\chapter{Introduction à la mécanique quantique}

\minitoc

\section*{Introduction}
\addcontentsline{toc}{section}{Introduction}

La notion d'atome est déjà bien établie et, grâce à diverses expériences, on connaît les constituants principaux de la matière :

\begin{itemize}
    \item Crookes et Perrin : expériences du rayonnement cathodique \(\imp\) mise en évidence des électrons ; \\
    \item Thomson : déviation de faisceaux d'électrons dans des champs électriques et magnétiques \(\imp\) détermination du rapport \(\dfrac{q}{m_e}\) ; \\
    \item Millikan (1909) : chute de goutelettes d'huile dans un champ électrique \(\imp\) détermination de \(q\) : \[q=-e=\SI{-1.602e-19}{\C}\qquad m_e=\SI{9.1e-31}{\kilo\gram}\]~
    \item Chadwick et Goldstein : détermination des caractéristiques des protons et des neutrons : \[q_p=e\qquad q_n=0\qquad m_p\approx m_n\approx\num{1840}m_e=\SI{1.67e-27}{\kilo\gram}.\]
\end{itemize}

Au cours du temps, divers modèles d'atomes ont été proposés :

\begin{itemize}
    \item 1901 : Perrin propose un modèle planétaire avec un \guillemets{soleil} de charge positive autour duquel gravitent des corpuscules minuscules de charge négative ; \\
    \item 1903 : Thomson propose un modèle globulaire : l'atome serait une sphère d'électricité positive à l'intérieur de laquelle gravitent les électrons (\guillemets{pudding} de Thomson) ; \\
    \item 1911 : l'expérience de Rutherford (étude de la déviation de particules \(\alpha\) chargées positivement et traversant une mince feuille d'or) met en évidence le caractère lacunaire de l'atome, ce qui permet de rejeter le modèle de Thomson. Rutherford propose un modèle planétaire avec des orbites circulaires pour les électrons autour du noyau.
\end{itemize}

Ordres de grandeur des rayons : \[\text{noyau : }r\approx\SI{e-15}{\meter}\qquad\text{atome : }R\approx\SI{e-10}{\meter}.\]

Problèmes rencontrés :

\begin{itemize}
    \item le modèle de Rutherford est a priori assez intéressant, mais un électron en accélération centrale doit, d'après la physique classique, émettre un spectre continu d'énergie, donc perdre de l'énergie et finir sur le noyau ! \\
    \item tous ces modèles sont incapables de décrire de façon satisfaisante l'infiniment petit et notamment la quantification de l'énergie des atomes.
\end{itemize}

\part{Thermodynamique}

\chapter{Bases de la thermodynamique \& modèle du gaz parfait}

\minitoc

\section*{Introduction}
\addcontentsline{toc}{section}{Introduction}

La thermodynamique est la science des phénomènes thermiques. Elle est née à la fin du \siecle{xviii} siècle, avec l'apparition de la machine à vapeur. De grands physiciens tels que William Thomson (Kelvin), Joule ou Watt ont participé à son développement.

\section{Description d'un système thermodynamique}

\subsection{Définition}

Un système thermodynamique est un corps ou un ensemble de corps séparés du milieu extérieur par une frontière (réelle ou fictive).

\begin{tkz}
\node[align=center] at (0,0) {Système\\thermodynamique};
\draw (0,0) circle (2); % cercle système
\draw (-4,-4) -- (-4,4) -- (4,4) -- (4,-4) -- (-4,-4); % carré milieu extérieur
\node[align=center] at (2.5,3) {Milieu\\extérieur};
\node[below] at (0,-2) {Frontière};
\node[below right] at (4,4) {Univers = Milieu extérieur \(\union\) Système thermodynamique};
\end{tkz}

À travers cette frontière, il peut exister des échanges de matière et d'énergie.

\begin{itemize}
\item Système fermé : pas d'échange de matière avec le milieu extérieur.

\item Système isolé : pas d'échange de matière ou d'énergie avec le milieu extérieur.

\item Système ouvert : échanges de matière ou d'énergie avec le milieu extérieur possibles.
\end{itemize}

\subsection{Échelles d'étude}

En thermodynamique, les systèmes étudiés seront toujours caractérisés par un très grand nombre de particules. Par exemple, \(\SI{1}{\milli\meter\cubed}\) d'air ambiant contient de l'ordre de \(\num{e16}\) molécules. En effet :

Aux conditions normales de température et de pression, on a le volume molaire : \[V_m=\SI{24}{\liter\per\mole}.\]

Or on a le volume \(V=\SI{1}{mm\cubed}=\SI{e-9}{\meter\cubed}\) donc on a la quantité de matière \[n=\dfrac{V}{V_m}=\dfrac{\num{e-9}}{\num{24e-3}}=\SI{4e-8}{\mole}.\]

Or une mole contient \(N_A=\SI{6.022e23}{\per\mole}\) particules donc on retrouve bien \[N=\numproduct{6.022e23x4e-8}=\num{2.4e16}\] molécules dans \(\SI{1}{mm\cubed}\) d'air.

Problème : \begin{itemize}
\item pour un système à deux corps en interaction : résolution analytique ;

\item pour un système à trois corps en interaction : résolution numérique ;

\item pour un système à \(\num{e16}\) corps en interaction : résolution numérique impossible, même pour les plus gros ordinateurs.\\
\end{itemize}

On définit donc trois échelles d'étude :

\begin{itemize}
\item \underline{Échelle microscopique} : chaque particule ponctuelle possède trois variables de position et trois variables de vitesse, donc pour \(N\) particules, on a \(6N\) variables. Étant donné que l'on ne peut pas suivre la trajectoire de chaque particule au cours du temps, l'étude ne peut être que statistique et relève donc de la thermodynamique statistique.\\

\item \underline{Échelle macroscopique} : le système présente un comportement collectif. En effet, la moyenne des effets microscopiques donne à toute grandeur un aspect continu. On a donc une description globale à partir de paramètres macroscopiques que l'on appelle paramètres d'état. C'est cette échelle que nous utiliserons généralement.\\

\item \underline{Échelle mésoscopique} : c'est une échelle intermédiaire entre celle de la mole (macroscopique) et celle de la molécule (microscopique) et on l'utilisera quelques fois, notamment en statique des fluides. On travaille sur des éléments de volume petits à l'échelle macroscopique mais quand même assez grands pour pouvoir contenir un grand nombre de particules (par exemple : \(\SI{1}{mm\cubed}\) d'air). Ce nombre est assez grand pour pouvoir considérer que la matière contenue dans l'élément de volume est homogène et identifiable par des paramètres thermodynamiques (pression, température, ...) ; le volume est assez petit pour pouvoir considérer que ces paramètres y ont la même valeur en tout point (moyennes statistiques).
\end{itemize}

Ordres de grandeur :

\begin{tabular}{c|c|c|c}
& Microscopique & Mésoscopique & Macroscopique \\[1em]
\(N\) (nombre de particules) & \(\num{1}\) & \(\numrange{e13}{e16}\) & \(\num{e23}\) \\[1em]
Taille (en \(\unit{m}\)) & \(\num{e-10}\) & \(\num{e-5}\) (\(\SI{10}{\micro\metre}\)) & \(\num{1}\)
\end{tabular}

\subsection{Équilibre thermodynamique}

L'expérience nous montre que tout système isolé tend vers un état d'équilibre pour lequel les grandeurs température, pression, densité moléculaire, etc..., sont les mêmes en tout point.

Ceci revient à dire qu'il n'y a pas de mouvement macroscopique à l'intérieur du système. Évidemment, les particules continuent à avoir un mouvement microscopique : c'est l'agitation thermique.

En fait, les grandeurs macroscopiques fluctuent autour d'une valeur moyenne qui reste constante au cours du temps.

\subsection{Paramètres/variables d'état}

Ce sont les grandeurs qui permettent de définir l'état d'un système à un instant donné.

Par exemple : pression \(P\), température \(T\), charge \(q\), etc...

Toutes ces grandeurs sont susceptibles d'être modifiées lors d'une transformation du système.

\subsubsection{Température}

Pour l'instant, on la considère sous son sens le plus courant, c'est-à-dire la grandeur macroscopique mesurable à l'aide d'un thermomètre.

L'expérience montre que : \begin{itemize}
\item un corps en équilibre thermodynamique possède la même température en chacun de ses points ;

\item deux corps mis en contact prolongé se mettent en équilibre thermique ;

\item deux corps en équilibre thermique avec un troisième corps sont en équilibre thermique entre eux : \[\paren{T_1=T_3\quad\text{et}\quad T_2=T_3}\imp T_1=T_2.\] C'est le principe \guillemets{zéro} de la thermodynamique ou principe de l'équilibre thermique.\\
\end{itemize}

Son unité légale est le Kelvin (\(\unit{\kelvin}\)). Elle est liée au degré Celsius (\(\unit{\degreeCelsius}\)) par : \[T\paren{\unit{\kelvin}}=T\paren{\unit{\degreeCelsius}}+\num{273.15}\]

\subsubsection{Pression}

\begin{tkz}[scale=3]
\draw (0.707,0.707) arc (45:-45:1); % arc de cercle du fluide
\draw[ultra thick] (1,0.2) -- (1,-0.2) node[below right] {\(\odif{S}\)}; % surface d'application de la force
\draw[->] (1,0) node[left] {\(M\)} -- (2,0) node[above left] {\(\vec{n}\)}; % vecteur normal unitaire
\draw[->] (1,0) -- (4,0) node[above left] {\(\odif{\vec{S}}\)};
\node[below left] at (0.707,0.707) {fluide};
\node at (3,0.707) {milieu extérieur};
\end{tkz}

Pour un fluide en équilibre au contact d'une paroi solide, on définit la pression au point \(M\) par : \[\odif{\vec{F}_{\paren{M}}}=p_{\paren{M}}\odif{\vec{S}_{\paren{M}}}=p_{\paren{M}}\odif{S_{\paren{M}}}\vec{n}.\]

La pression est donc la force exercée par le fluide par unité de surface. Elle existe en tout point du fluide, même si celui-ci n'est pas en contact avec une paroi (c'est celle que mesurerait un manomètre).

Son unité légale est le pascal (\(\unit{\pascal}\)) : \(\SI{1}{\pascal}=\SI{1}{\newton\per\square\meter}\).

Il existe d'autres unités usuelles : \begin{itemize}
\item le bar : \(\SI{1}{\bar}=\SI{e5}{\pascal}\) ;

\item l'atmosphère : \(\SI{1}{atm}=\SI{1.0135e5}{\pascal}\) ;

\item le millimètre de Mercure : \(\SI{760}{mm\,Hg}=\SI{1.0135e5}{\pascal}=\SI{1}{torr}\).
\end{itemize}

\subsubsection{Grandeurs extensives et grandeurs intensives}

On dit qu'une grandeur (paramètre ou fonction d'état) est intensive pour un système \(\paren{\Sigma}\) si elle prend la même valeur dans tout sous-système de \(\paren{\Sigma}\) à l'équilibre thermodynamique, indépendamment de sa taille. En clair, c'est une grandeur indépendante de la quantité de matière.

Par exemple, la pression, la température ou la masse volumique sont des grandeurs intensives.

On dit qu'une grandeur \(G\) (paramètre ou fonction d'état) est extensive si elle est additive, c'est-à-dire si elle est proportionnelle à la quantité de matière. En clair, si \(\paren{\Sigma}=\paren{\Sigma_1}\union\paren{\Sigma_2}\) alors \(G=G_1+G_2\).

Par exemple, le volume, la masse ou la quantité de matière sont des grandeurs extensives.

Pour un système constitué d'une phase homogène, on peut construire de nouvelles grandeurs intensives en faisant le rapport de deux grandeurs extensives pour chacun de ses éléments de volume \(\odif{V}\) : \[\text{int}=\dfrac{\text{ext}}{\text{ext}}\]

Par exemple, la masse volumique \(\mu=\odv{m}{V}\), le volume massique \(v=\dfrac{1}{\mu}=\odv{V}{m}\), etc...

\subsection{Équation d'état}

C'est une équation qui relie entre elles les différentes variables d'état qui caractérisent l'état d'un système.

Par exemple, dans le cas d'un gaz faiblement comprimé, l'équation d'état déterminée expérimentalement est : \[pV=nRT\] où \begin{description}
\item \(n\) : nombre de moles

\item \(R\) : constante des gaz parfaits (\(R=\SI{8.314}{\joule\per\kelvin\per\mole}\))\\
\end{description}

Il faut évidemment que toutes ces grandeurs soient exprimées dans le Système International : \[p:\unit{\pascal}\qquad V:\unit{m^3}\qquad T:\unit{\kelvin}\]

Pour les gaz fortement comprimés et les liquides, les équations d'état sont généralement plus compliquées et ne sont valables qu'au voisinage d'un état d'équilibre thermodynamique donné.

\section{Approche microscopique du gaz parfait monoatomique}

\subsection{Agitation moléculaire}

Bernoulli (\siecle{xviii} siècle) postule qu'un gaz est constitué d'un grand nombre de particules en agitation incessante.

Brown (1827) observe au microscope le mouvement désordonné de particules colloïdales dans un fluide : mouvement brownien.

La trajectoire de chaque particule est une marche au hasard en direction et en vitesse. On a un chaos moléculaire.

La marche au hasard est due aux collisions particule/particule et particule/paroi.

\subsection{Le modèle des gaz parfaits monoatomiques}

Un gaz parfait monoatomique est constitué uniquement d'atomes (gaz rares, par exemple : He, Ne, Ar, ...). Dans la nature, on rencontre essentiellement des gaz parfaits diatomiques (par exemple : O\(_2\), N\(_2\), H\(_2\), ...) ou triatomiques (par exemple : CO\(_2\), H\(_2\)O, ...).

Par définition, un gaz est parfait si les atomes n'interagissent pas entre eux. Les atomes sont ponctuels et tous identiques. Les seules interactions sont les collisions atome/paroi.

\subsection{La pression cinétique}

Hypothèses du modèle d'un gaz parfait monoatomique : \begin{itemize}
\item les atomes sont tous identiques et il n'existe pas d'interaction entre eux ;

\item les atomes sont animés de la même vitesse \(u\) ;

\item à tout instant, chaque atome ne se déplace que selon une direction (\(\vec{u}_x\), \(\vec{u}_y\) ou \(\vec{u}_z\)) et dans un sens (\(\rightleftarrows\)) ;

\item tous les sens et directions sont équiprobables.\\
\end{itemize}

Donc à un instant donné, un sixième des atomes se déplace dans un sens et une direction fixés.

\subsubsection{Choc d'un atome contre la paroi}

\begin{tkz}[scale=1.8]
\draw[<-] (-3,0) node[below left] {\(x\)} -- (6,0); % axe
\draw[<-,blue,thick] (-1,0) node[below left] {\(\vec{u}_x\)} -- (0,0); % vecteur unitaire ux
\draw[ultra thick] (0,1) -- (0,4); % paroi
\fill[pattern=north east lines] (0,1) -- (-0.5,1) -- (-0.5,4) -- (0,4); % bloc paroi
\draw[<-,green] (2,3) node[above left] {\(\vec{v}\)} -- (4,3); % vecteur vitesse v
\draw[->,green] (2,2) -- (4,2) node[below right] {\(\vec{v}^{\,\prime}\)}; % vecteur vitesse v'
\end{tkz}

Avant le choc, l'atome a pour vitesse \(\vec{v}=u\vec{u}_x\).

Après le choc, il a pour vitesse \(\vec{v}^{\,\prime}=-u\vec{u}_x\).

On calcule la variation de quantité de mouvement de l'atome lors de la collision : \[\odif{\vec{p}_{\text{atome}}}=\vec{p}_f-\vec{p}_i=m\vec{v}^{\,\prime}-m\vec{v}=-2mu\vec{u}_x.\]

Donc on a \[\odif{\vec{p}_{\text{paroi}}}=-\odif{\vec{p}_{\text{atome}}}=2mu\vec{u}_x.\]

\subsubsection{Choc de l'ensemble des atomes utiles pendant une durée \(\odif{t}\)}

\begin{tkz}
\draw[ultra thick] (0,0) -- (0,5); % paroi
\fill[pattern=north east lines] (0,0) -- (-0.5,0) -- (-0.5,5) -- (0,5); % bloc paroi
\draw[dashed] (0,1) -- (3,1) -- (3,4) -- (0,4); % aire contenant les atomes utiles
% vecteurs vitesse de quelques atomes
\draw[->,green] (0.5,2) -- (0.5,1);
\draw[->,green] (2,2) -- (1,2);
\draw[->,green] (2,2.5) -- (2,3.5);
\draw[->,green] (2.5,3) -- (3.5,3);
\draw[->,green] (0.5,4.5) -- (0.5,3.5);
\draw[->,green] (1.5,3) -- (0.5,3);
\draw[->,green] (1.5,1.5) -- (2.5,1.5);
\draw[<->] (0,0.5) -- (3,0.5); % longueur u x dt
\node[below] at (1.5,0.5) {\(u\odif{t}\)};
\end{tkz}

Seuls les atomes avec une vitesse selon \(+\vec{u}_x\) peuvent entrer en collision avec la paroi s'ils sont situés à une distance inférieure à \(u\odif{t}\) de la paroi.

On pose \(n^*\) le nombre d'atomes par unité de volume. C'est la densité particulaire (en \(\unit{\per\metre\cubed}\)).

On a le nombre de collisions : \[\odif{N}=\dfrac{n^*\overbrace{u\odif{t}\odif{S}}^{\substack{\text{volume où} \\ \text{sont logés} \\ \text{les atomes} \\ \text{utiles}}}}{6}\] où \(\odif{S}\) est la surface de la paroi.

Donc on a la variation de quantité de mouvement de la paroi pendant \(\odif{t}\) : \[\odif{\vec{p}_{\text{paroi}}}=\dfrac{2mu^2n^*\odif{t}\odif{S}\vec{u}_x}{6}\]

On applique le principe fondamental de la dynamique sur la paroi : \[\odv{\vec{p}_{\text{paroi}}}{t}=\odif{\vec{F}_{\text{paroi}}}=\dfrac{2mu^2n^*\odif{S}\vec{u}_x}{6}\] où \(\odif{\vec{F}_{\text{paroi}}}\) est la force que subit la paroi sur la surface \(\odif{S}\).

Or \(\odif{\vec{F}_{\text{paroi}}}=p\odif{\vec{S}}\) donc \[p=\dfrac{1}{3}n^*mu^2\] est la pression cinétique (due aux collisions).

\subsubsection{Critique du modèle}

\begin{itemize}
\item En réalité, il n'y a pas forcément que six couples \(\paren{\text{direction},\text{sens}}\).

\item Les vitesses de chaque atome ne sont pas forcément égales.

\item La paroi n'est pas nécessairement plane.

\item Les collisions peuvent ne pas être élastiques.
\end{itemize}

\subsection{Température cinétique du gaz parfait monoatomique}

On définit la température cinétique du gaz parfait monoatomique à l'équilibre thermodynamique comme une moyenne de l'énergie cinétique d'agitation thermique : \[\moy{E_{c_{\text{atome}}}}=\moy{\dfrac{1}{2}m\vec{v}^2}=\dfrac{1}{2}mu^2=\dfrac{3}{2}k_BT\] où \begin{description}
\item \(T\) est la température (en Kelvin, \(\unit{\kelvin}\)) ;

\item \(k_B\) est la constante de Boltzmann (\(k_B=\SI{1.38e-23}{\joule\per\kelvin}\)).\\
\end{description}

On en déduit : \[u=\sqrt{\dfrac{3k_BT}{m}}.\]

À \(T=\SI{0}{\kelvin}\), on a \(u=\SI{0}{\metre\per\second}\) (état d'ordre absolu).

À \(T=\SI{293}{\kelvin}\), en considérant la masse de l'Hélium \(m_{\text{He}}=4\times\SI{1.67e-27}{\kilo\gram}\), on a \(u\propto\SI{e3}{\metre\per\second}<c\), ce qui est cohérent. On a un grand nombre de collisions donc cela valide le modèle du gaz parfait monoatomique.

\subsection{Équation d'état}

On a : \[p=\dfrac{1}{3}n^*mu^2\quad\text{et}\quad\dfrac{3}{2}k_BT=\dfrac{1}{2}mu^2\] donc : \[p=n^*k_BT\] où \(n^*\) est le nombre de particules par unité de volume donc \(n^*=\dfrac{N}{V}\), où \(N\) est le nombre de particules donc \(N=nN_A\) donc on a : \[p=\dfrac{nN_A}{V}k_BT.\]

On pose : \[R=N_Ak_B=\SI{8.314}{\joule\per\kelvin\per\mole}.\]

On a \(p=\dfrac{nRT}{V}\), d'où : \[pV=nRT.\]

C'est l'équation d'état des gaz parfaits, obtenue à partir du modèle cinétique.

\(p\), \(V\), \(n\) et \(T\) sont des paramètres d'état donc des grandeurs macroscopiques.

\section{Énergie interne}

\subsection{Définition}

L'énergie interne est l'énergie totale contenue dans un système thermodynamique. Elle est notée \(U\). Elle est la somme de \begin{itemize}
\item l'énergie cinétique de translation des particules ;

\item l'énergie cinétique de rotation des particules sur elles-mêmes ;

\item l'énergie cinétique de vibration des particules polyatomiques ;

\item l'énergie potentielle d'interaction entre les particules.
\end{itemize}

\subsection{Gaz parfait monoatomique}

Pour un gaz parfait monoatomique, l'énergie interne n'est constituée que de l'énergie cinétique de translation des particules : \[U=\sum_iE_{c_i}=\sum_i\dfrac{1}{2}mu^2=N\times\dfrac{1}{2}mu^2.\]

Or on a \(\dfrac{3}{2}k_BT=\dfrac{1}{2}mu^2\) donc : \[U=\dfrac{3}{2}Nk_BT.\]

Or \(N=nN_A\) donc : \[U=\dfrac{3}{2}nN_Ak_BT=\dfrac{3}{2}nRT.\]

On a une équation d'état.

On sait que \(n\) est une grandeur extensive donc \(U\) est une grandeur extensive.

\(U\) ne dépend que des paramètres d'état donc c'est une fonction d'état. On peut donc décrire un système en donnant son énergie interne.

De plus, on définit : \[C_V=\pdv{U}{T}_V\] la capacité thermique à volume constant (en \(\unit{\joule\per\kelvin}\)).

Or \(U=\dfrac{3}{2}nRT\) donc : \[\pdv{U}{T}_V=\odv{U}{T}=\dfrac{3}{2}nR.\]

\(C_V\) est l'énergie qu'il faut fournir au gaz parfait monoatomique pour élever sa température de \(\SI{1}{\kelvin}\).

On définit aussi la capacité thermique molaire à volume constant (en \(\unit{\joule\per\kelvin\per\mole}\)) : \[C_{V_m}=\dfrac{C_V}{n}\] et la capacité thermique massique à volume constant (en \(\unit{\joule\per\kelvin\per\kilo\gram}\)) : \[c_V=\dfrac{C_V}{m}.\]

Pour un gaz parfait monoatomique, on a : \[C_{V_m}=\dfrac{\frac{3}{2}nR}{n}=\dfrac{3}{2}R=\SI{12.5}{\joule\per\kelvin\per\mole}.\]

\(C_V\) est extensive mais \(C_{V_m}\) et \(c_V\) sont intensives.

\subsection{Cas du gaz parfait polyatomique}

Les gaz parfait polyatomiques contiennent des molécules non-ponctuelles donc dans l'expression de l'énergie interne, il faut ajouter l'énergie cinétique de rotation et de vibration, ainsi que l'énergie potentielle élastique.

Pour les pressions faibles, le gaz parfait polyatomique tend vers un comportement limite tel que : \begin{itemize}
\item on a une équation d'état \(pV=nRT\) ;

\item son énergie interne ne dépend que de la température : \(U=U\paren{T}\). \\
\end{itemize}

Idem, on définit : \[C_V=\pdv{U}{T}_V=\odv{U}{T}.\]

On trouve \(C_{V_m}>\dfrac{3}{2}R\). Cela signifie qu'il faut plus d'énergie pour élever la température d'un gaz parfait polyatomique que celle d'un gaz parfait monoatomique.

\subsection{Phases condensées}

Les phases condensées sont les liquides ou les solides. Les molécules sont très proches les unes des autres et ont donc beaucoup d'interactions entre elles.

On définit la compressibilité isotherme (en \(\unit{\per\pascal}\)) : \[\chi=-\dfrac{1}{V}\pdv{V}{P}_T\]

En général, pour une phase condensée, \(\chi=0\) donc \(\odif{V}=0\).

Donc quelles que soient les conditions de température et de pression, \(V=\cte\) : on a une équation d'état pour les phases condensées.

De plus, de même que pour les gaz parfaits, l'énergie interne des phases condensées ne dépend que de la température : \(U=U\paren{T}\).

On définit \[C_V=\pdv{U}{T}\] et on obtient \[\odif{U}=C_V\odif{T}.\]

Pour \(C_V\), on utilise des tables thermodynamiques.

On n'a pas de modèle pour les phases condensées parce qu'il y a beaucoup d'interactions entre les molécules.

\chapter{Premier principe de la thermodynamique}\label{chap:premierPrincipe}

\minitoc

\section*{Introduction}

La thermodynamique est la discipline qui fait le lien entre les phénomènes mécaniques et thermiques.

Son but est de faire des bilans d'énergie sur des systèmes thermodynamiques au cours de transformations thermodynamiques.

\section{Transformations d'un système thermodynamique}

\subsection{Types de transformations}

Rappel : un système thermodynamique \(\paren{\Sigma}\) est à l'équilibre thermodynamique si ses paramètres d'état intensifs sont définis en tout point du système et ont la même valeur en tout instant.

Pour que \(\paren{\Sigma}\) évolue vers un autre état thermodynamique (\ie un autre état d'équilibre), il faut rompre le premier état d'équilibre en ajoutant ou en supprimant des contraintes.

Le passage d'un état d'équilibre à un autre est une transformation thermodynamique.

\subsubsection{Transformations infiniment lentes}

Au cours d'une transformation thermodynamique infiniment lente, le système passe par une succession d'états d'équilibre infiniment voisins les uns des autres.

Pour cela, il faut que le temps de réponse du système soit très faible, de sorte qu'après modification des contraintes, le système rejoigne un état d'équilibre.

Pour une transformation infiniment lente, \(\paren{\Sigma}\) est en état d'équilibre à tout instant.

\subsubsection{Transformations réversibles}

Il s'agit d'une transformation thermodynamique infiniment lente au cours de laquelle on peut inverser le sens de parcours.

Dans les deux sens, le système repasse par les mêmes états d'équilibre successifs.

Pour une transformation réversible, à tout instant, \(\paren{\Sigma}\) est en état d'équilibre et en équilibre avec le milieu extérieur.

\subsubsection{Transformations irréversibles}

Une transformation réversible est en réalité un modèle inatteignable.

Toute transformation est en réalité irréversible, soit parce qu'elle est rapide, soit parce qu'elle est lente mais non-renversable au cours du temps.

Les principales causes d'irréversibilité sont : \begin{itemize}
\item les transferts de matière et d'énergie dus aux hétérogénéités de température, de pression, de concentration, ... ;

\item les réactions chimiques en général ;

\item les frottements solides ou visqueux.\\
\end{itemize}

Par exemple, étudions la compression d'un gaz dans un cylindre.

\underline{Cas numéro 1 :}

\begin{tkz}[scale=1.5]
\draw (0,4.5) -- (0,0) -- (3,0) -- (3,4.5); % cylindre
\draw (0,3) -- (3,3); % haut piston
\draw (0,2.8) -- (3,2.8); % bas piston
\draw (1,3) -- (1,4) -- (2,4) -- (2,3); % masse
\foreach \i in {1,...,1000} \fill (rnd*3,rnd*2.8) circle (0.01); % atomes
\draw[->,red] (1.5,3.5) node[above,black] {\(m\)} -- (1.5,2.3) node[left] {\(\vec{P}\)}; % vecteur poids
\draw[<-] (2.7,2.9) -- (3.4,2.9) node[right] {piston mobile};
\draw[<-] (2.7, 0.3) -- (3.4,0.3) node[right] {gaz};
\end{tkz}

On dépose brutalement une masse \(m\) sur le piston mobile, qui descend donc rapidement, oscille, et finit par s'arrêter en raison des frottements.

La transformation n'est pas infiniment lente : juste après avoir déposé la masse, la pression et la température du gaz ne sont pas définies. La transformation est donc irréversible.

\underline{Cas numéro 2 :}

\begin{tkz}[scale=1.5]
\draw (0,4.5) -- (0,0) -- (3,0) -- (3,4.5); % cylindre
\draw (0,3) -- (3,3); % haut piston
\draw (0,2.8) -- (3,2.8); % bas piston
\foreach \i in {1,...,1000} \fill (rnd*3,rnd*2.8) circle (0.01); % atomes
\fill[pattern={Dots[distance=2pt]}] (1,3) -- (1.5,4) -- (2,3); % grains de sable
\draw[dotted] (1.5,4.6) -- (1.5,4); % grains de sable tombant
\draw[<-] (2.7,2.9) -- (3.4,2.9) node[right] {piston mobile};
\draw[<-] (2.7, 0.3) -- (3.4,0.3) node[right] {gaz};
\draw[<-] (1.55,4.3) -- (3.4,4.3) node[right,align=center] {on dépose des grains de sable doucement,\\jusqu'à obtenir la masse \(m\) sur le piston};
\end{tkz}

Le piston descend continûment, la transformation est infiniment lente, la température et la pression sont définies à tout instant. Le système est en équilibre avec le milieu extérieur. Donc la transformation est réversible.

Voyons un autre exemple, la détente d'un gaz :

\begin{tkz}
\draw (0,0) -- (3,0) -- (3,1) -- (6,1) -- (6,0) -- (9,0) -- (9,3) -- (6,3) -- (6,2) -- (3,2) -- (3,3) -- (0,3) -- (0,0); % contenant
\draw (4.5,2.5) -- (4.5,0.5); % robinet
\draw (4.4,2.5) -- (4.6,2.5); % robinet
\foreach \i in {1,...,142} \fill (3+rnd*1.5,1+rnd) circle (0.01); % gaz dans le couloir
\foreach \i in {1,...,858} \fill (3*rnd,3*rnd) circle (0.01); % gaz dans le carré
\draw[<-] (1.5,0.3) -- (1.5,-0.4) node[below] {gaz};
\draw[<-] (7.5,0.3) -- (7.5,-0.4) node[below] {\(\ensvide\)};
\end{tkz}

À \(t=0\), on ouvre le robinet :

\begin{tkz}
\draw (0,0) -- (3,0) -- (3,1) -- (6,1) -- (6,0) -- (9,0) -- (9,3) -- (6,3) -- (6,2) -- (3,2) -- (3,3) -- (0,3) -- (0,0); % contenant
\draw (4.5,4.5) -- (4.5,2.5); % robinet
\draw (4.4,4.5) -- (4.6,4.5); % robinet
\foreach \i in {1,...,160} \fill (3+rnd*3,1+rnd) circle (0.01); % gaz dans le couloir
\foreach \i in {1,...,420} \fill (3*rnd,3*rnd) circle (0.01); % gaz dans le carré de gauche
\foreach \i in {1,...,420} \fill (6+3*rnd,3*rnd) circle (0.01); % gaz dans le carré de droite
\end{tkz}

La transformation est irréversible : même en ouvrant le robinet doucement, la transformation est rapide. On n'a pas une succession d'états d'équilibre.

\subsubsection{Transformations particulières}

\(\begin{drcases}
T_{\paren{\Sigma}}=\cte&\text{transformation isotherme } \\
P_{\paren{\Sigma}}=\cte&\text{transformation isobare}
\end{drcases}\text{transformations réversibles}\)

\(\begin{drcases}
V_{\paren{\Sigma}}=\cte&\text{transformation isochore} \\
T_{\ext}=\cte&\text{transformation monotherme } \\
P_\ext=\cte&\text{transformation monobare}
\end{drcases}\text{transformations irréversibles}\)

\subsection{Énergie}

Rappel : dans un référentiel galiléen, on peut définir l'énergie interne \(U\) comme : \[U=E_{c_\text{micro}}+E_{p_\text{micro}}\] où \begin{description}
\item \(E_{c_\text{micro}}\) : la somme des énergies cinétiques microscopiques
\item \(E_{p_\text{micro}}\) : la somme des énergies potentielles microscopiques\\
\end{description} et l'énergie mécanique \(E_m\) comme : \[E_m=E_{c_\text{macro}}+E_{p_\text{macro}}\] avec \(E_{c_\text{macro}}=\dfrac{1}{2}m_{\paren{\Sigma}}v_{\paren{\Sigma}}^2\).

On définit alors l'énergie totale \(E\) comme : \[E=U+E_m.\]

C'est l'énergie totale stockée dans le système.

\subsection{Échanges d'énergie}

On reprend l'exemple du cylindre rempli de gaz avec un piston mobile :

\begin{tkz}[scale=1.5]
\draw (0,4.5) -- (0,0) -- (3,0) -- (3,4.5); % cylindre
\draw (0,3) -- (3,3); % haut piston
\draw (0,2.8) -- (3,2.8); % bas piston
\draw (1,3) -- (1,4) -- (2,4) -- (2,3); % masse
\foreach \i in {1,...,1000} \fill (rnd*3,rnd*2.8) circle (0.01); % atomes
\draw[->,red] (1.5,3.5) node[above,black] {\(m\)} -- (1.5,2.3) node[left] {\(\vec{P}\)}; % vecteur poids
\draw[<-] (2.7,2.9) -- (3.4,2.9) node[right] {piston mobile};
\draw[<-] (2.7, 0.3) -- (3.4,0.3) node[right] {gaz};
\end{tkz}

Si on pose une masse \(m\) sur le piston mobile, celui-ci s'enfonce. Donc la masse a cédé une partie de son énergie potentielle au système.

Expérimentalement, on remarque que si la température augmente, l'énergie interne augmente.

Il y a eu transfert d'énergie entre le travail du poids et l'énergie interne du système.

On plonge le cylindre dans une eau chaude :

\begin{tkz}[scale=1.5]
\draw (0,6.5) -- (0,0) -- (6,0) -- (6,6.5); % bac
\draw (0,4) -- (2,4); % eau
\draw (5,4) -- (6,4); % eau
\draw (2,6) -- (2,1.5) -- (5,1.5) -- (5,6); % cylindre
\draw (2,4.5) -- (5,4.5); % haut piston
\draw (2,4.3) -- (5,4.3); % bas piston
\foreach \i in {1,...,1000} \fill (2+rnd*3,1.5+rnd*2.8) circle (0.01); % atomes
\draw[<-] (4.7,4.4) -- (6.4,4.4) node[right] {piston mobile};
\draw[<-] (4.7, 1.8) -- (6.4,1.8) node[right] {gaz};
\draw[->] (-0.4,0.3) node[left] {eau chaude} -- (0.3,0.3);
\end{tkz}

Alors, la température du gaz augmente et donc son énergie interne aussi.

Il y a eu transfert d'énergie entre l'eau et le gaz.

Le transfert d'énergie sans transfert de matière ni déplacement s'appelle le transfert thermique \(Q\).

Le transfert thermique s'interprète au niveau microscopique comme l'agitation thermique se propageant de proche en proche à travers le cylindre.

On a le tableau suivant :

\begin{center}
\begin{tabular}{|l|c|c|}
\hline
\diagbox{Grandeur\\énergétique}{Échelle} & Macroscopique & Microscopique \\
\hline
Énergie & \(E_m\) & \(U\) \\
\hline
Transfert d'énergie & \(W\) & \(Q\) \\
\hline
\end{tabular}
\end{center}

\section{Premier principe de la thermodynamique}

\subsection{Énoncé}

Pour tout système fermé \(\paren{\Sigma}\), on peut définir une fonction \(U\) dépendant des variables d'état extensives et telle que l'énergie totale du système \(E=E_m+U\) soit conservative, c'est-à-dire se conserve si \(\paren{\Sigma}\) est isolé. \(U\) est l'énergie interne du système.

C'est un principe de conservation. Il traduit l'impossibilité de création d'énergie.

\subsection{Bilan d'énergie}

Si l'on fait un bilan d'énergie sur \(\paren{\Sigma}\), on obtient : \[\adif{E}=\adif{\paren{E_m+U}}=W+Q.\]

Si \(\paren{\Sigma}\) est au repos (\(E_{c_\text{macro}}=\cte\) et \(E_{p_\text{macro}}=\cte\)) alors \(\adif{E_m}=0\) donc : \[\adif{U}=W+Q.\] C'est la forme réduite du premier principe de la thermodynamique (Carnot, 1850). Cette expression donne l'équivalence travail/transfert thermique (Joule).

On peut donc faire varier \(U\) à l'aide soit d'un travail \(W\), soit d'un transfert thermique \(Q\).

Pour un système en évolution cyclique, les états d'équilibre initiaux et finaux sont égaux. Donc : \[\adif{U}=0.\]

Convention : \(W\) et \(Q\) sont toujours orientés vers le système.

Si \(Q>0\) ou \(W>0\), \(U\) augmente car le système reçoit du travail ou du transfert thermique du milieu extérieur.

Si \(Q<0\) ou \(W<0\), \(U\) diminue car le système fournit du travail ou du transfert thermique au milieu extérieur.

\section{Travail des forces de pression}

\subsection{Expression générale}

On considère un fluide \(\Sigma\) dans un cylindre avec un piston mobile de surface \(S\) :

\begin{tkz}[scale=1.8]
\draw[->] (-1,0) -- (4.5,0) node[below right] {\(x\)}; % axe
\draw[->,blue,thick] (0,0) -- (1,0) node[below right] {\(\vec{u}_x\)}; % vecteur unitaire ux
\draw (0,0.1) -- (0,-0.1) node[below] {\(O\)}; % graduation O
\draw (2,0.1) -- (2,-0.1) node[below] {\(x\)}; % graduation x
\draw (2.8,0.1) -- (2.8,-0.1) node[below] {\(x+\odif{x}\)}; % graduation x+dx
\draw (3.5,3) -- (0,3) -- (0,1) node[above right] {\(\Sigma\)} -- (3.5,1); % cylindre
\draw (2,3) -- (2,1); % piston gauche
\draw (2.8,3) -- (2.8,1); % piston droit
\draw[<->] (1.9,3) -- (1.9,1) node[left,pos=0.5] {\(S\)}; % surface S
\draw[<->] (2,3.1) -- (2.8,3.1) node[above,pos=0.5] {\(\odif{x}\)}; % distance dx
\fill[pattern=north east lines] (2,3) -- (2.8,3) -- (2.8,1) -- (2,1); % volume dV
\draw[<-] (2.5,2.7) -- (3.8,2.7) node[right] {\(\odif{V}\)};
\node at (3.4,1.5) {\(P_\ext\)};
\draw[<-] (2,1) -- (2.4,0.6);
\draw[<-] (2.8,1) -- (2.4,0.6) node[below] {piston mobile};
\end{tkz}

On exerce une contrainte (\(P_\ext\)) sur le piston. Si la transformation de \(\Sigma\) est quelconque, la pression \(P\) et la température \(T\) du fluide ne sont pas forcément définies. La pression extérieure l'est toujours.

Le piston se déplace de \(\odif{x}\) donc on a un accroissement du volume de \(\Sigma\) de : \[\odif{V}=S\odif{x}.\]

Ainsi, on a la force pressante \(\vec{F}\) exercée par le milieu extérieur sur \(\Sigma\) : \[\vec{F}=-SP_\ext\vec{u}_x.\]

D'où le travail élémentaire : \[\fdif{W}=\vec{F}\scal\odif{\vec{l}}=-SP_\ext\odif{x}=-P_\ext\odif{V}.\]

\(\fdif{W}\) est algébrique : \(\fdif{W}\supinf0\).

Si \(\odif{V}<0\) (compression) alors \(\fdif{W}>0\) : le système reçoit du travail du milieu extérieur.

Si \(\odif{V}>0\) (détente) alors \(\fdif{W}<0\) : le système fournit du travail au milieu extérieur.

Au cours d'une transformation non-élémentaire, on a le travail : \[W=\int\fdif{W}=\int_{V_1}^{V_2}-P_\ext\odif{V}=-\int_{V_1}^{V_2}P_\ext\odif{V}.\]

Donc on a une formule pour calculer le travail des forces de pression.

Si la transformation est réversible, \(\Sigma\) est en équilibre avec le milieu extérieur donc \(P=P_\ext\). On a donc : \[W=-\int_{V_1}^{V_2}P\odif{V}.\]

On a besoin de \(P\paren{V}\) pour obtenir l'équation d'état.

\subsection{Transformations usuelles}

\subsubsection{Transformation isochore}

On a \(V=\cte\) donc \(\odif{V}=0\) donc : \[W=0.\]

\subsubsection{Transformation monobare}

On a \(P_\ext=\cte\) donc : \[W=-P_\ext\int_{V_1}^{V_2}\odif{V}=-P_\ext\paren{V_2-V_1}.\]

\subsubsection{Transformation isobare}

On a \(P=P_\ext=\cte\) donc : \[W=-P\paren{V_2-V_1}.\]

\subsubsection{Transformation isotherme}

On a \(T=\cte\) et \(P=P_\ext\) donc : \[W=-\int_{V_1}^{V_2}P\odif{V}.\]

Si on a un gaz parfait, on a \(PV=nRT\) donc : \[P=\dfrac{nRT}{V}.\]

Donc on a : \[W=-\int_{V_1}^{V_2}\dfrac{nRT}{V}\odif{V}=-nRT\ln\paren{\dfrac{V_2}{V_1}}.\]

\subsection{Représentation graphique}

Dans le cas d'une transformation réversible, on a \(W=-\int_{V_1}^{V_2}P\odif{V}\).

Donc le travail est l'aire sous la courbe \(P\paren{V}\) :

\begin{tkz}
\begin{axis}[axis lines=left,
xlabel={\(V\)},
ylabel={\(P\)},
xmin=0,xmax=6,
ymin=0,ymax=2.5,
xtick={1,4},
xticklabels={\(V_1\),\(V_2\)},
ytick={0},
xlabel style={at={(axis description cs:1,0)},anchor=north west},
ylabel style={at={(axis description cs:0,1)},anchor=south east,rotate=-90}]
\addplot[name path=A,domain=1:4,samples=1000] {exp(-x)+1};
\addplot[domain=0:6,samples=1000] {exp(-x)+1};
\addplot[name path=B,domain=1:4,samples=1000] {0};
\addplot[pattern={Lines[angle=45,distance=3mm]}] fill between [of=A and B];
\end{axis}
\end{tkz}

Un tel diagramme est appelé diagramme de Watt. On a \(\abs{W}=\begin{tikzpicture}
\filldraw[pattern={Lines[angle=45,distance=3mm]}] (0,-0.25) -- (0,0.25) -- (1,0.25) -- (1,-0.25) -- (0,-0.25);
\end{tikzpicture}\)

\subsubsection{Transformation isobare}

On a \(P=\cte\) :

\begin{tkz}
\begin{axis}[axis lines=left,
xlabel={\(V\)},
ylabel={\(P\)},
xmin=0,xmax=6,
ymin=0,ymax=2.5,
xtick={1,4},
xticklabels={\(V_1\),\(V_2\)},
ytick={0},
xlabel style={at={(axis description cs:1,0)},anchor=north west},
ylabel style={at={(axis description cs:0,1)},anchor=south east,rotate=-90}]
\addplot[name path=A,domain=1:4,samples=1000] {2};
\addplot[domain=0:6,samples=1000] {2};
\addplot[name path=B,domain=1:4,samples=1000] {0};
\addplot[pattern={Lines[angle=45,distance=3mm]}] fill between [of=A and B];
\end{axis}
\end{tkz}

On a \(W=-P\paren{V_2-V_1}=-\,\begin{tikzpicture}
\filldraw[pattern={Lines[angle=45,distance=3mm]}] (0,-0.25) -- (0,0.25) -- (1,0.25) -- (1,-0.25) -- (0,-0.25);
\end{tikzpicture}\)

\subsubsection{Évolution cyclique}

\begin{tkz}
\begin{axis}[axis lines=left,
xlabel={\(V\)},
ylabel={\(P\)},
xmin=0,xmax=6,
ymin=0,ymax=3,
xtick={1,5},
xticklabels={\(V_1\),\(V_2\)},
ytick={0},
xlabel style={at={(axis description cs:1,0)},anchor=north west},
ylabel style={at={(axis description cs:0,1)},anchor=south east,rotate=-90},
trig format plots=rad]
%\draw (axis cs:3,1.5) ellipse (2 and 0.8);
\addplot[name path=A,domain=0:pi,samples=1000] ({3+2*cos(x)},{1.5+0.8*sin(x)});
\addplot[name path=C,domain=pi:2*pi,samples=1000] ({3+2*cos(x)},{1.5+0.8*sin(x)});
\addplot[name path=B,domain=1:5,samples=1000] {0};
\addplot[pattern={Lines[angle=-45,distance=3mm]},pattern color=blue] fill between [of=A and B];
\addplot[pattern={Lines[angle=45,distance=3mm]},pattern color=red] fill between [of=C and B];
\addplot[pattern={Lines[angle=90,distance=3mm]}] fill between [of=C and A];
\addplot[domain=0:2*pi,samples=1000,decoration={markings,mark=at position 0.25 with {\arrow{<}},mark=at position 0.75 with {\arrow{<}}},postaction={decorate},very thick] ({3+2*cos(x)},{1.5+0.8*sin(x)});
\end{axis}
\end{tkz}

De \(V_1\) à \(V_2\) on a : \[W_\text{aller}=-\int_{V_1}^{V_{2}}P\odif{V}=-\,\begin{tikzpicture}
\filldraw[pattern={Lines[angle=-45,distance=3mm]},pattern color=blue] (0,-0.25) -- (0,0.25) -- (1,0.25) -- (1,-0.25) -- (0,-0.25);
\end{tikzpicture}\]

De \(V_2\) à \(V_1\) on a : \[W_\text{retour}=-\int_{V_2}^{V_{1}}P\odif{V}=\begin{tikzpicture}
\filldraw[pattern={Lines[angle=45,distance=3mm]},pattern color=red] (0,-0.25) -- (0,0.25) -- (1,0.25) -- (1,-0.25) -- (0,-0.25);
\end{tikzpicture}\]

Donc on a : \[W_\text{cycle}=W_\text{aller}+W_\text{retour}=-\,\begin{tikzpicture}
\filldraw[pattern={Lines[angle=90,distance=3mm]}] (0,-0.25) -- (0,0.25) -- (1,0.25) -- (1,-0.25) -- (0,-0.25);
\end{tikzpicture}\] donc : \[\abs{W_\text{cycle}}=A_\text{cycle}\] où \(A_\text{cycle}\) désigne l'aire du cycle.

Si on a un cycle dans ce sens : \begin{tikzpicture}
\draw[decoration={markings,mark=at position 1 with {\arrow{<}}},postaction={decorate}] (0,0) circle (0.5);
\end{tikzpicture} alors le système est moteur : il fournit du travail au milieu extérieur.

Si on a un cycle dans ce sens : \begin{tikzpicture}
\draw[decoration={markings,mark=at position 1 with {\arrow{>}}},postaction={decorate}] (0,0) circle (0.5);
\end{tikzpicture} alors le système est récepteur : il reçoit du travail du milieu extérieur.

Le travail dépend du chemin suivi (il n'est donc pas conservatif) :

\begin{tkz}
\begin{axis}[axis lines=left,
xlabel={\(V\)},
ylabel={\(P\)},
xmin=0,xmax=6,
ymin=0,ymax=2.5,
ytick={0},
xmajorticks=false,
xlabel style={at={(axis description cs:1,0)},anchor=north west},
ylabel style={at={(axis description cs:0,1)},anchor=south east,rotate=-90}]
\fill[pattern={Lines[angle=45,distance=3mm]},pattern color=blue] (1,2) -- (4,2) -- (4,0) -- (1,0) -- (1,2);
\fill[pattern={Lines[angle=-45,distance=3mm]},pattern color=red] (1,2) -- (4,1) -- (4,0) -- (1,0) -- (1,2);
\addplot[name path=A,samples=1000,blue,ultra thick,decoration={markings,mark=at position 0.5 with {\arrow{>}}},postaction={decorate}] coordinates {(1,2) (4,2) (4,1)};
\addplot[name path=C,samples=1000,red,ultra thick,decoration={markings,mark=at position 0.5 with {\arrow{>}}},postaction={decorate}] coordinates {(1,2) (4,1)};
\end{axis}
\end{tkz}

On a bien \(\begin{tikzpicture}
\filldraw[pattern={Lines[angle=-45,distance=3mm]},pattern color=red] (0,-0.25) -- (0,0.25) -- (1,0.25) -- (1,-0.25) -- (0,-0.25);
\end{tikzpicture}\,\not=\,\begin{tikzpicture}
\filldraw[pattern={Lines[angle=45,distance=3mm]},pattern color=blue] (0,-0.25) -- (0,0.25) -- (1,0.25) -- (1,-0.25) -- (0,-0.25);
\end{tikzpicture}\)

\section{Transferts thermiques}

\subsection{Méthode de calcul}

D'après le premier principe de la thermodynamique, on a \(\adif{U}=W+Q\) donc on a : \[Q=\adif{U}-W.\] On doit calculer \(\adif{U}\) et \(W\) pour calculer \(Q\).

\subsection{Transformations particulières}

\subsubsection{Transformation isochore}

On a \(V=\cte\) donc \(W=0\) donc \[Q_V=\adif{U}.\]

\subsubsection{Transformation monobare}

On a \(P_\ext=\cte\) donc \(W=-P_\ext\paren{V_2-V_1}\) donc on a : \[\begin{aligned}
Q&=\adif{U}-W \\
&=U_2-U_1+P_\ext\paren{V_2-V_1} \\
&=U_2+P_\ext V_2-\paren{U_1+P_\ext V_1}.
\end{aligned}\]

On pose \(H^*=U+P_\ext V\) et on a donc : \[Q=H_2^*-H_1^*.\]

\(H^*\) est en fonction de \(P_\ext\) donc ce n'est pas une fonction d'état.

On impose donc une nouvelle contrainte : le système doit être en équilibre avec le milieu extérieur aux états initial et final. Ainsi, à l'état initial on a \(P_1=P_\ext\), et à l'état final on a \(P_2=P_\ext\).

On définit donc la fonction d'état appelée enthalpie par : \[H=U+PV.\]

Ainsi, pour une transformation monobare entre états d'équilibre, on a : \[Q=H_2-H_1=\adif{H}.\]

\subsection{Capacités thermiques}

On travaille avec un corps pur monophasé soumis aux seules forces de pression.

Les paramètres d'état \(T\), \(P\) et \(V\) sont liés par une équation d'état. Seuls deux suffisent pour décrire le système.

\subsubsection{Choix du couple \(\paren{T,V}\) (variables naturelles de \(U\))}

On a : \[\begin{WithArrows}
U&=U\paren{T,V} \Arrow{\(\dif\)} \\
\odif{U}&=\pdv{U}{T}_V\odif{T}+\pdv{U}{V}_T\odif{V}.
\end{WithArrows}\]

On définit la capacité thermique à volume constant (en \(\unit{\joule\per\kelvin}\)) par : \[C_V=\pdv{U}{T}_V\]

On considère une transformation isochore. On a donc \(Q_V=\adif{U}\) d'après le premier principe de la thermodynamique.

De plus, on a \(\odif{U}=C_V\odif{T}+0\) donc : \[\adif{U}=\int_{T_1}^{T_2}C_V\odif{T}.\]

Enfin, si \(C_V=\cte\) alors on a : \[\adif{U}=C_V\paren{T_2-T_1}.\]

\(C_V\) est donnée par les tables thermodynamiques.

On peut donc calculer \(\adif{U}\) et \(Q\) à partir des températures \(T_1\) et \(T_2\).

\subsubsection{Choix du couple \(\paren{T,P}\) (variables naturelles de \(H\))}

On a : \[\begin{WithArrows}
H&=H\paren{T,P} \Arrow{\(\dif\)} \\
\odif{H}&=\pdv{H}{T}_{P}\odif{T}+\pdv{H}{P}_{T}\odif{P}.
\end{WithArrows}\]

On définit la capacité thermique à pression constante (en \(\unit{\joule\per\kelvin}\)) par : \[C_P=\pdv{H}{T}_P\]

On définit aussi les capacités thermiques à pression constante molaire et massique par : \[C_{P_m}=\dfrac{C_P}{n}\quad\text{et}\quad c_P=\dfrac{C_P}{m}.\]

On a : \[\begin{WithArrows}
H&=U+PV \Arrow[i]{\(\dif\)} \\
\odif{H}&=\odif{U}+\odif{\paren{PV}} \\
&=\odif{U}+V\odif{P}+P\odif{V} \\
&=\fdif{W}+\fdif{Q}+V\odif{P}+P\odif{V}.
\end{WithArrows}\]

On considère une transformation isobare. On a donc \(P_\ext=P=\cte\). On a donc : \[\fdif{W}=-P_\ext\odif{V}=-P\odif{V}.\] Donc on a : \[\begin{WithArrows}
\odif{H}&=\fdif{Q} \Arrow{\(\textstyle\int\)} \\
\adif{H}&=Q_P.
\end{WithArrows}\]

On a aussi : \[\begin{WithArrows}
\odif{H}&=C_P\odif{T} \Arrow{\(\textstyle\int\)} \\
\adif{H}&=\int_{T_1}^{T_2}C_P\odif{T}.
\end{WithArrows}\]

Enfin, si \(C_P=\cte\), on a : \[\adif{H}=C_P\paren{T_2-T_1}.\]

\(C_P\) est donnée par les tables thermodynamiques.

On peut donc calculer \(\adif{H}\) à partir des températures \(T_1\) et \(T_2\).

Par exemple, calculons la variation d'enthalpie associée au passage d'un mélange gazeux de la température \(T_1=\SI{298}{\kelvin}\) à \(T_2=\SI{600}{\kelvin}\). On a \(C_{P_m}=\paren{\num{31.4}+\num{2.1e-2}T}\unit{\joule\per\kelvin\per\mole}\) (valable sur \(\intervii{T_1}{T_2}\)). On a \(n=\SI{1}{\mole}\). On a donc : \[\begin{aligned}
\adif{H}&=\int_{T_1}^{T_2}C_P\odif{T} \\
&=\int_{T_1}^{T_2}nC_{P_m}\odif{T} \\
&=\int_{T_1}^{T_2}\paren{\num{31.4}+\num{2.1e-2}T}\odif{T} \\
&=\croch{\num{31.4}T+\dfrac{\num{2.1e-2}}{2}T^2}_{T_1}^{T_2} \\
&=\SI{12.3}{\kilo\joule}
\end{aligned}\]

Pour un gaz parfait, on a : \[\begin{WithArrows}
H&=U+PV \Arrow{\(\pdv{}{T}\)} \\
\pdv{H}{T}_P&=\pdv{U}{T}_V+\pdv{\paren{PV}}{T}_{P\text{ ou }V} \\
\pdv{H}{T}_P&=\pdv{U}{T}_V+\pdv{\paren{nRT}}{T}_{P\text{ ou }V} \\
C_P&=C_V+nR.
\end{WithArrows}\]

On en déduit les relations : \[C_{P_m}=C_{V_m}+R\] et \[\begin{aligned}
c_P&=c_V+\dfrac{n}{m}R \\
&=c_V+\dfrac{R}{M}
\end{aligned}\] où \(M\) est la masse molaire du gaz considéré.

On a donc les relations de Mayer pour les gaz parfaits : \[C_P=C_V+nR\qquad C_{P_m}=C_{V_m}+R\qquad c_P=c_V+\dfrac{R}{M}.\]

On définit l'indice adiabatique (sans unité) par : \[\gamma=\dfrac{C_P}{C_V}.\]

On obtient deux relations (uniquement valables pour les gaz parfaits) : \[C_{P_m}=\dfrac{\gamma R}{\gamma-1}\quad\text{et}\quad C_{V_m}=\dfrac{R}{\gamma-1}.\]

Par exemple : \begin{itemize}
\item pour un gaz parfait monoatomique, on a \(C_{V_m}=\dfrac{3}{2}R\) et \(C_{P_m}=\dfrac{5}{2}R\) donc \(\gamma=\dfrac{5}{3}=\num{1.67}\) ;
\item pour un gaz parfait diatomique, on a \(C_{V_m}=\dfrac{5}{2}R\) et \(C_{P_m}=\dfrac{7}{2}R\) donc \(\gamma=\dfrac{7}{5}=\num{1.4}\) ;
\item pour l'air ambiant, on mesure \(\gamma=\num{1.41}\) ce qui confirme la large majorité du dioxygène et du diazote dans sa composition. \\
\end{itemize}

Pour les phases condensées, \ie indilatables et incompressibles, quelles que soient la température et la pression, on a \(V=\cte\). Donc dans les conditions normales de température et de pression, on a : \[H=U\] et la capacité thermique (en \(\unit{\joule\per\kelvin}\)) : \[C=C_V=C_P.\]

Si \(C\) est une constante dans l'intervalle de températures considéré alors on a : \[\adif{H}=\adif{U}=C\adif{T}=Q\] car \(\adif{U}=W+Q\) avec \(W=-\int P_\ext\odif{V}=0\).

Ordres de grandeur : \begin{itemize}
\item on a \(C_\text{eau}=\SI{4.185}{\kilo\joule\per\kelvin\per\kilo\gram}=\SI{1}{\cal\per\kelvin\per\gram}\), avec \(\SI{1}{\cal}=\SI{4.185}{\joule}\) (une calorie est l'énergie qu'il faut fournir à \(\SI{1}{\gram}\) d'eau pour élever sa température de \(\SI{1}{\kelvin}\) ou de \(\SI{15}{\degreeCelsius}\) à \(\SI{16}{\degreeCelsius}\)) ;
\item on a \(C_\text{fer}=\SI{0.46}{\kilo\joule\per\kelvin\per\kilo\gram}\approx\num{0.1}C_\text{eau}\).
\end{itemize}

\section{Transformation adiabatique d'un gaz parfait}

\subsection{Transformation réversible}

On passe d'un état initial avec une pression \(P_1\), une température \(T_1\) et un volume \(V_1\) à un état final avec une pression \(P_2\), une température \(T_2\) et un volume \(V_2\) par le biais d'une transformation adiabatique (donc \(Q=0\) et \(T\not=T_\ext\)) et réversible (donc \(P=P_\ext\not=\cte\)) :

\begin{tkz}[scale=1.6]
\filldraw[pattern={Lines[angle=-45,distance=5mm]}] (0,0) -- (5,0) -- (5,4) -- (4,4) -- (4,1) node[above left] {\(\Sigma\)} -- (1,1) node[above right] {\(P,T,V\)} -- (1,4) -- (0,4) -- (0,0); % enceinte isolée
\filldraw[pattern={Lines[angle=-45,distance=5mm]}] (1,3.5) -- (4,3.5) -- (4,3.3) -- (1,3.3) -- (1,3.5); % piston isolé
\node at (2.5,4.2) {\(P_\ext\)};
\draw[<-] (3.7,3.4) -- (5.4,3.4) node[right] {piston mobile (\(P=P_\ext\))};
\draw[<-] (4.7,0.3) -- (5.4,0.3) node[right] {enceinte isolée (\(T\not=T_\ext\))};
\end{tkz}

D'après le premier principe de la thermodynamique, on a \(\adif{U}=W+Q\) donc pour une transformation infinitésimale, on a : \[\odif{U}=\fdif{W}+\fdif{Q}\] avec : \[\odif{U}=C_V\odif{T}=\dfrac{nR}{\gamma-1}\odif{T}\] et, comme on a une transformation réversible : \[\fdif{W}=-P_\ext\odif{V}=-P\odif{V}\] et, comme on a une transformation adiabatique : \[\fdif{Q}=0.\]

D'où : \[\dfrac{nR}{\gamma-1}\odif{T}=-P\odif{V}.\]

Or on a un gaz parfait donc \(PV=nRT\) donc on a : \[T=\dfrac{PV}{nR}.\]

Donc on a : \[\begin{WithArrows}
\dfrac{nR}{\gamma-1}\odif{\paren{\dfrac{PV}{nR}}}&=-P\odif{V} \\
\dfrac{1}{\gamma-1}\paren{P\odif{V}+V\odif{P}}&=-P\odif{V} \\
P\odif{V}+V\odif{P}&=-\paren{\gamma-1}P\odif{V} \\
V\odif{P}+\gamma P\odif{V}&=0 \Arrow[i]{\(\div PV\)} \\
\dfrac{\odif{P}}{P}+\gamma\dfrac{\odif{V}}{V}&=0 \Arrow[i]{\(\textstyle\int\)} \\
\ln P+\gamma\ln V&=\cte \\
\ln\paren{PV^\gamma}&=\cte.
\end{WithArrows}\]

On obtient finalement la loi de Laplace pour un gaz parfait en évolution adiabatique et réversible : \[PV^\gamma=\cte.\]

Par exemple, prenons un gaz parfait d'indice adiabatique \(\gamma=\num{1.4}\) passant d'une pression \(P_1=\SI{1}{\bar}\), d'un volume \(V_1=\SI{1}{\liter}\) et d'une température \(T_1=\SI{300}{\kelvin}\) à une pression \(P_2\), un volume \(V_2=\SI{0.1}{\liter}\) et une température \(T_2\) par le biais d'une transformation adiabatique et réversible.

D'après la loi de Laplace, on a \(PV^\gamma=\cte\) donc on a : \[P_1V_1^\gamma=P_2V_2^\gamma.\]

Donc on a : \[P_2=P_1\paren{\dfrac{V_1}{V_2}}^\gamma=1\times10^{\num{1.4}}=\SI{25.1}{\bar}.\]

De plus, on a \(PV=nRT\) donc \(P=\dfrac{nRT}{V}\) donc d'après d'après la loi de Laplace, on a : \[\dfrac{nRT}{V}\times V^\gamma=\cte.\]

On en déduit \(TV^{\gamma-1}=\cte\) donc \(T_1V_1^{\gamma-1}=T_2V_2^{\gamma-1}\) donc on a : \[T_2=T_1\paren{\dfrac{V_1}{V_2}}^{\gamma-1}=300\times10^{\num{0.4}}=\SI{754}{\kelvin}.\]

D'où \(T\not=\cte\) (transformation non-isotherme).

De plus, on obtient les lois de Laplace, en combinant la loi de Laplace pour un gaz parfait en évolution adiabatique et réversible et la loi des gaz parfaits : \[PV^\gamma=\cte\qquad TV^{\gamma-1}=\cte\qquad P^{1-\gamma}T^\gamma=\cte.\]

\subsection{Cas d'une évolution irréversible}

On reprend l'exemple précédent mais on supprime l'hypothèse de réversibilité :

On passe d'un état initial avec une pression \(P_1=\SI{1}{\bar}\), une température \(T_1=\SI{300}{\kelvin}\) et un volume \(V_1=\SI{1}{\liter}\) à un état final avec une pression \(P_3=\SI{25.1}{\bar}\), une température \(T_3\) et un volume \(V_3\) par le biais d'une transformation adiabatique (donc \(Q=0\) et \(T\not=T_\ext\)) :

\begin{tkz}[scale=1.6]
\filldraw[pattern={Lines[angle=-45,distance=5mm]}] (0,0) -- (5,0) -- (5,4) -- (4,4) -- (4,1) node[above left] {\(\Sigma\)} -- (1,1) node[above right] {\(P_0,T_0,V_0\)} -- (1,4) -- (0,4) -- (0,0); % enceinte isolée
\filldraw[pattern={Lines[angle=-45,distance=5mm]}] (1,3.5) -- (4,3.5) -- (4,3.3) -- (1,3.3) -- (1,3.5); % piston isolé
\draw (2,3.5) -- (2,4.5) -- (3,4.5) -- (3,3.5); % masse
\draw (1,3.3) -- (1.3,3.3) -- (1.3,3) -- (1,3); % cale
\node at (3.5,4.2) {\(P_0\)};
\node at (2.5,4) {\(m\)};
\draw[<-] (3.7,3.4) -- (5.4,3.4) node[right] {piston mobile};
\draw[->] (-0.4,3.15) node[left] {cale} -- (1.15,3.15);
\end{tkz}

À \(t=0\), on enlève la cale :

\begin{tkz}[scale=1.6]
\filldraw[pattern={Lines[angle=-45,distance=5mm]}] (0,0) -- (5,0) -- (5,4) -- (4,4) -- (4,1) node[above left] {\(\Sigma\)} -- (1,1) node[above right] {\(P_3,T_3,V_3\)} -- (1,4) -- (0,4) -- (0,0); % enceinte isolée
\filldraw[pattern={Lines[angle=-45,distance=5mm]}] (1,2.5) -- (4,2.5) -- (4,2.3) -- (1,2.3) -- (1,2.5); % piston isolé
\draw (2,2.5) -- (2,3.5) -- (3,3.5) -- (3,2.5); % masse
\node at (3.5,3.7) {\(P_0\)};
\node at (2.5,3) {\(m\)};
\end{tkz}

On a : \[P_\ext=P_0+\dfrac{mg}{S}.\]

La transformation est brutale donc \(P\) n'est pas définie à tout instant dans le système.

De plus, elle est monobare donc on a : \[P_\ext=P_0+\dfrac{mg}{S}=\cte\] mais pas entre états d'équilibre.

À l'équilibre mécanique, on a : \[P_3=P_\ext=P_0+\dfrac{mg}{S}=\SI{25.1}{\bar}.\]

Comme la transformation est adiabatique, on a \(Q=0\). Donc d'après le premier principe de la thermodynamique, on a : \[\adif{U}=W.\]

Or on a : \[\adif{U}=\dfrac{nR}{\gamma-1}\paren{T_3-T_1}=\dfrac{P_1V_1}{\paren{\gamma-1}T_1}\paren{T_3-T_1}\] et : \[W=-\int_{V_1}^{V_3}P_\ext\odif{V}=-\int_{V_1}^{V_3}\paren{P_0+\dfrac{mg}{S}}\odif{V}=-\paren{P_0+\dfrac{mg}{S}}\paren{V_3-V_1}.\]

Donc on a : \[\dfrac{P_1V_1}{\paren{\gamma-1}T_1}\paren{T_3-T_1}=-\paren{P_0+\dfrac{mg}{S}}\paren{V_3-V_1}.\]

On en déduit : \[V_3=\SI{0.31}{\liter}\qquad P_3=\SI{25.1}{\bar}\qquad T_3=\SI{2366}{\kelvin}.\]

Si l'on compare ces résultats avec ceux trouvés pour la transformation réversible, on remarque que l'hypothèse de réversibilité n'est pas anodine.

Interprétation physique (travail) :

\begin{center}
\begin{tikzpicture}
\begin{axis}[axis lines=left,
title=Transformation réversible,
xlabel={\(V\)},
ylabel={\(P\)},
xmin=0,xmax=6,
ymin=0,ymax=2.5,
xtick={1,4},
xticklabels={\(V_2\),\(V_1\)},
ytick={0,1.018,1.368},
yticklabels={\(0\),\(P_1\),\(P_2\)},
xlabel style={at={(axis description cs:1,0)},anchor=north west},
ylabel style={at={(axis description cs:0,1)},anchor=south east,rotate=-90}]
\addplot[name path=A,domain=1:4,samples=1000,] {exp(-x)+1};
\addplot[domain=1:4,samples=1000,decoration={markings,mark=at position 0 with {\filldraw circle (2pt);},mark=at position 0.5 with {\arrow{<}},mark=at position 1 with {\filldraw circle (2pt);}},postaction={decorate},very thick] {exp(-x)+1};
\addplot[name path=B,domain=1:4,samples=1000] {0};
\addplot[pattern={Lines[angle=45,distance=3mm]},pattern color=blue] fill between [of=A and B];
\node[right] at (4,1.018) {\(P=\dfrac{\cte}{V^{1.4}}\)};
\end{axis}
\end{tikzpicture}
\hskip5pt
\begin{tikzpicture}
\begin{axis}[axis lines=left,
title=Transformation irréversible,
xlabel={\(V\)},
ylabel={\(P\)},
xmin=0,xmax=6,
ymin=0,ymax=2.5,
xtick={1,4},
xticklabels={\(V_3\),\(V_1\)},
ytick={0,1,2},
yticklabels={\(0\),\(P_1\),\(P_3\)},
xlabel style={at={(axis description cs:1,0)},anchor=north west},
ylabel style={at={(axis description cs:0,1)},anchor=south east,rotate=-90}]
\addplot[name path=A,domain=1:4,samples=1000] {2};
\addplot[domain=1:4,samples=1000,decoration={markings,mark=at position 0 with {\filldraw circle (2pt);},mark=at position 0.5 with {\arrow{<}}},postaction={decorate},very thick] {2};
\addplot[name path=B,domain=1:4,samples=1000] {0};
\addplot[pattern={Lines[angle=45,distance=3mm]},pattern color=red] fill between [of=A and B];
\node[above left,align=center] at (4,2) {\(P_\ext=P_0+\dfrac{mg}{S}\)};
\draw[dashed] (4,2) -- (4,1);
\filldraw (4,1) node[right,align=center] {état\\d'équilibre} circle (2pt);
\end{axis}
\end{tikzpicture}
\end{center}

Pour la transformation réversible, on a : \[W=-\int_{V_2}^{V_1}P_\ext\odif{V}=-\int_{V_2}^{V_1}P\odif{V}=\,\begin{tikzpicture}
\filldraw[pattern={Lines[angle=45,distance=3mm]},pattern color=blue] (0,-0.25) -- (0,0.25) -- (1,0.25) -- (1,-0.25) -- (0,-0.25);
\end{tikzpicture}\]

Pour la transformation irréversible, on a : \[W=-\int_{V_3}^{V_1}P_\ext\odif{V}=-P_\ext\paren{V_3-V_1}=\,\begin{tikzpicture}
\filldraw[pattern={Lines[angle=45,distance=3mm]},pattern color=red] (0,-0.25) -- (0,0.25) -- (1,0.25) -- (1,-0.25) -- (0,-0.25);
\end{tikzpicture}\]

On remarque donc que : \[\begin{tikzpicture}
\filldraw[pattern={Lines[angle=45,distance=3mm]},pattern color=blue] (0,-0.25) -- (0,0.25) -- (1,0.25) -- (1,-0.25) -- (0,-0.25);
\end{tikzpicture}\,<\,\begin{tikzpicture}
\filldraw[pattern={Lines[angle=45,distance=3mm]},pattern color=red] (0,-0.25) -- (0,0.25) -- (1,0.25) -- (1,-0.25) -- (0,-0.25);
\end{tikzpicture}\]

\section{Annexes}

\subsection{Quelques remarques sur la différence entre l'énergie totale \(E\) en thermodynamique et l'énergie mécanique \(E_m\) en mécanique}

Si on applique le théorème de l'énergie cinétique à un système \(\paren{\Sigma}\) fermé, on a : \[\adif{E_c}=W_\ext+W_\inte=W_{\ext,\ncons}+W_{\ext,\cons}+W_{\inte,\ncons}+W_{\inte,\cons}\] donc on a : \[\adif{E_c}+\adif{E_p}=W_{\ext,\ncons}+W_{\inte,\ncons}\] puisque les travaux des forces extérieures et intérieures conservatives se mettent sous la forme de l'opposé d'une variation d'énergie potentielle.

Si en plus d'être fermé on suppose \(\paren{\Sigma}\) isolé, on a : \[W_{\ext,\ncons}=0\imp\adif{E_c}+\adif{E_p}=W_{\inte,\ncons}\]

Avec \(E=E_c+E_p=E_m+U\), d'après ce que l'on vient de voir, on a : \[\adif{E_m}+\adif{U}=\adif{E}=W_{\inte,\ncons}\]

Si on veut que \(E\) soit conservative (\ie avoir le premier principe de la thermodynamique), il faut que \(\adif{E}=0\) pour \(\paren{\Sigma}\) isolé et donc : \[W_{\inte,\ncons}=0.\]

\emph{Problème} : ceci implique que toutes les forces intérieures d'interaction microscopiques dérivent d'une énergie potentielle, c'est-à-dire soient conservatives ! N'est-ce alors pas contradictoire avec l'existence en mécanique de forces de frottement non-conservatives permettant d'expliquer la dissipation d'énergie ?

\emph{Réponse} : non, car ces forces de frottement sont des modèles utilisés à l'échelle macroscopique pour expliquer la disparition d'énergie mécanique. Mais la mécanique ne précise pas du tout ce que devient cette énergie mécanique perdue. La thermodynamique, en revanche, montre qu'elle se transforme en énergie interne pour \(\paren{\Sigma}\) : on a \(\adif{E}=0\) donc : \[\adif{E_m}<0\imp\adif{U}>0.\]

\emph{Conclusion} : \begin{itemize}
\item il faut bien faire la distinction entre l'énergie mécanique et l'énergie totale : la première n'est pas conservative tandis que l'autre l'est ;
\item contrairement à la mécanique, la thermodynamique peut être qualifiée de science \guillemets{achevée} en ce sens qu'elle permet de prolonger la mécanique à l'échelle microscopique.
\end{itemize}

\subsection{Interprétation microscopique de l'expression de l'énergie interne}

On peut se demander pourquoi un gaz polyatomique nécessite plus d'énergie pour être chauffé qu'un gaz monoatomique. Il faut pour répondre introduire la notion de degré de liberté.

Pour une molécule ponctuelle, il existe trois degrés de liberté : les trois directions de l'espace suivant lesquelles elle peut se déplacer. Les trois directions sont équiprobables et nécessitent donc chacune autant d'énergie, soit un tiers de l'énergie interne ou \(\dfrac{1}{2}Nk_BT\). Statistiquement, pour chaque particule, un degré de liberté va donc nécessiter : \[\dfrac{1}{3}\dfrac{U}{N}=\dfrac{1}{2}k_BT.\]

Une molécule diatomique possède plus de degrés de liberté, car comme cela a été vu en mécanique des systèmes de deux points matériels, le mouvement de la molécule se décompose en un mouvement de translation du centre d'inertie (trois degrés de liberté) composé à un mouvement de rotation autour de ce centre d'inertie. Si on imagine la molécule rigide, comme une règle, il y a deux possibilités de rotation autour des deux axes de rotation perpendiculaires à la direction de la liaison. Cela ajoute deux degrés de liberté aux trois de translations, soit cinq degrés de liberté en tout, eux aussi équiprobables. Si l'on compte comme précédemment, statistiquement, \(\dfrac{1}{2}k_BT\) par molécule et par degré de liberté, cela fait en tout une énergie \(\dfrac{5}{2}k_BT\) par molécule. On a alors l'énergie interne : \[U=\dfrac{5}{2}Nk_BT=\dfrac{5}{2}nRT.\]

On en déduit alors la capacité thermique molaire à volume constant : \[C_{V_m}=\dfrac{5}{2}R\approx\SI{20.8}{\joule\per\kelvin\per\mole}.\]

\emph{Remarque} : en réalité, pour un gaz quelconque, la capacité thermique molaire à volume constant dépend de la température. Pour un gaz parfait diatomique, \(C_{V_m}\) évolue avec \(T\) de la façon suivante : \begin{itemize}
\item \(C_{V_m}=\dfrac{3}{2}R\) pour \(T\leq\SI{60}{\kelvin}\) ;
\item \(C_{V_m}=\dfrac{5}{2}R\) pour \(\SI{60}{\kelvin}\leq T\leq\SI{7000}{\kelvin}\) ;
\item \(C_{V_m}=\dfrac{7}{2}R\) pour \(T\geq\SI{7000}{\kelvin}\).
\end{itemize}

\begin{tkz}
\begin{axis}[axis lines=left,
xlabel={\(T\)},
ylabel={\(C_{V_m}\)},
xmin=0,xmax=10,
ymin=0,ymax=32,
xtick={1,7},
xticklabels={\(T_\text{rot}\),\(T_\text{vib}\)},
ytick={12.471,20.785,29.099},
yticklabels={\(\dfrac{3}{2}R\),\(\dfrac{5}{2}R\),\(\dfrac{7}{2}R\)},
xlabel style={at={(axis description cs:1,0)},anchor=north west},
ylabel style={at={(axis description cs:0,1)},anchor=south west,rotate=-90}]
\draw[rounded corners,very thick] (0,12.471) -- (1,12.471) -- (1,20.785) -- (7,20.785) -- (7,29.099) -- (10,29.099);
\draw[dashed] (1,0) -- (1,12.471);
\draw[dashed] (7,0) -- (7,20.785);
\draw[dashed] (1,20.785) -- (0,20.785);
\draw[dashed] (7,29.099) -- (0,29.099);
\end{axis}
\end{tkz}

Cela signifie qu'aux très basses températures, l'énergie interne n'est due qu'aux mouvements de translation du centre d'inertie de la molécule. Les mouvements propres de rotation et de vibration n'apparaissent qu'à partir de certaines températures limites appelées \(T_\text{rot}\) et \(T_\text{vib}\). Dans le domaine des températures usuelles de la thermodynamique (\(\qtyrange{200}{1000}{\kelvin}\)), pour les molécules usuelles (H\(_2\), N\(_2\), O\(_2\)), il y a cinq degrés de liberté.

\emph{Conclusion} : l'énergie s'homogénéise entre les différents degrés de liberté du système. On parle d'équipartition de l'énergie. Chaque degré de liberté \guillemets{consomme} \(\dfrac{1}{2}k_BT\) au niveau de la molécule. Plus le nombre d'atomes augmente dans la molécule, plus le nombre de degrés de liberté augmente et plus il faut d'énergie pour pouvoir tous les exciter.

\subsection{Cas d'un mélange de gaz parfaits}

Considérons un récipient de volume \(V\) contenant plusieurs gaz parfaits à la température \(T\), en équilibre thermodynamique.

Le nombre total de molécules de gaz est : \[n=\sum_in_i\]

La fraction molaire du gaz \(i\) est : \[X_i=\dfrac{n_i}{\ds\sum_in_i}=\dfrac{n_i}{n}\]

La pression partielle du gaz \(i\) est : \[p_i=\dfrac{n_iRT}{V}\]

Ceci revient à considérer le gaz \(i\) comme s'il était seul dans le récipient. C'est logique puisqu'il n'y a pas d'interaction entre les molécules. Ainsi, le mélange total se comporte comme un gaz parfait et on a : \[pV=nRT\imp p=\dfrac{\paren{\ds\sum_in_i}RT}{V}\]

On en déduit la loi de Dalton : \[p=\sum_ip_i\]

\subsection{Bac to basics : la température}

En physique, biologie ou chimie, aucune expérience digne de ce nom ne se passe de sa mesure. Partout, elle impose sa loi : état de la matière, fonctionnement du vivant, jusqu'aux propriétés cachées des matériaux. Les physiciens détiennent-ils la clé de son sens profond ?

\subsubsection{Qu'est-ce que la température ?}

Si les mots perdaient leur sens dès lors qu'on les employait à tort et à travers, notre vocable sonnerait bien creux tellement il surgit çà et là au détour des conversations : on le confond souvent avec la chaleur, parfois avec l'énergie, voire avec un tempérament ou un état d'âme.

Pour les physiciens, la température est une grandeur couramment utilisée pour décrire un milieu. Que mesure-t-elle en réalité ? L'agitation moyenne des particules qui composent le corps : celle des atomes pour un corps simple ou des molécules pour un corps composé. Grâce à la température, le physicien se forge une idée sur l'état des particules qui composent l'objet, mais aussi celle des forces qui assurent à l'échelle microscopique la cohésion du corps. En somme, mesurer la température permet d'ausculter l'atome en palpant l'objet. De l'art et la manière d'effectuer un détour au cœur de la matière sans bourse délier, ni microscope à traîner...

Cette vision est celle qui convient le mieux au physicien d'aujourd'hui. Mais pour en arriver là, il a fallu deux siècles de réflexion... Car définir la température comme une mesure de l'agitation thermique moyenne des particules, voilà qui aurait heurté l'honnête homme du siècle passé. Quelles particules ? Quelle agitation thermique ? \textit{\guillemets{La température, c'est ce qui fixe le sens des échanges thermiques}}, aurait-il lancé. En effet, tout au long de la première moitié du \siecle{xix} siècle, les esprits scientifiques les plus futés s'étaient emparés d'un problème aux multiples retombées : comment améliorer les machines thermiques ? En d'autres termes, comment rentabiliser au maximum son tas de charbon ? Honneur à l'initiateur : dans son ouvrage paru en 1824, mais devenu célèbre bien après sa mort, le Français Sadi Carnot évoquait déjà le principe des machines qui allaient porter son nom : la chaleur -- le fluide calorique en termes de l'époque -- circule de la source ayant la plus haute température à celle ayant la température la plus basse. Ajoutez à cela, quelque vingt ans plus tard, les cogitations d'un fils de brasseur, James Prescott Joule : excellant dans l'art de la bière, il démontre que la chaleur est une forme d'énergie qui ne se conserve pas. Mais, pour que le concept s'affine et que les idées sur la nature de la température évoluent, il fallait encore le grain de sel de l'Anglais William Thomson alias lord Kelvin.

\subsubsection{Peut-on parler de la température d'un seul atome ?}

Non. Bien avant la découverte de l'atome, un génie incompris du début du \siecle{xx} siècle, l'Autrichien Ludwig Boltzmann, émit l'hypothèse que les propriétés des objets résultaient des comportements microscopiques de la matière. C'est sur ce trait d'union entre l'infiniment petit et le macroscopique que s'est construite la thermodynamique moderne. Pourtant, la température d'un atome isolé ne signifie alors pas grand-chose, et c'est là tout le paradoxe. Car bien que cette grandeur soit déterminée par l'agitation des particules, elle ne peut s'appliquer qu'à la collectivité : les divagations d'un individu n'intéressent guère le physicien qui s'inquiète plutôt des mouvements des atomes les uns par rapport aux autres. Et sachant que douze malheureux grammes de carbone contiennent \(\num{6.026e23}\) atomes (le nombre d'Avogadro), on comprend la nécessité d'adopter un point de vue collectiviste et de régler le comportement de tous par une seule loi. Or, depuis Boltzmann, entre l'individu et la société il y a la statistique et ses règles d'or égalitaristes, attribuant la même probabilité à toutes les configurations que peuvent prendre les atomes. Du coup, lorsque l'on chauffe un corps, ses constituants en bénéficient en moyenne de manière équitable : dans un corps, à une température donnée, à un instant donné, des frénétiques peuvent côtoyer des flegmatiques et, l'instant d'après, se comporter tous comme un bataillon d'excités... Mais la distribution des rôles reste constamment orchestrée par les lois de Boltzmann. C'est pourquoi la température est une grandeur statistique et qu'elle ne représente que l'agitation thermique moyenne. La température d'un seul atome n'a de sens que si celui-ci est en contact avec une collection d'atomes.

\subsubsection{Deux corps à la même température se comportent-ils de la même manière ?}

Pas vraiment. Petite démonstration : prenons deux pots, l'un en fer et l'autre en terre, ayant la même température de \(\SI{50}{\degreeCelsius}\). Dans une pièce isolée à \(\SI{20}{\degreeCelsius}\), ils vont tous les deux commencer par se refroidir, donc dégager de la chaleur pour atteindre pratiquement \(\SI{20}{\degreeCelsius}\) : \guillemets{pratiquement}, car les deux pots vont à leur tour un peu réchauffer la pièce, même si leur masse est bien trop petite pour que ce réchauffement soit perceptible. Cependant, ils ne vont pas dégager la même quantité de chaleur pour se refroidir. Tout dépend de leur capacité calorifique, qui est propre à chaque matériau.

Certains dégagent beaucoup de chaleur pour atteindre les \(\SI{20}{\degreeCelsius}\), d'autres bien moins. Que se passe-t-il donc dans leur for intérieur ? D'aucuns encaissent mine de rien toute l'agitation de leurs entrailles, tandis que d'autres ne veulent rien savoir de ce gargouillis interne. Impassibles à la bougeotte de leurs atomes. Pourquoi ? Tout dépend du degré de liberté des atomes dudit matériau : plus ils sont capables de se tortiller dans tous les sens, dansant sur tous les modes possibles (exhibant des mouvements de translation, de vibration et de rotation), plus leur capacité calorifique est grande. Résultat : il leur faut dépenser beaucoup de chaleur pour perdre un petit degré. La capacité calorifique du fer et de la terre ne sont pas les mêmes.

Mais au bout d'un quart d'heure le pot de fer refroidi peut narguer le pot en terre qui est encore tout chaud. Là aussi terre et fer diffèrent : le second évacue rapidement ses soubresauts, tandis que le premier peine à transmettre sa chaleur interne. Ses atomes sont rangés de manière plus régulière et transmettent leur agitation de proche en proche. Tout dépend de la conductivité thermique du matériau.

Le premier paramètre agit dans l'instant, le second opère dans le temps. Les choses se compliquent toutefois quand on sait que la conductivité thermique et la capacité calorifique pour un solide dépendent elles-mêmes de la température !

\subsubsection{Comment mesure-t-on la température ?}

Avec un thermomètre, pardi ! Sauf que ces instruments de mesure peuvent prendre différentes formes et figures... Deux types de graduation peuvent porter la mesure la mesure de la température, chacune arborant le nom de leur inventeur. Celle de Celsius, qui remonte vers 1742, est centésimale, -- c'est-à-dire comprend cent graduations entre deux points fixes (0 pour la température de la glace fondante et 100 pour celle de l'eau bouillante à pression atmosphérique).

L'échelle de température thermodynamique date de 1852. Elle est l'œuvre de William Thomson, qui deviendra plus tard Lord Kelvin. L'Écossais, ayant fait un détour sur le continent, a beaucoup usé non seulement ses neurones, mais aussi ses talents d'expérimentateur dans les machines de Carnot, qui avait remarqué que la chaleur échangée ne dépend que du rapport entre la température de la source chaude et celle de la source froide. Et que le rapport entre le travail produit et la chaleur absorbée définit le rendement d'une machine. Thomson s'est mis alors en tête de déterminer la température qui permet le meilleur rendement : pour le cas idéal où le rendement est égal à un, on trouve en extrapolant vers les basses températures que la source froide doit afficher une température de \(\SI{-273.15}{\kelvin}\). L'échelle absolue des températures venait de naître. Le choix de ce zéro absolu (qui fixe le point triple de l'eau à \(\SI{+273.16}{\kelvin}\)) facilitera considérablement les concepts thermodynamiques : il évite les températures négatives et des zéros aux dénominateurs, terreurs des mathématiciens. Mais toute sa portée ne se révélera qu'un peu plus tard...

Quant aux thermomètres, ils possèdent dans l'ensemble le même principe de fonctionnement : ils visent tous un corps, solide, liquide ou gaz, dont le comportement en fonction de la température est bien connu et bien calibré. Si, en outre, celui-ci obéit à une loi simple, alors nous tenons tous les ingrédients indispensables pour fabriquer un thermomètre. Par exemple, un gaz contenu dans un volume défini voit sa pression augmenter avec la température. Mesurer l'augmentation de pression revient alors directement à mesurer la température. Le mercure de nos thermomètres d'antan est un liquide dont la dilatation volumique avec la température est bien calibrée. Il suffit de connaître le volume de mercure pour accéder à la température...

\subsubsection{À quelle température les atomes ne s'agitent-ils donc plus ?}

Aucune. On a beau abaisser autant que possible la température, les atomes restent animés des mouvements de vibration, de rotation et de translation qui les caractérisent. Mais il est possible de les assagir sérieusement : à la température du zéro absolu, soit à \(\SI{-273.15}{\kelvin}\), ils se retrouvent tous dans le même état, l'état fondamental. Pour illustrer cette configuration extrême, on peut prendre l'exemple d'un régiment, chaque soldat représentant un atome. À température élevée, l'agitation thermique moyenne des atomes, autrement dit la température, peut être comparée au taux d'alcoolémie des membres de la patrouille : or, chacun le sait, une armée de soûlards ne marche pas au pas. Outre les traîne-savates qui prennent du retard, il y a ceux qui s'égarent à gauche ou à droite, un peu, beaucoup, au gré de leur humeur. Le commandant qui se retourne fréquemment pour les surveiller peut les surprendre dans toutes les configurations possibles et imaginables : deux à droite, un à gauche, cinq dans le rang, ou bien encore trois à gauche, deux à droite, trois dans le rang... Or, plus ils sont soûls, plus le nombre de configurations qu'ils donnent à voir à leur commandant est élevé. Cela fait pour le moins désordre... De la même manière, plus la température est élevée, plus les atomes s'agitent et sont susceptibles de se placer de manière différente. Le nombre de configurations possibles indique si un système renferme beaucoup de désordre. Or, il apparaît bien que ce désordre est directement proportionnel à la température.

La multitude d'états possibles que peut prendre un système est en fait une notion importante pour les physiciens : c'est l'entropie du système. Or, à la température du zéro absolu, l'entropie est nulle, et le désordre pareillement. L'échelle définie auparavant par Lord Kelvin pour déterminer le rendement maximal d'une machine thermique a donc une portée bien plus profonde. Que se passe-t-il chez les atomes et les soldats ? Disons que le froid dessoûle sérieusement. Les atomes sont contraints à occuper leur niveau d'énergie le plus bas, le même pour tous, l'état fondamental. Résultat : chez les atomes, une seule configuration l'emporte. Au sein du régiment, chacun a recouvré ses esprits et marche au pas. Le commandant a beau se retourner, pas une seule tête ne dépasse : le degré zéro du désordre, sans que les hommes restent figés. L'entropie d'un tel système est nulle, et il règne une température de \(\SI{-273.15}{\kelvin}\). Cette valeur de la température garantit une entropie nulle à tous les atomes quels qu'ils soient : on perçoit alors le sens profond de l'échelle absolue de température définie auparavant par Kelvin.

\begin{center}
Désordre (\(T=\SI{300}{\kelvin}\))

\begin{tikzpicture}[scale=0.8]
\begin{axis}[axis lines=left,
y axis line style={very thick},
x axis line style={draw=none},
ylabel={énergie},
xmin=0,xmax=12,
ymin=0,ymax=6,
xmajorticks=false,
ymajorticks=false,
xlabel style={at={(axis description cs:1,0)},anchor=north west},
ylabel style={at={(axis description cs:0,1)},anchor=south east,rotate=-90}]
\pgfplotsinvokeforeach{1,...,5}{\draw[gray] (0,#1) -- (11,#1);}
\filldraw (1,1) circle (6pt);
\filldraw (2,3) circle (6pt);
\filldraw (3,2) circle (6pt);
\filldraw (4,4) circle (6pt);
\filldraw (5,3) circle (6pt);
\filldraw (6,2) circle (6pt);
\filldraw (7,3) circle (6pt);
\filldraw (8,3) circle (6pt);
\filldraw (9,3) circle (6pt);
\filldraw (10,4) circle (6pt);
\filldraw (11,5) circle (6pt);
\end{axis}
\end{tikzpicture}
\hskip5pt
\begin{tikzpicture}[scale=0.8]
\begin{axis}[axis lines=left,
xlabel={nombre d'atomes},
ylabel={énergie},
xmin=0,xmax=12,
ymin=0,ymax=6,
xmajorticks=false,
ymajorticks=false,
xlabel style={at={(axis description cs:1,0)},anchor=south west},
ylabel style={at={(axis description cs:0,1)},anchor=south east,rotate=-90}]
\addplot[ultra thick,samples=1000] (exp(-(x-3)*(x-3))*8,x);
\draw[dashed] (0,3) -- (12,3);
\end{axis}
\end{tikzpicture}

\begin{tikzpicture}[scale=0.8]
\begin{axis}[axis lines=left,
y axis line style={very thick},
x axis line style={draw=none},
ylabel={énergie},
xmin=0,xmax=12,
ymin=0,ymax=6,
xmajorticks=false,
ymajorticks=false,
xlabel style={at={(axis description cs:1,0)},anchor=north west},
ylabel style={at={(axis description cs:0,1)},anchor=south east,rotate=-90}]
\pgfplotsinvokeforeach{1,...,5}{\draw[gray] (0,#1) -- (11,#1);}
\filldraw (1,3) circle (6pt);
\filldraw (2,2) circle (6pt);
\filldraw (3,5) circle (6pt);
\filldraw (4,4) circle (6pt);
\filldraw (5,3) circle (6pt);
\filldraw (6,3) circle (6pt);
\filldraw (7,3) circle (6pt);
\filldraw (8,4) circle (6pt);
\filldraw (9,2) circle (6pt);
\filldraw (10,3) circle (6pt);
\filldraw (11,3) circle (6pt);
\end{axis}
\end{tikzpicture}
\hskip5pt
\begin{tikzpicture}[scale=0.8]
\begin{axis}[axis lines=left,
xlabel={nombre d'atomes},
ylabel={énergie},
xmin=0,xmax=12,
ymin=0,ymax=6,
xmajorticks=false,
ymajorticks=false,
xlabel style={at={(axis description cs:1,0)},anchor=south west},
ylabel style={at={(axis description cs:0,1)},anchor=south east,rotate=-90}]
\addplot[ultra thick,samples=1000] (exp(-(x-3)*(x-3))*8,x);
\draw[dashed] (0,3) -- (12,3);
\end{axis}
\end{tikzpicture}

\begin{tikzpicture}[scale=0.8]
\begin{axis}[axis lines=left,
y axis line style={very thick},
x axis line style={draw=none},
ylabel={énergie},
xmin=0,xmax=12,
ymin=0,ymax=6,
xmajorticks=false,
ymajorticks=false,
xlabel style={at={(axis description cs:1,0)},anchor=north west},
ylabel style={at={(axis description cs:0,1)},anchor=south east,rotate=-90}]
\pgfplotsinvokeforeach{1,...,5}{\draw[gray] (0,#1) -- (11,#1);}
\filldraw (1,4) circle (6pt);
\filldraw (2,3) circle (6pt);
\filldraw (3,3) circle (6pt);
\filldraw (4,3) circle (6pt);
\filldraw (5,2) circle (6pt);
\filldraw (6,5) circle (6pt);
\filldraw (7,2) circle (6pt);
\filldraw (8,3) circle (6pt);
\filldraw (9,4) circle (6pt);
\filldraw (10,3) circle (6pt);
\filldraw (11,1) circle (6pt);
\end{axis}
\end{tikzpicture}
\hskip5pt
\begin{tikzpicture}[scale=0.8]
\begin{axis}[axis lines=left,
xlabel={nombre d'atomes},
ylabel={énergie},
xmin=0,xmax=12,
ymin=0,ymax=6,
xmajorticks=false,
ymajorticks=false,
xlabel style={at={(axis description cs:1,0)},anchor=south west},
ylabel style={at={(axis description cs:0,1)},anchor=south east,rotate=-90}]
\addplot[ultra thick,samples=1000] (exp(-(x-3)*(x-3))*8,x);
\draw[dashed] (0,3) -- (12,3);
\end{axis}
\end{tikzpicture}

\begin{tikzpicture}
\begin{axis}[axis lines=left,
title={État fondamental (\(T=\SI{0}{\kelvin}\))},
y axis line style={very thick},
x axis line style={draw=none},
ylabel={énergie},
xmin=0,xmax=12,
ymin=0,ymax=6,
xmajorticks=false,
ymajorticks=false,
xlabel style={at={(axis description cs:1,0)},anchor=north west},
ylabel style={at={(axis description cs:0,1)},anchor=south east,rotate=-90}]
\pgfplotsinvokeforeach{1,...,5}{\draw[gray] (0,#1) -- (11,#1);}
\filldraw (1,1) circle (6pt);
\filldraw (2,1) circle (6pt);
\filldraw (3,1) circle (6pt);
\filldraw (4,1) circle (6pt);
\filldraw (5,1) circle (6pt);
\filldraw (6,1) circle (6pt);
\filldraw (7,1) circle (6pt);
\filldraw (8,1) circle (6pt);
\filldraw (9,1) circle (6pt);
\filldraw (10,1) circle (6pt);
\filldraw (11,1) circle (6pt);
\end{axis}
\end{tikzpicture}
\end{center}

À la température du zéro absolu, tous les atomes sont dans leur état de plus basse énergie. Dès que la température s'élève, les atomes s'écartent de cet état fondamental : à un température donnée, leur agitation thermique moyenne peut résulter d'un grand nombre de configurations différentes à l'échelle atomique (trois sont représentées ici).

\subsubsection{Quelle est l'étendue des valeurs que peut prendre la température ?}

Les \(\SI{300}{\kelvin}\) qui règnent sur Terre laissent l'eau s'écouler en rivière, jaillir en cascades. C'est la condition préalable à l'émergence de la vie sur la planète. Cette température clémente règne-t-elle ailleurs dans l'Univers ? Nul ne le sait aujourd'hui, mais les plus extrêmes s'y côtoient : le gaz qui est en passe d'être englouti par les astres les plus denses, trous noirs ou étoiles à neutrons, affiche une température de plus d'un million de degrés. Ainsi chauffée, la matière se transforme en plasma : les électrons se libèrent du giron de l'atome, et le noyau part en solitaire. Un véritable océan d'électrons et de noyaux atomiques à la dérive. Or, cet état de la matière est extrêmement courant dans l'Univers : ce gaz chaud émet des rayons très énergétiques, des X et des gamma, que l'on peut capter grâce aux satellites d'observations astronomiques.

Mais les températures à proximité d'un trou noir sont sans commune mesure avec celles qui font le quotidien des théoriciens. Les cosmologistes qui tentent d'expliquer à coup d'équations les premiers instants de l'Univers ont estimé la température qui colle à leurs modèles : ainsi après un milliardième de seconde après le Big Bang, aurait régné une température de \(\SI{e13}{\kelvin}\), indispensable pour la formation des premières particules stables, les protons et les neutrons, éléments des futurs noyaux atomiques...

Côté basses températures, la nature se fait largement doubler par les expériences de laboratoire : des solides entiers ont pu être refroidis jusqu'à quelques millionièmes de Kelvin, tandis que les atomes froids frisent quelques milliardièmes de Kelvin. Qu'est-ce qui distingue un atome froid de son cousin réchauffé ? L'atome glacé reste comme suspendu dans le vide, sans cette agitation thermique qui distingue son état habituel lorsqu'il est enfoui au sein d'un objet à température ambiante. Pour le tenir ainsi, on braque sur sa personne une multitude de lasers d'égale intensité. Quant à l'intérêt de l'opération... ce n'est pas de voir son nom gravé dans le livre des records, mais de manipuler les atomes quasiment comme des prunes : les déplacer un par un et les compter. Dans quel but ? Celui de vérifier les lois de la physique quantique, et, à plus long terme, de fabriquer des faisceaux laser où les grains de lumière seront remplacés par les atomes.

\subsubsection{Quelle gamme de températures convient à la vie ?}

Aux températures élevées, la matière a la faculté -- en fournissant de l'énergie interne sous forme de chaleur -- d'arracher un électron à un atome et ainsi de l'ioniser. Or, il s'agit là du principe même de la réactivité chimique. La température impose la vitesse des réactions chimiques : oxydation, digestion, putréfaction, tout ce qui relève de la chimie lui doit une fière chandelle. Sans oublier la vie qui englobe une multitude de réactions : le métabolisme du vivant n'est possible que dans une gamme restreinte de températures. Si les \(\SI{37}{\degreeCelsius}\) du corps humain optimisent la vitesse des réactions chimiques, c'est en grande partie à cause de l'existence de l'eau sous sa forme liquide. C'est elle qui se charge, par le biais des fluides biologiques, de transporter et de distribuer les substances nécessaires aux différents organes. La vie est-elle possible sans eau liquide ? Jamais vu, jamais envisagé, répondent en chœur les exobiologistes. Mais des cas d'adaptation aux températures extrêmes sont suivis de très près. Côté forte température, le record du vivant dépasse la centaine de degrés : à proximité des sources hydrothermales des grands fonds, dans une eau à plus de \(\SI{100}{\degreeCelsius}\) où pullulent des colonies de micro-organismes...

À l'autre bout de l'échelle, les psychrophiles sont des organismes unicellulaires qui se contentent d'un petit \(\SI{20}{\degreeCelsius}\), voire moins. Certains restent en vie à des températures de \(\SI{-12}{\degreeCelsius}\). Non pas qu'ils se passent d'eau liquide, vivant au ralenti, mais parce qu'ils ont la faculté de conserver l'eau sous forme liquide à des températures inférieures au point de congélation. Leur secret ? Le même que des milliards d'automobilistes pris par le froid hivernal : la cellule psychrophile produit des molécules antigel à base de sucre et d'alcool.

L'antigel est aussi la substance fétiche de certaines personnes qui, séduites par la cryogénisation, ont émis le désir de se réveiller un jour dans un monde meilleur : après leur mort, les fluides vitaux de leurs corps ont été remplacés par... du glycérol... qui est censé préserver leur dépouille dans un état de conservation telle que, des années plus tard, les progrès de la science aidant, ils puissent revenir à la vie. L'escroquerie a rencontré quelque succès outre-Atlantique. De leur côté, les cryobiologistes ont déjà réussi la conservation de cellules : les banques de sang et de sperme utilisent des méthodes de conservation par le froid. Mais leur objectif pour les prochaines décennies est de concevoir une banque d'organes : des reins, foies ou poumons à prélever, à conserver et au besoin à transplanter, parfois bien longtemps après. La difficulté dans ce domaine est de refroidir très rapidement -- vitrifier, car la même technique est utilisée pour produire des verres --, afin que les cristaux de glace ne puissent pas se former. En effet, l'eau en gelant voit son volume augmenter et ferait éclater les membranes. De plus, la glace en se constituant abandonne ses sels minéraux. L'antigel idéal se laisse encore désirer.

\subsubsection{Pourquoi tenter d'atteindre de très basses températures ?}

Pour démasquer la vraie nature de la matière : la température, à cause de l'agitation qu'elle provoque chez les atomes, cache quelques phénomènes fondamentaux dont la compréhension éclairerait tout un pan de la physique. La supraconductivité, par exemple, en fait partie : il s'agit de la faculté de transporter du courant sans perte. En dessous d'une certaine température critique, la résistivité électrique du matériau devient pratiquement nulle, il devient alors supraconducteur. Seulement les matériaux disponibles ont des températures critiques si basses qu'aujourd'hui le gain d'énergie ne rembourserait pas le coût du refroidissement. Du coup, l'utilisation des supraconducteurs est pour l'instant limitée aux instruments de mesure des très faibles champs magnétiques, aux outils de diagnostic médical fonctionnant sur la base de la résonance magnétique nucléaire, ou encore au sein de gros accélérateurs de particules.

Une autre propriété de la matière est à mettre en parallèle avec la supraconductivité : elle se manifeste chez l'hélium 4 liquide. Placé dans un récipient, celui-ci grimpe aux parois pour se déverser spontanément. En effet, en dessous d'une certaine température critique, l'hélium 4 perd toute viscosité : il se transforme en un fluide parfait qui n'oppose aucune adhérence aux parois. Ces deux bizarreries qui se manifestent à basse température sont encore dans l'attente d'une théorie qui rende compte de tous leurs aspects.

\chapter{Second principe de la thermodynamique}

\minitoc

\section{Phénomènes irréversibles}

\subsection{Limites du premier principe, nécessité d'un deuxième principe}

Nous avons introduit, dans l'énoncé du premier principe, les notions de transfert thermique et de travail. Nous avons pu constater que ces deux notions traduisent des transferts d'énergie et qu'a priori rien ne différencie ces deux notions, qui ont d'ailleurs la même unité.

Le premier principe pour un cycle de transformations thermodynamiques s'écrit : \[\adif{U}=W+Q=0\imp Q=-W.\] \Cad qu'il semble qu'il y ait équivalence entre \(Q\) et \(W\). En particulier, le premier principe peut laisser penser que le moteur (\(W<0\)) en contact avec une seule source de chaleur (\(Q>0\)) peut exister et qu'il a un rendement égal à \(1\) puisque \(\abs{W}=\abs{Q}\). Cependant, l'expérience prouve que ce moteur n'existe pas (\cf \hyperref[chap:machinesThermiques]{chapitre sur les machines thermiques}).

D'autre part, rien n'interdit dans le premier principe d'inverser le sens d'une machine thermique, \cad de pouvoir lui faire décrire un cycle dans le sens horaire ou trigonométrique : autrement dit, d'après le premier principe, il serait par exemple possible d'utiliser un réfrigérateur pour faire avancer une voiture !

Cet exemple du fonctionnement d'une machine thermique est tout à fait représentatif du rôle du premier principe : il nous permet de faire des bilans d'énergie au cours de diverses transformations d'un système, mais il ne nous renseigne pas du tout sur le sens naturel de ces transformations. Ces transformations sont dites irréversibles et l'expérience prouve l'existence d'un sens unique d'évolution et donc l'impossibilité des transformations inverses.

Ainsi, le premier principe est insuffisant pour expliquer ces phénomènes et étudier la notion d'irréversibilité.

\subsection{Les principales causes d'irréversibilité}

\subsubsection{Sens naturel ou spontané des transformations}

Certains phénomènes naturels sont irréversibles car ils tendent à réuniformiser une distribution non homogène (concentration de matière, température, concentration de charges, etc...).

\underline{Non homogénéité de concentration : transfert de particules.}

Un flacon de parfum ouvert à l'air libre laisse s'évader une odeur se répandant dans tout l'espace environnant : des molécules odorantes se sont déplacées de la solution (zone de forte concentration) vers l'air extérieur (zone de faible concentration). Si l'on considère le système global \guillemets{parfum + air extérieur}, la transformation spontanée tend à l'uniformiser.

\underline{Non homogénéité de température : transfert thermique.}

Une casserole d'eau chaude laissée à l'air libre se refroidit : de l'énergie thermique se perd par rayonnement, par conduction (à travers le support matériel), et par convection (l'air chaud au contact de la casserole a tendance à se déplacer vers des régions plus froides). Pour le système global \guillemets{casserole d'eau chaude + air extérieur}, le transfert thermique se fait spontanément du corps chaud vers le corps froid, et tend à réuniformiser la température.

Dans les deux cas présentés, l'évolution spontanée inverse ramenant le système à son état initial n'est jamais observée. En conclusion, ces transformations constituent des processus irréversibles. On peut remarquer que cela revient à faire intervenir le temps comme nouveau paramètre : l'évolution d'un système isolé est associée au sens d'écoulement du temps, encore dit flèche du temps.

\subsubsection{Frottements solides et frottements fluides}

Les forces de frottement sont des causes d'irréversibilité car elles sont toujours résistives donc dissipatives.

Le frottement fluide se définit en mécanique par une force proportionnelle et opposée à la vitesse, donc il intervient quel que soit le sens du mouvement mais on peut le rendre quasi négligeable si l'on effectue un déplacement infiniment lent.

Par contre, le frottement solide est proportionnel au déplacement et indépendant de la vitesse. Par conséquent, il est toujours créateur d'une force résistive quels que soient le sens et la vitesse du mouvement. Pour le minimiser, on peut lubrifier les surfaces en regard, mais on ne peut l'annuler.

Remarque : l'effet Joule est lui aussi toujours dissipatif. L'énergie est toujours dissipée, quel que soit le sens du courant. On peut l'interpréter par une force de frottement fluide, traduisant le freinage ressenti par les porteurs de charge.

\subsection{Réversibilité, irréversibilité}

Les transformations réversibles ont déjà été évoquées dans le \hyperref[chap:premierPrincipe]{chapitre sur le premier principe}, mais c'est grâce au deuxième principe que cette notion peut être développée.

Une transformation est dite réversible si une modification infiniment petite des paramètres extérieurs permet d'en inverser le sens.

Une transformation réelle est en général non-réversible, mais on peut parfois réaliser des transformations réelles dont les états successifs sont très voisins d'une transformation réversible qui apparaît comme une transformation idéale limite.

Mais pourquoi s'intéresse-t-on à des transformations réversibles ? Tout d'abord, pour une machine fonctionnant avec de telles transformations qui sont idéales, les pertes sont minimisées et le rendement augmenté. D'autre part, on a vu dans le \hyperref[chap:premierPrincipe]{chapitre sur le premier principe} qu'il est souvent plus aisé de calculer les échanges d'énergie lorsque la transformation est réversible.

Ainsi, il arrive bien souvent que l'on fasse l'hypothèse qu'une machine est réversible sachant qu'alors, le rendement calculé est maximal (\cf \hyperref[chap:machinesThermiques]{chapitre sur les machines thermiques}). Le rendement réel est toujours inférieur au rendement calculé dans le cas réversible.

\section{Second principe de la thermodynamique}

\subsection{Énoncé}

Rappel : le premier principe \(\adif{U}=W+Q\) a permis de faire des bilans d'énergie. C'est un principe de conservation.

Problème : il ne nous dit pas si une transformation est possible ou non (sens d'évolution).

On introduit donc une nouvelle grandeur non-conservative, l'entropie, notée \(S\) et telle que : \[\adif{S}=S_e+S_c\] de manière à avoir \(\adif{S}=S_c\) si le système est isolé.

Dans le cas d'une transformation réversible, on a \(S_c=0\). Sinon, on a \(S_c>0\).

Énoncé : pour tout système thermodynamique fermé, on peut définir une fonction d'état extensive notée \(S\) et appelée entropie, et telle que : \begin{itemize}
\item au cours d'une transformation adiabatique, \(S\) ne peut qu'augmenter (\(\adif{S}\geq0\)) ;
\item au cours d'une transformation quelconque, on a la variation d'entropie \(\adif{S}=S_e+S_c\) où \(S_e\) est l'entropie échangée et \(S_c\geq0\) est l'entropie créée. \\
\end{itemize}

Ce principe est un principe d'évolution.

\subsection{Conséquences immédiates}

Le travail \(W\) n'a aucune conséquence immédiate sur l'entropie.

Remarque : de même qu'on a, pour le premier principe : \[\adif{U}=W+Q\ssi\odif{U}=\fdif{W}+\fdif{Q}\] on a, pour le second principe : \[\adif{S}=S_e+S_c\ssi\odif{S}=\fdif{S_e}+\fdif{S_c}.\]

\(S\) est une fonction d'état et \(\adif{S}\) ne dépend pas du chemin suivi, là où \(S_e\) et \(S_c\) ne sont pas des fonctions d'état et dépendent donc du chemin suivi.

Dans le cas d'une évolution quelconque, on a \(S_e\supinf0\) et \(S_c\geq0\) donc on a : \[\adif{S}\supinf0.\]

Si on considère une transformation d'un état 1 vers un état 2 réversible et adiabatique et ensuite une transformation de l'état 2 vers l'état 1 réversible et adiabatique, on a : \[\adif{S}=S_1-S_2=S_c=0\quad\text{et}\quad\adif{S}=S_2-S_1=S_c=0\] donc \(\adif{S}=0\). On parle de transformation isentropique.

Au cours d'une transformation adiabatique et non-réversible, on a : \[\adif{S}=S_c>0.\] Donc l'entropie ne peut qu'augmenter et l'état d'équilibre correspond à l'entropie maximale \(S_\maxi\).

\subsection{Température et pression thermodynamiques}

\subsubsection{Première identité thermodynamique}

\(S\) est une fonction d'état donc dépend des paramètres d'état \(\paren{P,V,T,\dots}\).

Donc l'énergie interne \(U\) peut-être décrite à partir de \(\paren{T,V}\) mais aussi de \(\paren{S,V}\).

On aurait donc : \[\odif{U}=\pdv{U}{S}_V\odif{S}+\pdv{U}{V}_S\odif{V}.\]

On définit la température et la pression thermodynamiques : \[P=-\pdv{U}{V}_S\quad\text{et}\quad T=\pdv{U}{S}_V\]

On obtient la première identité thermodynamique : \[\odif{U}=-P\odif{V}+T\odif{S}.\]

On repart du premier principe : \[\begin{aligned}
\adif{U}&=W+Q \\
\odif{U}&=\fdif{W}+\fdif{Q} \\
\odif{U}&=-P_\ext\odif{V}+\fdif{Q}
\end{aligned}\]

Lors d'une transformation réversible, on a \(P=P_\ext\) et on obtient \(\fdif{Q}=T\odif{S}\) donc : \[\odif{S}=\dfrac{\fdif{Q}}{T}.\] On obtient aussi \(S:\unit{\joule\per\kelvin}\).

\subsubsection{Deuxième identité thermodynamique}

On a : \[\begin{WithArrows}
H&=U+PV \Arrow{\(\dif\)} \\
\odif{H}&=\odif{U}+V\odif{P}+P\odif{V} \\
&=-P\odif{V}+T\odif{S}+V\odif{P}+P\odif{V} \\
&=V\odif{P}+T\odif{S}
\end{WithArrows}\] C'est la deuxième identité thermodynamique.

\subsection{Cas d'une transformation réversible}

On a : \[\begin{WithArrows}
\odif{S}&=\dfrac{\fdif{Q}}{T} \Arrow{\(\int\)} \\
\adif{S}&=\int\dfrac{\fdif{Q}}{T}
\end{WithArrows}\]

Donc un transfert thermique \(Q\), même effectué réversiblement, peut faire varier l'entropie.

\subsection{Transformation monotherme}

\subsubsection{Notion de thermostat}

Un thermostat est un système thermodynamique ne pouvant échanger de l'énergie que sous forme de transfert thermique.

\begin{tkz}
\draw (0,0) node {Thermostat} circle (3);
\node at (0,-1) {\(T_\ther\)};
\draw[decoration={markings,mark=at position 0.5 with {\arrow{<}}},postaction={decorate}] (3,0) -- (6,0);
\draw[decoration={markings,mark=at position 0.5 with {\arrow{<}}},postaction={decorate}] (-3,0) -- (-6,0);
\node[above right] at (4.5,0) {\(Q\)};
\node[above] at (-4.5,0) {\(W=0\)};
\end{tkz}

De plus, on a \(T_\ther=\cte\).

\note{À FINIR}

\chapter{Changements d'état du corps pur}

\minitoc

\note{À VENIR}

\chapter{Machines thermiques}

\minitoc

\note{À VENIR}

\part{Électricité}

\chapter{Introduction à l'électrocinétique}

\minitoc

\section*{Introduction}

L'électrocinétique est la science qui permet d'étudier les circuits et les grandeurs électriques (tension, courant, charge...).

On distingue le régime continu (\(f=\SI{0}{\hertz}\)) et le régime variable (\(f\) va de \(0\) à quelques \(\unit{\giga\hertz}\)).

\section{Les porteurs de charge électrique}

Par définition, un courant électrique est un déplacement d'ensemble ordonné de particules chargées. Évidemment, la nature de ces particules dépend du milieu dans lequel elles se trouvent.

\subsection{Nature et déplacement des porteurs de charge}

\subsubsection{Conduction dans les solides}

Les conducteurs solides sont les plus usuels.

Les atomes sont constitués d'électrons (de charge \(-e\)), de protons (de charge \(+e\)) et de neutrons (de charge nulle).

\(e\) est la charge élémentaire : \[e=\SI{1.602e-19}{\coulomb}.\]

Globalement, la matière est neutre, mais dans certains solides, il peut y avoir une modification de la répartition des charges dans l'espace : c'est ce qui fait la différence entre les solides conducteurs (appelés aussi métaux) et les isolants.

Dans les métaux, chaque atome peut libérer facilement un ou deux électrons (en moyenne) qui deviennent alors des électrons libres. Un métal est donc un réseau d'ions de charge strictement positive au travers duquel peuvent se déplacer des électrons libres (on parle de \guillemets{nuage} d'électrons libres).

Conducteur au repos : les cations effectuent des oscillations de faible amplitude autour de leur position d'équilibre, alors que les électrons ont un mouvement complètement aléatoire d'agitation thermique (\(v\approx\SI{e6}{\metre\per\second}\)). Globalement, il n'y a donc aucun déplacement de particules.

Déplacement d'ensemble : si l'on crée une différence de potentiel (donc un champ électrique \(\vec{E}\)) entre deux extrémités d'un conducteur, les électrons libres sont entraînés (\(\vec{F}=q\vec{E}=-e\vec{E}\)) alors que les cations restent fixes (ils sont piégés dans le réseau cristallin). On a donc un mouvement d'ensemble de conduction électrique : c'est le courant électrique.

Un solide isolant est donc un solide qui contient très peu d'électrons libres, voire aucun, sauf à haute température.

Les solides semi-conducteurs ont une nature intermédiaire entre les conducteurs et les isolants. À température usuelle, le nombre d'électrons libres est faible, mais celui-ci peut augmenter très rapidement lorsque la température augmente ou sous l'effet d'une différence de potentiel. C'est donc parce que leur conductivité électrique est \guillemets{variable} et \guillemets{commandable} que les semi-conducteurs comme le silicium et le germanium sont très largement utilisés dans les composants électroniques usuels (diodes, transistors, A.O., etc...).

\subsubsection{Conduction dans les liquides}

Les liquides conducteurs sont appelés des électrolytes. Ils contiennent des ions dont la migration assure la conduite électrique.

Cette propriété est notamment utilisée en conductimétrie pour déterminer les évolutions des concentrations des différentes espèces ioniques dans une solution.

\subsubsection{Conduction dans les gaz}

Dans les conditions usuelles, les gaz ne sont pas conducteurs. Cependant, ils peuvent s'ioniser et devenir très conducteurs s'ils sont portés à de très hautes températures (plasma) ou s'ils sont soumis à des champs électriques très intenses (foudre).

Le champ disruptif de l'air est \(E\approx\SI{e6}{\volt\per\meter}\).

Ceci signifie que si dans l'air on approche deux conducteurs à \(\SI{1}{\micro\meter}\) l'un de l'autre en maintenant entre eux une différence de potentiel de \(\SI{1}{\volt}\), il y aura court-circuit par étincelle.

\subsection{Les différents types de courants}

\subsubsection{Courants particulaires}

Déplacement de particules chargées dans le vide.

Exemple : faisceau électronique dans un tube cathodique.

\subsubsection{Courants de conduction}

Déplacement de porteurs de charge dans un milieu matériel immobile.

Exemples : électrons dans les métaux, ions dans les électrolytes, etc...

\subsubsection{Courants de convection}

Déplacement de charges provoqué par le mouvement du support matériel chargé dans le référentiel d'étude, les charges étant fixes par rapport au support mobile.

\section{Le courant électrique}

\subsection{Intensité et sens conventionnel du courant}

Le courant électrique résulte du déplacement d'ensemble des porteurs de charge dans un milieu matériel. Par définition, l'intensité \(i\) du courant électrique à travers une surface \(S\) est la quantité de charges qui traversent par unité de temps. On a : \[i=\odv{q}{t}\] avec \begin{description}
\item \(i\) l'intensité du courant qui traverse la surface \(S\) (en \(\unit{\ampere}\))
\item \(q\) la charge électrique qui traverse la surface \(S\) (en \(\unit{\coulomb}\))
\item \(t\) le temps (en \(\unit{\second}\)). \\
\end{description}

On dit que \(i\) est le flux de charge.

De plus, \(i=\odv{q}{t}\) est une relation algébrique donc \(i\supinf0\), selon le sens de déplacement des charges et selon la nature des porteurs de charge.

\subsection{Conservation de la charge et loi des nœuds}

La charge \(q\) ne peut être ni créée ni détruite. C'est une grandeur conservative : elle se conserve si le système est isolé. La quantité de charges qui arrive à un nœud de jonction est donc égale à celle qui en repart.

\begin{circuit}[scale=1.5]
\draw (-1,1) to[short,-*,i_=\(i_1\)] (0,0);
\draw (1,1) to[short,-*,i_=\(i_2\)] (0,0);
\draw (1,0) to[short,-*,i=\(i_3\)] (0,0);
\draw (0,0) to[short,*-,i_=\(i_4\)] (0,-1);
\end{circuit}

On a : \[i_1+i_2+i_3=i_4\quad\text{ou}\quad\sum_k\epsilon_ki_k=0\] avec \begin{description}
\item \(i_k\) le courant dans la branche \(k\)
\item \(\epsilon_k=\begin{dcases}1&\text{si le courant est dirigé vers le nœud} \\ -1 &\text{si le courant est dirigé à partir du nœud}\end{dcases}\) \\
\end{description}

C'est la loi des nœuds (ou première loi de Kirchhoff).

\subsection{Densité volumique de charge}

On définit la quantité de charges électriques par unité de volume (en \(\unit{\coulomb\per\meter\cubed}\)) : \[\rho_V=\odv{Q}{\tau}\] avec \begin{description}
\item \(\odif{Q}\) : charge (en \(\unit{\coulomb}\))
\item \(\odif{\tau}\) : volume (en \(\unit{\metre\cubed}\)) \\
\end{description}

Dans un échantillon élémentaire de volume \(\odif{\tau}\), on compte la charge \(\odif{Q}\).

Soit \(\odif{N}\) le nombre de porteurs de charge dans \(\odif{\tau}\). S'il s'agit d'électrons, chacun porte la charge \(-e\).

Donc : \[\odif{Q}=-e\odif{N}.\]

On pose \(n=\odv{N}{\tau}\) la densité volumique de porteurs de charge.

On a donc : \[\begin{aligned}
\odif{Q}&=-e\odif{N} \\
\odv{Q}{\tau}&=-e\odv{N}{\tau} \\
\rho_V&=-en
\end{aligned}\]

Dans un métal, chaque atome libère un ou deux électrons en moyenne. Donc : \[n=\SI{e29}{\per\metre\cubed}.\]

Donc : \[\rho_V=-en=\numproduct{1.6e-19x10e29}\approx\SI{e10}{\coulomb\per\meter\cubed}.\]

\subsection{Approximation des régimes quasi-stationnaires (ARQS)}

Régime continu : toutes les grandeurs électriques sont des constantes du temps.

Régime variable : les grandeurs électriques sont des fonctions périodiques à la fréquence \(f\).

Si les variations des grandeurs électriques sont lentes, les lois de l'électrocinétique en régime variable sont les mêmes qu'en régime continu.

En réalité, les grandeurs électriques sont des ondes électromagnétiques qui se propagent à la célérité \(c\approx\SI{3e8}{\metre\per\second}\).

Dans un conducteur de longueur \(l\), le temps de propagation est donc : \[\tau=\dfrac{l}{c}.\]

Pour être dans l'approximation des régimes quasi-stationnaires, il faut vérifier que \(\tau\) est très inférieur à \(T=\dfrac{1}{f}\) avec \(f\) la fréquence du générateur, le temps de variation des sources (\ie le générateur). Autrement dit : \[\begin{aligned}
\text{ARQS}&\iff\tau\ll T \\
&\iff\dfrac{l}{c}\ll\dfrac{1}{f} \\
&\iff l\ll\dfrac{c}{f}.
\end{aligned}\]

Avec le réseau domestique (\(\SI{230}{\volt}\) et \(\SI{50}{\hertz}\)) : \[\text{ARQS}\iff l\ll\dfrac{c}{f}=\dfrac{\num{3e8}}{50}=\SI{6000}{\kilo\meter}.\]

Ainsi, à l'échelle domestique, on est toujours dans l'ARQS donc les règles de l'électrocinétique s'appliquent. On ne tient pas compte des phénomènes de propagation.

En TP on a des circuits de longueur \(l=\SI{1}{\metre}\) donc on est dans l'ARQS si \(f\ll\SI{300}{\mega\hertz}\). On utilise donc des générateurs basse fréquence.

\subsection{La tension électrique}

On définit le potentiel électrique \(V\) en tout point du circuit.

Prenons par exemple un dipôle : \begin{circuitikz}
\draw (0,0) to[short,-*] ++(0.5,0) -- ++(0.5,0) to[generic] ++(1,0) -- ++(0.5,0) to[short,*-] ++(0.5,0);
\node[above] at (0.5,0) {\(A\)};
\node[above] at (2.5,0) {\(B\)};
\end{circuitikz}

La tension électrique est la différence de potentiel entre \(A\) et \(B\) : \[U_{AB}=V_A-V_B.\]

C'est la tension non-nulle qui permet aux charges électriques de s'écouler et donne naissance au courant électrique.

Analogie avec la mécanique :

\begin{tkz}[scale=0.8]
\draw[gray,->] (-1,0) -- (6,0); % axe
\draw[gray,->] (0,-1) -- (0,6) node[above left] {\(z\)}; % axe
\draw[thick] (1,5) node[above left] {\(A\)} -- (5,1) node[below right] {\(B\)}; % chemin
\draw (3.37,3.37) circle (0.5); % bille
\draw[->] (3.37,3.37) -- (4.37,2.37); % mouvement
\draw[->,violet] (5.5,5.5) -- (5.5,4.5) node[right] {\(\vec{g}\)}; % g
\node[gray,below left] at (0,0) {\(z_0\)};
\draw[gray,dashed] (0,1) node[left] {\(z_B\)} -- (5,1);
\draw[gray,dashed] (0,5) node[left] {\(z_A\)} -- (1,5);
\end{tkz}

La bille descend car \(z_A>z_B\).

De plus, de même que pour l'altitude, on définit une référence de potentiel, c'est-à-dire le \guillemets{potentiel 0}.

Cette référence est la masse, que l'on représente ainsi : \begin{circuitikz}
\draw (0,0) node[eground]{};
\end{circuitikz}

Exemple de circuit :

\begin{circuit}
\draw (0,0) to[short,-*,i=\(i\)] ++(3,0) -- ++(1,0) to[R,l_=\(R\),v^=\(u_R\)] ++(0,-3) to[short,-*] ++(-1,0) -- ++(-1,0) node[eground] {} -- ++(-2,0) to[vsource,v=\(E\)] (0,0);
\node[above] at (3,0) {\(A\)};
\node[below] at (3,-3) {\(M\)};
\node at (2,-4) {\(V=\SI{0}{\volt}\)};
\end{circuit}

\(u_R=V_A-V_M=V_A\) est la tension aux bornes de la résistance \begin{circuitikz}
\draw (0,0) to[R,l=\(R\)] (3,0);
\end{circuitikz}

\(E=V_A-V_M=V_A\) est la tension aux bornes du générateur \begin{circuitikz}
\draw (0,0) to[vsource] (3,0);
\end{circuitikz}

On considère le circuit suivant :

\begin{circuit}
\draw (0,0) to[twoport,t=\(D_2\),v^=\(U_2\),*-*] ++(3,0) to[twoport,t=\(D_3\),v^=\(U_3\)] ++(0,-3) to[short,-*] ++(-1.5,0) -- ++(-1.5,0) to[twoport,t=\(D_1\),v^>=\(U_1\)] (0,0);
\node[above left] at (0,0) {\(A\)};
\node[above right] at (3,0) {\(B\)};
\node[below] at (1.5,-3) {\(C\)};
\node[scale=1.5] at (1.5,-1.5) {\(\circlearrowleft\)};
\end{circuit}

On choisit un sens arbitraire et on parcourt le circuit dans ce sens en écrivant les tensions : \[-U_1+U_3+U_2=V_A-V_A=0.\]

On a : \[U_3+U_2=U_1\quad\text{ou}\quad\sum_k\epsilon_kU_k=0\] avec \begin{description}
\item \(U_k\) la tension aux bornes du dipôle \(k\)
\item \(\epsilon_k=\begin{dcases}1&\text{si la flèche de tension est dans le sens arbitraire} \\ -1&\text{sinon}\end{dcases}\) \\
\end{description}

C'est la loi des mailles (ou deuxième loi de Kirchhoff).

\section{Dipôle électrocinétique}

\subsection{Définition}

Un dipôle électrocinétique est un dispositif relié au circuit par deux bornes :

\begin{circuit}
\draw (0,0) to[short,*-,i=\(i_\text{e}\)] ++(1,0) to[twoport,t=\(D\),v=\(u\)] ++(1,0) to[short,-*,i=\(i_\text{s}\)] ++(1,0);
\node[above] at (0,0) {\(A\)};
\node[above] at (3,0) {\(B\)};
\end{circuit}

Dans l'ARQS, on a : \[i_\text{e}=i_\text{s}=i\].

On a : \[u=V_A-V_B\] la différence de potentiel aux bornes du dipôle.

On effectue un premier pour le sens du courant électrique :

\begin{center}
\begin{circuitikz}
\draw (0,0) to[short,i=\(i\)] ++(1,0) to[twoport,t=\(D\)] ++(1,0) -- ++(1,0);
\end{circuitikz}
\qquad
\begin{circuitikz}
\draw (0,0) to[short,i<=\(i\)] ++(1,0) to[twoport,t=\(D\)] ++(1,0) -- ++(1,0);
\end{circuitikz}
\end{center}

Une fois ce choix effectué, il faut choisir le sens de la flèche de tension :

\begin{center}
\begin{circuitikz}
\draw (0,0) to[short,i=\(i\)] ++(1,0) to[twoport,t=\(D\),v=\(u\)] ++(1,0) -- ++(1,0);
\node[below] at (1.5,-1) {convention récepteur};
\end{circuitikz}
\qquad
\begin{circuitikz}
\draw (0,0) to[short,i=\(i\)] ++(1,0) to[twoport,t=\(D\),v>=\(u\)] ++(1,0) -- ++(1,0);
\node[below] at (1.5,-1) {convention générateur};
\end{circuitikz}
\end{center}

Remarque : \(u\) et \(i\) sont des grandeurs algébriques donc on a : \[i\supinf0\quad\text{et}\quad u\supinf0.\]

Donc les flèches de tension et de courant ne traduisent pas forcément le sens réel.

\subsection{Puissance reçue par un dipôle}

On considère le dipôle suivant :

\begin{circuit}
\draw (0,0) to[twoport,t=\(D\),i>^=\(i\),v=\(u\)] ++(3,0);
\end{circuit}

Entre les instants \(t\) et \(t+\odif{t}\) (durée \(\odif{t}\) infinitésimale), la charge \[\odif{q}=i\odif{t}\] a traversé le dipôle.

Cela correspond à une énergie \[\fdif{W}=u\odif{q}=ui\odif{t}.\]

Or la puissance est la dérivée de l'énergie par rapport au temps donc \[P=\odv{W}{t}=ui.\]

Soit \(P\) la puissance reçue par un dipôle en convention récepteur.

Si \(P>0\), le dipôle a un comportement récepteur.

Si \(P<0\), le dipôle a un comportement générateur.

\attention convention \(\not=\) comportement.

Prenons par exemple une résistance :

\begin{center}
\begin{tabular}{c|c|c}
& Convention récepteur & Convention générateur \\
\hline
Schéma & \begin{circuitikz}\draw (0,0) to[R,l=\(R\),i>^=\(i\),v=\(u\)] ++(3,0);\end{circuitikz} & \begin{circuitikz}\draw (0,0) to[R,l=\(R\),i>^=\(i\),v>=\(u\)] ++(3,0);\end{circuitikz} \\[1em]
Loi d'Ohm & \(u=Ri\) & \(u=-Ri\) \\[1em]
Puissance reçue & \(P=ui\) & \(P=-ui\) \\[1em]
Puissance & \(P=Ri^2\geq0\) & \(P=Ri^2\geq0\)
\end{tabular}
\end{center}

On remarque donc que peu importe la convention utilisée, une résistance a un comportement récepteur.

\subsection{Caractéristique d'un dipôle}

Généralement, l'intensité \(i\) qui traverse un dipôle dépend de la tension \(u\) à ses bornes. On a donc \[i=f\paren{u}\] où \(f\) est la caractéristique du dipôle.

\begin{tkz}
\begin{axis}[axis lines=left,
xlabel={\(u\)},
ylabel={\(i\)},
xmin=0,xmax=6,
ymin=0,ymax=5,
xtick={4},
xticklabels={\(u_M\)},
ytick={0,2.718},
yticklabels={\(0\),\(i_M\)},
xlabel style={at={(axis description cs:1,0)},anchor=north west},
ylabel style={at={(axis description cs:0,1)},anchor=south east,rotate=-90}]
\addplot[domain=0:6,samples=1000,smooth] {exp(x-3)};
\draw[dashed] (4,0) -- (4,2.718) node[below right] {\(M\)} -- (0,2.718);
\end{axis}
\end{tkz}

L'ensemble des points \(M\) appartenant à la caractéristique sont appelés des points de fonctionnement du dipôle.

\subsection{Les dipôles fondamentaux}

Un dipôle est dit passif si sa caractéristique passe par l'origine. Sinon, il est dit actif.

Un dipôle est dit linéaire si la relation entre \(i\) et \(u\) est une équation différentielle linéaire. Sinon, il est dit non-linéaire.

Un dipôle est dit symétrique si sa caractéristique est impaire.

\subsubsection{Dipôles passifs}

\underline{Résistance} (convention récepteur) :

\begin{circuit}
\draw (0,0) to[R,l^=\(R\),i>^=\(i\),v_=\(u\)] ++(3,0);
\end{circuit}

Caractéristique : loi d'Ohm \(u=Ri\)

\begin{tkz}
\begin{axis}[axis lines=middle,
xlabel={\(u\)},
ylabel={\(i\)},
xmin=-6,xmax=6,
ymin=-5,ymax=5,
xmajorticks=false,
ytick={0}]
\addplot[domain=-6:6,samples=1000,smooth] {x};
\end{axis}
\end{tkz}

C'est un dipôle passif, linéaire et symétrique.

\underline{Condensateur} (convention récepteur) :

\begin{circuit}
\draw (0,0) to[C,l^=\(C\),v_=\(u\),i>^=\(i\)] ++(3,0);
\end{circuit}

\(C\) est la capacité du condensateur (en farad : \(\unit{\farad}\)).

Caractéristique : \[i=C\odv{u}{t}\]

C'est un dipôle passif et linéaire.

On a : \[P=ui=Cu\odv{u}{t}=\odv{}{t}\paren{\dfrac{1}{2}Cu^2}=\odv{E}{t}\] où \(E\) est l'énergie stockée dans le condensateur (en joule : \(\unit{\joule}\)).

\underline{Bobine} (convention récepteur) :

\begin{circuit}
\draw (0,0) to[L,l^=\(L\),v_=\(u\),i>^=\(i\)] ++(3,0);
\end{circuit}

\(L\) est l'inductance de la bobine (en henry : \(\unit{\henry}\)).

Caractéristique : \[u=L\odv{i}{t}\]

C'est un dipôle passif et linéaire.

On a : \[P=ui=Li\odv{i}{t}=\odv{}{t}\paren{\dfrac{1}{2}Li^2}=\odv{E}{t}\] où \(E\) est l'énergie emmagasinée dans la bobine (\(E>0\), en joule).

\subsubsection{Dipôles actifs}

\underline{Générateur de tension} (convention générateur) :

\begin{circuit}
\draw (0,0) to[vsource,i<^=\(i\),v_<=\(E\)] ++(3,0);
\end{circuit}

Caractéristique : \[u=E\]

\begin{tkz}[brace/.style={thick,decorate,decoration={calligraphic brace,amplitude=7pt,raise=0.5ex}}]
\begin{axis}[axis y line=left, axis x line=middle,
xlabel={\(u\)},
ylabel={\(i\)},
xmin=0,xmax=2,
ymin=-5,ymax=5,
xmajorticks=false,
ytick={0},
ylabel style={at={(axis description cs:0,1)},anchor=south east,rotate=-90}]
\addplot[samples=1000,smooth] (1,x);
\node[below left] at (1,0) {\(E\)};
\draw[brace] (1.1,5) -- node[right=1.5ex,align=center] {comportement\\générateur} (1.1,0);
\draw[brace] (1.1,0) -- node[right=1.5ex,align=center] {comportement\\récepteur} (1.1,-5);
\end{axis}
\end{tkz}

Les piles et les batteries sont des exemples de générateurs de tension.

Comportement générateur : \[P_\text{reçue}=-ui=-Ei<0\] donc \(i>0\) : le générateur fournit du travail au milieu extérieur.

Si \(i<0\) on a : \[P_\text{reçue}=-Ei>0\] donc comportement récepteur (il pourrait par exemple s'agir du rechargement d'une batterie).

\underline{Générateur de courant} (convention générateur) :

\begin{circuit}
\draw (0,0) to[isource,i=\(i\),l_=\(\eta\),v^=\(u\)] ++(3,0);
\end{circuit}

\(\eta\) est le courant électromoteur fourni par le générateur de courant (en ampère : \(\unit{\ampere}\)).

Caractéristique : \[i=\eta\]

\begin{tkz}[brace/.style={thick,decorate,decoration={calligraphic brace,amplitude=7pt,raise=0.5ex}}]
\begin{axis}[axis y line=middle, axis x line=bottom,
xlabel={\(u\)},
ylabel={\(i\)},
xmin=-5,xmax=5,
ymin=0,ymax=2,
ymajorticks=false,
xtick={0},
xlabel style={at={(axis description cs:1,0)},anchor=north west}]
\addplot[samples=1000,smooth] (x,1);
\node[above left] at (0,1) {\(\eta\)};
\draw[brace] (5,0.9) -- node[below=1.5ex,align=center] {comportement\\générateur} (0,0.9);
\draw[brace] (0,0.9) -- node[below=1.5ex,align=center] {comportement\\récepteur} (-5,0.9);
\end{axis}
\end{tkz}

\section{Simplification des circuits}

\subsection{Lois d'association des dipôles}

\subsubsection{Résistances}

\underline{En série} : toutes les résistances sont parcourues par le même courant d'intensité \(i\).

\begin{circuit}
\draw (0,0) to[short,-*,i=\(i\)] ++(1,0) coordinate (A) -- ++(1,0) to[R,l^=\(R_1\),v_=\(u_{R_1}\)] ++(1,0) to[short,-*] ++(1,0) coordinate (A1) -- ++(1,0) to[R,l^=\(R_2\),v_=\(u_{R_2}\)] ++(1,0) to[short,-*] ++(1,0) coordinate (A2) -- ++(1,0) coordinate (B) to[open] ++(1,0) coordinate (C) -- ++(1,0) to[R,l^=\(R_n\),v_=\(u_{R_n}\)] ++(1,0) to[short,-*] ++(1,0) coordinate (An) to[short,i=\(i\)] ++(1,0);
\draw (B) -- (C) node[midway,fill=white] {\(\ldots\)};
\node[below] at (A) {\(A\)};
\node[below] at (A1) {\(A_1\)};
\node[below] at (A2) {\(A_2\)};
\node[below] at (An) {\(A_n\)};
\draw (0,-1) to[open,v=\(u\)] (13,-1);
\end{circuit}

Ce circuit possède l'équivalent suivant :

\begin{circuit}
\draw (0,0) to[short,-*,i=\(i\)] ++(1,0) coordinate (A) -- ++(1,0) to[R,l^=\(R_\eq\),v_=\(u\)] ++(1,0) to[short,-*] ++(1,0) coordinate (B) -- ++(1,0);
\node[below] at (A) {\(A\)};
\node[below] at (B) {\(B\)};
\end{circuit}

On a : \[\begin{aligned}
u&=V_A-V_B \\
&=\underbrace{V_A-V_{A_1}}_{u_{R_1}}+\underbrace{V_{A_1}-V_{A_2}}_{u_{R_2}}+\ldots-V_B \\
&=\sum_ku_{R_k}.
\end{aligned}\]

C'est la loi d'additivité des tensions.

D'après la loi d'Ohm, on a : \[\forall k,u_{R_k}=R_ki\quad\text{et}\quad u=R_\eq i.\]

Donc \[u=\sum_kR_k i=i\sum_kR_k=R_\eq i.\] D'où \[R_\eq=\sum_kR_k.\]

Donc en série, les résistances s'additionnent.

\underline{En parallèle} : toutes les résistances sont soumises à la même tension \(u\).

\begin{circuit}
\draw (0,0) to[short,i=\(i\),-*] ++(1,0) node[below left] {\(N\)} to[open] ++(3,0) -- ++(1,0);
\draw (1,3) to[R,*-,l^=\(R_1\),i>^=\(i_1\)] ++(3,0);
\draw (1,1.5) to[R,*-,l^=\(R_2\),i>^=\(i_2\)] ++(3,0);
\draw (1,-1.5) to[R,v_=\(u\),*-,l^=\(R_n\),i>^=\(i_n\)] ++(3,0);
\draw (1,3) -- (1,-1.5);
\draw (4,3) -- (4,-1.5);
\node at (2.5,0.25) {\(\vdots\)};
\end{circuit}

Ce circuit possède l'équivalent suivant :

\begin{circuit}
\draw (0,0) to[R,l^=\(R_\eq\),i>^=\(i\),v=\(u\)] ++(3,0);
\end{circuit}

On applique la loi des nœuds en \(N\) : \[\begin{aligned}
\sum_k\epsilon_ki_k=0&\iff i-i_1-i_2-\ldots-i_n=0 \\
&\iff i=\sum_ki_k.
\end{aligned}\]

D'après la loi d'Ohm, on a : \[i_k=\dfrac{u}{R_k}\quad\text{et}\quad i=\dfrac{u}{R_\eq}.\]

Donc \(\dfrac{u}{R_\eq}=\sum_k\dfrac{u}{R_k}\) donc on a : \[\dfrac{1}{R_\eq}=\sum_k\dfrac{1}{R_k}.\]

Donc en parallèle, les inverses des résistances s'additionnent.

Cas particulier avec deux résistances \(R_1\) et \(R_2\) : on a \[\dfrac{1}{R_\eq}=\dfrac{1}{R_1}+\dfrac{1}{R_2}=\dfrac{R_1+R_2}{R_1R_2}\] donc \[R_\eq=\dfrac{R_1R_2}{R_1+R_2}.\]

\subsubsection{Condensateurs}

\underline{En série} :

\begin{circuit}
\draw (0,0) to[C,i>^=\(i\),v_=\(u_1\),l^=\(C_1\)] ++(3,0) to[C,v_=\(u_2\),l^=\(C_2\)] ++(3,0) coordinate (A) -- ++(1,0) to[open] ++(1,0) coordinate (B) to[C,v_=\(u_n\),l^=\(C_n\)] ++(3,0);
\draw (A) -- (B) node[midway,fill=white] {\(\ldots\)};
\end{circuit}

Ce circuit possède l'équivalent suivant :

\begin{circuit}
\draw (0,0) to[C,l^=\(C_\eq\),v_=\(u\),i>^=\(i\)] ++(3,0);
\end{circuit}

Par la loi d'additivité des tensions, on a : \[u=\sum_ku_k.\]

D'après la caractéristique des condensateurs, on a : \[i=C_k\odv{u_k}{t}\quad\text{et}\quad i=C_\eq\odv{u}{t}.\]

Or on a : \[\dfrac{i}{C_\eq}=\odv{u}{t}=\sum_k\odv{u_k}{t}=\sum_k\dfrac{i}{C_k}\] donc \[\dfrac{1}{C_\eq}=\sum_k\dfrac{1}{C_k}.\]

Donc en série, les inverses des capacités s'additionnent.

\underline{En parallèle} :

\begin{circuit}
\draw (0,0) to[short,i=\(i\),-*] ++(1,0) node[below left] {\(N\)} to[open] ++(3,0) -- ++(1,0);
\draw (1,3) to[C,*-,l^=\(C_1\),i>^=\(i_1\)] ++(3,0);
\draw (1,1.5) to[C,*-,l^=\(C_2\),i>^=\(i_2\)] ++(3,0);
\draw (1,-1.5) to[C,v_=\(u\),*-,l^=\(C_n\),i>^=\(i_n\)] ++(3,0);
\draw (1,3) -- (1,-1.5);
\draw (4,3) -- (4,-1.5);
\node at (2.5,0.25) {\(\vdots\)};
\end{circuit}

Ce circuit possède l'équivalent suivant :

\begin{circuit}
\draw (0,0) to[C,l^=\(C_\eq\),v_=\(u\),i>^=\(i\)] ++(3,0);
\end{circuit}

On applique la loi des nœuds en \(N\) : \[i=\sum_ki_k.\]

D'après la caractéristique des condensateurs, on a : \[i_k=C_k\odv{u}{t}\quad\text{et}\quad i=C_\eq\odv{u}{t}.\]

Donc on a : \[C_\eq\odv{u}{t}=\sum_kC_k\odv{u}{t}.\]

Donc \[C_\eq=\sum_kC_k.\]

Donc en parallèle, les capacités s'ajoutent.

\subsubsection{Bobines}

\underline{En série} :

\begin{circuit}
\draw (0,0) to[L,i>^=\(i\),v_=\(u_1\),l^=\(L_1\)] ++(3,0) to[L,v_=\(u_2\),l^=\(L_2\)] ++(3,0) coordinate (A) -- ++(1,0) to[open] ++(1,0) coordinate (B) to[L,v_=\(u_n\),l^=\(L_n\)] ++(3,0);
\draw (A) -- (B) node[midway,fill=white] {\(\ldots\)};
\end{circuit}

Ce circuit possède l'équivalent suivant :

\begin{circuit}
\draw (0,0) to[L,l^=\(L_\eq\),v_=\(u\),i>^=\(i\)] ++(3,0);
\end{circuit}

Par la loi d'additivité des tensions, on a : \[u=\sum_ku_k.\]

D'après la caractéristique des bobines, on a : \[u_k=L_k\odv{i}{t}\quad\text{et}\quad u=L_\eq\odv{i}{t}.\]

Donc on a : \[L_\eq\odv{i}{t}=\sum_kL_k\odv{i}{t}.\]

Donc \[L_\eq=\sum_kL_k.\]

Donc en série, les inductances s'additionnent.

\underline{En parallèle} :

\begin{circuit}
\draw (0,0) to[short,i=\(i\),-*] ++(1,0) node[below left] {\(N\)} to[open] ++(3,0) -- ++(1,0);
\draw (1,3) to[L,*-,l^=\(L_1\),i>^=\(i_1\)] ++(3,0);
\draw (1,1.5) to[L,*-,l^=\(L_2\),i>^=\(i_2\)] ++(3,0);
\draw (1,-1.5) to[L,v_=\(u\),*-,l^=\(L_n\),i>^=\(i_n\)] ++(3,0);
\draw (1,3) -- (1,-1.5);
\draw (4,3) -- (4,-1.5);
\node at (2.5,0.25) {\(\vdots\)};
\end{circuit}

Ce circuit possède l'équivalent suivant :

\begin{circuit}
\draw (0,0) to[L,l^=\(L_\eq\),v_=\(u\),i>^=\(i\)] ++(3,0);
\end{circuit}

On applique la loi des nœuds en \(N\) : \[i=\sum_ki_k.\]

D'après la caractéristique des bobines, on a : \[u=L_k\odv{i_k}{t}\quad\text{et}\quad u=L_\eq\odv{i}{t}.\]

Donc on a : \[\dfrac{u}{L_\eq}=\odv{i}{t}=\sum_k\odv{i_k}{t}=\sum_k\dfrac{u}{L_k}.\]

Donc \[\dfrac{1}{L_\eq}=\sum_k\dfrac{1}{L_k}.\]

Donc en parallèle, les inverses des inductances s'additionnent.

\subsection{Pont diviseur de tension}

On considère le circuit suivant :

\begin{circuit}
\draw (0,0) to[short,i=\(i\)] ++(2,0) to[R,l_=\(R_1\)] ++(0,-3) coordinate (A) to[R,l_=\(R_2\),v^=\(u_2\)] ++(0,-3) -- ++(-1,0) node[eground]{} -- ++(-1,0) coordinate (B);
\draw (0,0) to[open,v=\(u_1\)] (B);
\draw (A) to[short,i=\({i^{\,\prime}=0}\)] ++(2,0);
\end{circuit}

Ce circuit possède l'équivalent suivant :

\begin{circuit}
\draw (0,0) to[short,i=\(i\)] ++(2,0) to[R,l=\(R_1+R_2\)] ++(0,-3) -- ++(-2,0) coordinate (B);
\draw (0,0) to[open,v=\(u_1\)] (B);
\end{circuit}

On a : \[u_1=\paren{R_1+R_2}i\quad\text{et}\quad u_2=R_2i.\] Donc \[i=\dfrac{u_1}{R_1+R_2}.\]

Donc on a : \[u_2=u_1\times\dfrac{R_2}{R_1+R_2}.\]

C'est la relation du pont diviseur de tension.

\subsection{Pont diviseur de courant}

On considère le circuit suivant :

\begin{circuit}
\draw (0,0) to[short,i=\(i\),-*] ++(2,0) coordinate (A) -- ++(2,0) to[R,l=\(R_2\),i>^=\(i_2\)] ++(0,-3) -- ++(-4,0) coordinate (B);
\draw (A) to[R,l=\(R_1\),i>^=\(i_1\),-*] ++(0,-3);
\draw (0,0) to[open,v=\(u\)] (B);
\end{circuit}

D'après la loi d'Ohm, on a : \[i_1=\dfrac{u}{R_1}\quad\text{et}\quad i_2=\dfrac{u}{R_2}.\]

Donc d'après la loi des nœuds, on a : \[i=i_1+i_2=u\paren{\dfrac{1}{R_1}+\dfrac{1}{R_2}}=u\paren{\dfrac{R_1+R_2}{R_1R_2}}.\]

De plus, on a \(\dfrac{i_1}{i}=\dfrac{\dfrac{1}{R_1}}{\dfrac{R_1+R_2}{R_1R_2}}=\dfrac{R_2}{R_1+r_2}\) donc on a : \[i_1=i\times\dfrac{R_2}{R_1+R_2}.\]

C'est la relation du pont diviseur de courant.

\subsection{Générateur réel}

En pratique, le générateur idéal n'existe pas. La caractéristique d'un générateur réel (expérimentale) est la suivante :

\begin{tkz}
\begin{axis}[axis lines=middle,
xlabel={\(u\)},
ylabel={\(i\)},
xmin=-6,xmax=6,
ymin=-5,ymax=5,
xtick={4},
ytick={2},
xticklabels={\(e\)},
yticklabels={\(\eta\)}]
\addplot[domain=-6:6,samples=1000,smooth] {-0.5*x+2};
\draw (2,1) -- (2,1.5) node[right] {pente : \(-r\)} -- (1,1.5);
\end{axis}
\end{tkz}

On a : \[\begin{aligned}
\dfrac{1}{r}&=\dfrac{\eta}{e} \\
i&=-\dfrac{u}{r}+\eta \\
ir&=-u+r\eta \\
u&=r\eta-ri \\
u&=\underbrace{e}_{\substack{\text{générateur} \\ \text{de tension}}}-\underbrace{ri}_{\text{résistance}}
\end{aligned}\]

On obtient donc deux équations équivalentes : \[u=e-ri\quad\text{et}\quad i=\eta-\dfrac{u}{r}.\]

Cela nous donne l'équivalence entre les deux circuits suivants :

\begin{circuit}
\draw (0,0) to[vsource,i<=\(i\),v<=\(e\)] ++(2,0) to[R, l=\(r\)] ++(2,0) to[open] ++(2,0) to[short,i<=\(i\),-*] ++(1,0) coordinate (N);
\draw (N) -- ++(0,1) coordinate (A) -- ++(0,-2) coordinate (B);
\draw (A) to[isource,v^<=\(u\),l_=\(\eta\)] ++(2,0) -- ++(0,-1) coordinate (M);
\draw (B) to[R,l=\(r\)] ++(2,0) -- ++(0,1);
\draw (M) to[short,*-] ++(1,0);
\draw (0,-0.75) to[open,v=\(u\)] (4,-0.75);
\node at (2,-2) {Modélisation de Thévenin};
\node at (8,-2) {Modélisation de Norton};
\end{circuit}

\chapter{Régime transitoire des circuits linéaires}

\minitoc

\section{Généralités}

\subsection{Régime transitoire et régime permanent}

Par définition, le régime permanent est le régime où les grandeurs électriques ont des valeurs qui sont imposées par l'excitation (\ie le générateur).

Pour passer d'un régime permanent à un autre régime permanent, il faut temporairement passer par un régime transitoire.

Du point de vue mathématique, les grandeurs électriques sont solutions d'équations différentielles. Si \(y\) est une grandeur électrique alors \(y\paren{t}=y_P\paren{t}+y_G\paren{t}\) où \(y_P\paren{t}\) est une solution particulière et \(y_G\paren{t}\) la solution générale de l'équation homogène. \(y_P\paren{t}\) caractérise le régime permanent imposé par le générateur et \(y_G\paren{t}\) caractérise le régime transitoire du circuit.

On a \(y_G\paren{t}\xrightarrow[t\to\pinf]{}0\) donc à partir d'une certaine durée, le régime transitoire s'arrête et laisse place au régime permanent.

\subsection{Rappels sur les dipôles linéaires}

\subsubsection{Condensateurs}

\begin{circuit}
\draw (0,0) to[C,i>^=\(i\),v=\(u\),l=\(C\)] ++(3,0);
\end{circuit}

Un condensateur possède deux armatures métalliques en influence et porte la charge \(q\) :

\begin{tkz}
\draw[decoration={markings,mark=at position 0.5 with {\arrow{>}}},postaction={decorate}] (0,0) -- ++(2,0);
\draw (2,-2) -- (2,2) node[above left] {\(+q\)};
\draw (3,-2) -- (3,2) node[above right] {\(-q\)};
\draw (3,0) -- (5,0);
\foreach \x in {-1.8,-1.6,...,1.6,1.8} \node[right] at (2,\x) {\tiny+};
\foreach \x in {-1.8,-1.6,...,1.6,1.8} \node[left] at (3,\x) {\tiny-};
\node[above] at (1,0) {\(i\)};
\end{tkz}

On a : \[q=Cu\] avec \(C\) la capacité du condensateur (en farad : \(\unit{\farad}\)).

Comme \(i=\odv{q}{t}\), on a : \[i=C\odv{u}{t}.\]

On a la puissance reçue : \[P=ui=Cu\odv{u}{t}=\odv{}{t}\paren{\dfrac{1}{2}Cu^2}.\]

On a l'énergie stockée dans le condensateur : \[E=\dfrac{1}{2}Cu^2.\]

Comme \(P\not=\infty\), \(u\) est continue aux bornes du condensateur.

Condensateur réel : l'isolant entre les armatures n'est pas parfait donc il existe un courant de fuite à travers le condensateur.

Modèle du condensateur réel :

\begin{circuit}
\draw (0,0) -- ++(1,0);
\draw (1,1) -- (1,-1);
\draw (1,1) to[C,l=\(C\)] ++(2,0);
\draw (1,-1) to[R,l_=\(R\)] ++(2,0);
\draw (3,1) -- (3,-1);
\draw (3,0) -- (4,0);
\end{circuit}

avec \(R\approx\SI{e8}{\ohm}\).

Ordre de grandeur de la capacité : \(\unit{\micro\farad}\) à \(\unit{\nano\farad}\).

\subsubsection{Bobines}

\begin{circuit}
\draw (0,0) to[L,i>^=\(i\),v=\(u\),l=\(L\)] ++(3,0);
\end{circuit}

Ordre de grandeur de l'inductance : \(\unit{\milli\henry}\).

On a \[u=L\odv{i}{t}.\]

On a donc la puissance reçue \[P=ui=Li\odv{i}{t}=\odv{}{t}\paren{\dfrac{1}{2}Li^2}.\]

On a donc l'énergie emmagasinée dans la bobine : \[E=\dfrac{1}{2}Li^2.\]

De plus, on a \(P\not=\infty\) donc \(i\) est continue dans une bobine.

Bobine réelle : on fabrique une bobine en enroulant un fil autour d'un axe. Le fil étant résistif, le modèle de la bobine réelle est le suivant :

\begin{circuit}
\draw (0,0) to[L,l=\(L\),v=\(u\),i>^=\(i\)] ++(3,0) to [R,l=\(R\)] ++(3,0);
\end{circuit}

Avec \(R\) entre \(\SI{10}{\ohm}\) et \(\SI{100}{\ohm}\).

\section{Circuit RC série}

\subsection{Régime libre}

On considère le circuit suivant :

\begin{circuit}
\draw (0,0) to[cosw,i>^=\(i\),l=\(K\)] ++(3,0) to[R,l_=\(R\),v^=\(u_R\)] ++(0,-3) -- ++(-3,0) to[C,l_=\(C\),v^>=\(u_C\)] ++(0,3);
\end{circuit}

À l'instant \(t=0^-\), le condensateur est chargé sous une tension \(u_C\paren{t=0^-}=E\). À \(t=0\), on ferme \(K\).

D'après la loi des mailles, on a : \[u_C-u_R=0.\]

D'après les caractéristiques des dipôles, on a : \[i=-C\odv{u_C}{t}\quad\text{et}\quad u_R=Ri.\]

D'où \(u_C-Ri=0\) et donc : \[u_C+RC\odv{u_C}{t}=0.\]

On a une équation différentielle linéaire d'ordre 1 à coefficients constants et homogène. On la réécrit comme suit : \[\odv{u_C}{t}+\dfrac{1}{RC}u_C=0.\]

On pose \[\tau=RC\] la constante de temps du circuit RC.

On a : \[\begin{aligned}
\odv{u_C}{t}+\dfrac{1}{\tau}u_C&=0 \\
\odv{u_C}{t}&=-\dfrac{1}{\tau}u_C \\
\int\dfrac{\odif{u_C}}{u_C}&=\int-\dfrac{\odif{t}}{\tau} \\
\ln u_C&=-\dfrac{t}{\tau}+\mu \\
u_C\paren{t}&=\e{\mu}\e{-\frac{t}{\tau}}.
\end{aligned}\]

On pose \(\lambda=\e{\mu}\) et on obtient : \[u_C\paren{t}=\lambda\e{-\frac{t}{\tau}}.\]

On détermine \(\lambda\) avec la condition initiale : \(u_C\paren{t=0^-}=E\).

Comme la tension aux bornes d'un condensateur est continue, on a : \[u_C\paren{t=0^-}=u_C\paren{t=0^+}=E.\]

Donc \(\lambda\e{0}=E\) donc \[\lambda=E.\]

Finalement, on a : \[u_C\paren{t}=E\e{-\frac{t}{\tau}}.\]

De plus, on a : \[\begin{aligned}
i\paren{t}&=-C\odv{u_C}{t} \\
&=\dfrac{1}{\tau}EC\e{-\frac{t}{\tau}} \\
&=\dfrac{EC}{RC}\e{-\frac{t}{\tau}} \\
&=\dfrac{E}{R}\e{-\frac{t}{\tau}}
\end{aligned}\]

On obtient les graphes suivants (RP et RT signifiant respectivement \guillemets{régime permanent} et \guillemets{régime transitoire}) :

\begin{center}
\begin{tikzpicture}
\begin{axis}[axis lines=middle,
xlabel={\(t\)},
ylabel={\(u_C\)},
xmin=-4,xmax=16,
ymin=0,ymax=7,
xtick={3},
xticklabels={\(\tau\)},
ytick={1.84,5},
yticklabels={\(\dfrac{1}{\e{}}E\),\(E\)},
clip=false]
\addplot[domain=0:16,samples=1000,smooth,thick,blue] {5*exp(-x/3)};
\addplot[domain=-4:0,samples=1000,smooth,thick,blue] {5};
\draw (0,5) -- (3,0);
\draw[dashed] (3,0) -- (3,1.84) -- (0,1.84);
\draw[dashed,->] (-4,6) -- (0,6);
\draw[<->] (0,6) -- (12,6);
\draw[<-] (12,6) -- (16,6);
\node[above] at (-2,6) {RP};
\node[above] at (6,6) {RT};
\node[above] at (14,6) {RP};
\end{axis}
\end{tikzpicture}
\hskip10pt
\begin{tikzpicture}
\begin{axis}[axis lines=middle,
xlabel={\(t\)},
ylabel={\(i\)},
xmin=-4,xmax=16,
ymin=0,ymax=7,
xtick={0},
xticklabels={\(0\)},
ytick={5},
yticklabels={\(\dfrac{E}{R}\)},
clip=false]
\addplot[domain=0:16,samples=1000,smooth,thick,blue] {5*exp(-x/3)};
\draw[{}-{Arc Barb [length=0.1cm]},thick,blue] (-4,0) -- (0,0);
\node[below] at (0,0) {\(0\)};
\end{axis}
\end{tikzpicture}
\end{center}

\(\tau\) est appelé constante de temps du circuit, temps caractéristique ou temps de relaxation du circuit.

Temps de réponse à 5\% : on cherche \(t_\text{5\%}\) tel que \(u_C\paren{t_\text{5\%}}\) diffère de 5\% de sa valeur finale : \[u_C\paren{t_\text{5\%}}=\num{0.05}E=E\e{-\frac{t_\text{5\%}}{\tau}}.\]

On a : \[\begin{aligned}
\num{0.05}&=\e{-\frac{t_\text{5\%}}{\tau}} \\
\ln\num{0.05}&=-\dfrac{t_\text{5\%}}{\tau} \\
t_\text{5\%}&=\tau\ln20 \\
&=3\tau.
\end{aligned}\]

On trouve de même : \[t_\text{10\%}=5\tau.\]

Finalement, on obtient que le régime transitoire dure de \(3\tau\) à \(5\tau\).

\subsection{Réponse à un échelon de tension}

On considère le circuit suivant :

\begin{circuit}
\draw (0,0) to[cosw,i>^=\(i\),l=\(K\)] ++(3,0) to[C,l_=\(C\),v^=\(u_C\)] ++(3,0) to[R,l_=\(R\),v^=\(u_R\)] ++(0,-3) -- ++(-6,0) to[vsource,v^=\(E\)] ++(0,3);
\end{circuit}

À \(t=0^-\), le condensateur est de charge \(u_C\paren{t=0^-}=0\). À \(t=0\), on ferme \(K\).

D'après la loi des mailles, on a : \[u_R+u_C-E=0.\]

D'après les caractéristiques des dipôles, on a : \[u_R=Ri\quad\text{et}\quad i=C\odv{u_C}{t}.\]

D'où l'équation différentielle : \[RC\odv{u_C}{t}+u_C-E=0.\]

On pose \(\tau=RC\) et on obtient : \[\odv{u_C}{t}+\dfrac{1}{\tau}u_C=\dfrac{E}{\tau}.\]

Donc \[u_C\paren{t}=u_P+u_H\paren{t}\] où \begin{description}
\item \(u_P\) est une solution particulière de l'équation différentielle
\item \(u_H\) est la solution de l'équation homogène. \\
\end{description}

On a l'équation homogène \(\odv{u_C}{t}+\dfrac{1}{\tau}u_C=0\) donc \[u_H\paren{t}=\lambda\e{-\frac{t}{\tau}}.\]

\attention On ne détermine pas \(\lambda\) immédiatement car \(u_H\) n'est pas solution de l'équation complète !

De plus, on a \[u_P\paren{t}=\cte\] car \(\dfrac{E}{\tau}\) est une constante.

On a : \[\begin{aligned}
\odv{u_P}{t}+\dfrac{1}{\tau}u_P&=\dfrac{E}{\tau} \\
\dfrac{1}{\tau}u_P&=\dfrac{E}{\tau} \\
u_P&=E.
\end{aligned}\]

Finalement, on a : \[u_C\paren{t}=E+\lambda\e{-\frac{t}{\tau}}.\]

On détermine \(\lambda\) avec la condition initiale : \(u_C\paren{t=0^-}=0\).

Comme la tension aux bornes d'un condensateur est continue, on a : \[u_C\paren{t=0^-}=u_C\paren{t=0^+}=0.\]

Donc \(E+\lambda\e{0}=0\) donc \[\lambda=-E.\]

Finalement, on a : \[u_C\paren{t}=E-E\e{-\frac{t}{\tau}}=E\paren{1-\e{-\frac{t}{\tau}}}.\]

De plus, on a : \[i\paren{t}=\dfrac{E}{R}\e{-\frac{t}{\tau}}.\]

On obtient les graphes suivants :

\begin{center}
\begin{tikzpicture}
\begin{axis}[axis lines=middle,
xlabel={\(t\)},
ylabel={\(u_C\)},
xmin=-4,xmax=16,
ymin=0,ymax=7,
xtick={3},
xticklabels={\(\tau\)},
ytick={3.16,5},
yticklabels={\(1-\dfrac{1}{\e{}}E\),\(E\)},
clip=false]
\addplot[domain=0:16,samples=1000,smooth,thick,blue] {5*(1-exp(-x/3))};
\addplot[domain=-4:0,samples=1000,smooth,thick,blue] {0};
\draw[dashed] (0,5) -- (16,5);
\draw (0,0) -- (3,5);
\draw[dashed] (3,0) -- (3,5);
\draw[dashed] (3,3.16) -- (0,3.16);
\draw[dashed,->] (-4,6) -- (0,6);
\draw[<->] (0,6) -- (12,6);
\draw[<-] (12,6) -- (16,6);
\node[above] at (-2,6) {RP};
\node[above] at (6,6) {RT};
\node[above] at (14,6) {RP};
\node[below] at (0,0) {\(0\)};
\end{axis}
\end{tikzpicture}
\hskip10pt
\begin{tikzpicture}
\begin{axis}[axis lines=middle,
xlabel={\(t\)},
ylabel={\(i\)},
xmin=-4,xmax=16,
ymin=0,ymax=7,
xtick={0},
xticklabels={\(0\)},
ytick={5},
yticklabels={\(\dfrac{E}{R}\)},
clip=false]
\addplot[domain=0:16,samples=1000,smooth,thick,blue] {5*exp(-x/3)};
\draw[{}-{Arc Barb [length=0.1cm]},thick,blue] (-4,0) -- (0,0);
\node[below] at (0,0) {\(0\)};
\end{axis}
\end{tikzpicture}
\end{center}

\subsection{Aspect énergétique}

On considère le circuit suivant :

\begin{circuit}
\draw (0,0) to[cosw,i>^=\(i\),l=\(K\)] ++(3,0) to[C,l_=\(C\),v^=\(u_C\)] ++(3,0) to[R,l_=\(R\),v^=\(u_R\)] ++(0,-3) -- ++(-6,0) to[vsource,v^=\(E\)] ++(0,3);
\end{circuit}

On a : \[u_C\paren{t}=E\paren{1-\e{-\frac{t}{\tau}}}.\]

D'après la loi des mailles on a : \[\begin{WithArrows}
u_C+Ri&=E \Arrow{\(\times i\)} \\
iu_C+Ri^2&=Ei \\
Cu_C\odv{u_C}{t}+Ri^2&=Ei.
\end{WithArrows}\] où \begin{description}
\item \(Ei\) : puissance fournie par le générateur
\item \(Ri^2\) : puissance reçue par la résistance
\item \(Cu_C\odv{u_C}{t}=\odv{}{t}\paren{\dfrac{1}{2}Cu_C^2}\) : puissance reçue par le condensateur. \\
\end{description}

On en déduit \(E=\dfrac{1}{2}Cu_C^2\) l'énergie stockée par le condensateur.

Pour passer de la puissance à l'énergie, on intègre : \[\int_0^{\pinf} Ei\odif{t}=\int_0^{\pinf} Ri^2\odif{t}+\int_0^{\pinf}\odv{}{t}\paren{\dfrac{1}{2}Cu_C^2}\odif{t}.\]

On a : \[\int_0^{\pinf}Ei\odif{t}=\int_{q\paren{t=0}}^{q\paren{t=\pinf}}E\odif{q}\] car \(i=\odv{q}{t}\) où \(q\) est la charge portée par le condensateur.

Or on a : \[q\paren{t=0}=Cu_C\paren{t=0}=CE\paren{1-\e{0}}=0\quad\text{et}\quad q\paren{t=\pinf}=CE.\]

Donc on a : \[\int_0^{\pinf}Ei\odif{t}=\int_0^{CE}E\odif{q}=CE^2.\]

Donc \(CE^2\) est l'énergie fournie par le générateur entre \(t=0\) et \(t=\pinf\) (\ie au cours de la charge du circuit).

De plus, on a : \[\int_0^{\pinf}\odv{}{t}\paren{\dfrac{1}{2}Cu_C^2}\odif{t}=\int_0^{\frac{1}{2}CE^2}\odif{\paren{\dfrac{1}{2}Cu_C^2}}=\dfrac{1}{2}CE^2.\]

Finalement, on a : \[\int_0^{\pinf}Ri^2\odif{t}=\int_0^{\pinf}Ei\odif{t}-\int_0^{\pinf}\odv{}{t}\paren{\dfrac{1}{2}Cu_C^2}\odif{t}=CE^2-\dfrac{1}{2}CE^2=\dfrac{1}{2}CE^2.\]

D'où le bilan d'énergie entre \(t=0\) et \(t=\pinf\) : \begin{description}
\item Énergie fournie par le générateur : \(CE^2\)
\item Énergie reçue par le condensateur : \(\dfrac{1}{2}CE^2\)
\item Énergie reçue par la résistance (dissipée par effet Joule) : \(\dfrac{1}{2}CE^2\). \\
\end{description}

On définit le rendement : \[\eta=\dfrac{\text{énergie utile}}{\text{énergie fournie}}.\]

Ici, on a : \[\eta=\dfrac{\frac{1}{2}CE^2}{CE^2}=\dfrac{1}{2}=50\%.\]

\note{À FINIR}

\chapter{Régime sinusoïdal forcé et résonance}

\minitoc

\note{À VENIR}

\part{Mécanique}

\chapter{Cinématique}

\minitoc

\note{À VENIR}

\chapter{Dynamique du point en référentiel galiléen}

\minitoc

\note{À VENIR}

\chapter{Énergie d'un point matériel}

\minitoc

\note{À VENIR}

\chapter{Mouvement de particules chargées dans des champs électriques et magnétiques}

\minitoc

\note{À VENIR}

\chapter{Moment cinétique}

\minitoc

\note{À VENIR}

\chapter{Mouvement dans un champ de force centrale}

\minitoc

\note{À VENIR}

\chapter{Solide en rotation}

\minitoc

\note{À VENIR}

\part{Électromagnétisme}

\chapter{Le champ magnétique et ses effets}

\minitoc

\note{À VENIR}

\chapter{La loi de l'induction}

\minitoc

\note{À VENIR}

\chapter{Induction de Neumann}

\minitoc

\note{À VENIR}

\chapter{Induction de Lorentz}

\minitoc

\note{À VENIR}
\end{document}