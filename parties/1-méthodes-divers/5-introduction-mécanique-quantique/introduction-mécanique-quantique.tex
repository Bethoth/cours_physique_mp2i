\chapter{Introduction à la mécanique quantique}

\minitoc

\section*{Introduction}
\addcontentsline{toc}{section}{Introduction}

La notion d'atome est déjà bien établie et, grâce à diverses expériences, on connaît les constituants principaux de la matière :

\begin{itemize}
    \item Crookes et Perrin : expériences du rayonnement cathodique \(\imp\) mise en évidence des électrons ; \\
    \item Thomson : déviation de faisceaux d'électrons dans des champs électriques et magnétiques \(\imp\) détermination du rapport \(\dfrac{q}{m_e}\) ; \\
    \item Millikan (1909) : chute de goutelettes d'huile dans un champ électrique \(\imp\) détermination de \(q\) : \[q=-e=\SI{-1.602e-19}{\C}\qquad m_e=\SI{9.1e-31}{\kilo\gram}\]~
    \item Chadwick et Goldstein : détermination des caractéristiques des protons et des neutrons : \[q_p=e\qquad q_n=0\qquad m_p\approx m_n\approx\num{1840}m_e=\SI{1.67e-27}{\kilo\gram}.\]
\end{itemize}

Au cours du temps, divers modèles d'atomes ont été proposés :

\begin{itemize}
    \item 1901 : Perrin propose un modèle planétaire avec un \guillemets{soleil} de charge positive autour duquel gravitent des corpuscules minuscules de charge négative ; \\
    \item 1903 : Thomson propose un modèle globulaire : l'atome serait une sphère d'électricité positive à l'intérieur de laquelle gravitent les électrons (\guillemets{pudding} de Thomson) ; \\
    \item 1911 : l'expérience de Rutherford (étude de la déviation de particules \(\alpha\) chargées positivement et traversant une mince feuille d'or) met en évidence le caractère lacunaire de l'atome, ce qui permet de rejeter le modèle de Thomson. Rutherford propose un modèle planétaire avec des orbites circulaires pour les électrons autour du noyau.
\end{itemize}

Ordres de grandeur des rayons : \[\text{noyau : }r\approx\SI{e-15}{\meter}\qquad\text{atome : }R\approx\SI{e-10}{\meter}.\]

Problèmes rencontrés :

\begin{itemize}
    \item le modèle de Rutherford est a priori assez intéressant, mais un électron en accélération centrale doit, d'après la physique classique, émettre un spectre continu d'énergie, donc perdre de l'énergie et finir sur le noyau ! \\
    \item tous ces modèles sont incapables de décrire de façon satisfaisante l'infiniment petit et notamment la quantification de l'énergie des atomes.
\end{itemize}